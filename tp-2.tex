% % Apunte de modulos basicos
%
\documentclass[a4paper,10pt]{article}
\usepackage[paper=a4paper, hmargin=1.5cm, bottom=1.5cm, top=3.5cm]{geometry}
\usepackage[latin1]{inputenc}
\usepackage[T1]{fontenc}
\usepackage[spanish]{babel}
\usepackage{fancyhdr}
\usepackage{lastpage}
\usepackage{xspace}
\usepackage{xargs}
\usepackage{ifthen}
\usepackage{aed2-tad,aed2-symb,aed2-itef}
\usepackage{algorithmicx, algpseudocode, algorithm, caratula, color, latex-lucas}
\usepackage[colorlinks=true, linkcolor=blue]{hyperref}
\usepackage[ddmmyyyy]{datetime}
\usepackage{epigraph}
\usepackage[spanish]{babel}
\selectlanguage{spanish}

\setlength{\epigraphwidth}{0.35\linewidth}
\setlength{\epigraphrule}{0pt}
\renewcommand*{\textflush}{flushright}
\renewcommand*{\epigraphsize}{\normalsize\itshape}

\hypersetup{%
 % Para que el PDF se abra a p�gina completa.
 pdfstartview= {FitH \hypercalcbp{\paperheight-\topmargin-1in-\headheight}},
 pdfauthor={Grupo 15},
 pdfkeywords={Trabajo pr�ctico 2: Dise�o - Grupo 15},
 pdftitle={Trabajo pr�ctico 2: Dise�o - Grupo 15},
 pdfsubject={Trabajo pr�ctico 2: Dise�o - Grupo 15}
}

%%%%%%%%%%%%%%%%%%%%%%%%%%%%%%%%%%%%%%%%%%%%%%%%%
% PARAMETROS A SER MODIFICADOS
%%%%%%%%%%%%%%%%%%%%%%%%%%%%%%%%%%%%%%%%%%%%%%%%%

%cuatrimestre de acuerdo a la opcion
\newcommand{\Cuatrimestre}{$1^\mathrm{er}$ cuatrimestre de 2016 (compilado $\today$)}
%\newcommand{\Cuatrimestre}{$1^\mathrm{er}$ cuatrimestre de 2012 (compilado 08/05/2012)}
%\renewcommand{\Cuatrimestre}{$2^\mathrm{do}$ cuatrimestre de 2010 (compilado 13/05/2011)}

%%%%%%%%%%%%%%%%%%%%%%%%%%%%%%%%%%%%%%%%%%%%%%%%%
% OTRAS OPCIONES QUE NO HAY QUE MODIFICAR
%%%%%%%%%%%%%%%%%%%%%%%%%%%%%%%%%%%%%%%%%%%%%%%%%

%opening
\title{Trabajo Pr�ctico 2}
\author{Grupo 15}
\date{\Cuatrimestre}

% Acomodo fancyhdr.
\pagestyle{fancy}
\thispagestyle{fancy}
\lhead{Algoritmos y Estructuras de Datos II}
\rhead{\Cuatrimestre}
\cfoot{\thepage /\pageref{LastPage}}
\renewcommand{\footrulewidth}{0.4pt}
\setlength{\headheight}{13pt}

%%%%%%%%%%%%%%%%%%%%%%%%%%%%%%%%%%%%%%%%%%%%%%%%%%%%%%%%%%%%
% COMANDOS QUE ALGUN DIA PUEDAN FORMAR UN PAQUETE.
%%%%%%%%%%%%%%%%%%%%%%%%%%%%%%%%%%%%%%%%%%%%%%%%%%%%%%%%%%%%
\newcommand{\moduloNombre}[1]{\textbf{#1}}

\let\NombreFuncion=\textsc
\let\TipoVariable=\texttt
\let\ModificadorArgumento=\textbf
\newcommand{\res}{$res$\xspace}

\newcommandx{\TipoFuncion}[3]{%
  \NombreFuncion{#1}(#2) \ifx#3\empty\else $\to$ \res\,: \TipoVariable{#3}\fi%
}
\newcommand{\In}[2]{\ModificadorArgumento{in} \ensuremath{#1}\,: \TipoVariable{#2}\xspace}
\newcommand{\Out}[2]{\ModificadorArgumento{out} \ensuremath{#1}\,: \TipoVariable{#2}\xspace}
\newcommand{\Inout}[2]{\ModificadorArgumento{in/out} \ensuremath{#1}\,: \TipoVariable{#2}\xspace}
\newcommand{\Aplicar}[2]{\NombreFuncion{#1}(#2)}

\newlength{\IntFuncionLengthA}
\newlength{\IntFuncionLengthB}
\newlength{\IntFuncionLengthC}
%InterfazFuncion(nombre, argumentos, valor retorno, precondicion, postcondicion, complejidad, descripcion, aliasing)
\newcommandx{\InterfazFuncion}[9][4=true,6,7,8,9]{%
  \hangindent=\parindent
  \TipoFuncion{#1}{#2}{#3}\\%
  \textbf{Pre} $\equiv$ \{#4\}\\%
  \textbf{Post} $\equiv$ \{#5\}%
  \ifx#6\empty\else\\\textbf{Complejidad:} #6\fi%
  \ifx#7\empty\else\\\textbf{Descripci�n:} #7\fi%
  \ifx#8\empty\else\\\textbf{Aliasing:} #8\fi%
  \ifx#9\empty\else\\\textbf{Requiere:} #9\fi%
}

\newenvironment{Interfaz}{%
  \parskip=2ex%
  \noindent\textbf{\Large Interfaz}%
  \par%
}{}

\newenvironment{Representacion}{%
  \vspace*{2ex}%
  \noindent\textbf{\Large Representaci�n}%
  \vspace*{2ex}%
}{}

\newenvironment{Algoritmos}{%
  \vspace*{2ex}%
  \noindent\textbf{\Large Algoritmos}%
  \vspace*{2ex}%
}{}


\newcommand{\Titulo}[1]{
  \vspace*{1ex}\par\noindent\textbf{\large #1}\par
}

\newenvironmentx{Estructura}[2][2={estr}]{%
  \par\vspace*{2ex}%
  \TipoVariable{#1} \textbf{se representa con} \TipoVariable{#2}%
  \par\vspace*{1ex}%
}{%
  \par\vspace*{2ex}%
}%

\newboolean{EstructuraHayItems}
\newlength{\lenTupla}
\newenvironmentx{Tupla}[1][1={estr}]{%
    \settowidth{\lenTupla}{\hspace*{3mm}donde \TipoVariable{#1} es \TipoVariable{tupla}$($}%
    \addtolength{\lenTupla}{\parindent}%
    \hspace*{3mm}donde \TipoVariable{#1} es \TipoVariable{tupla}$($%
    \begin{minipage}[t]{\linewidth-\lenTupla}%
    \setboolean{EstructuraHayItems}{false}%
}{%
    $)$%
    \end{minipage}
}

\newcommandx{\tupItem}[3][1={\ }]{%
    %\hspace*{3mm}%
    \ifthenelse{\boolean{EstructuraHayItems}}{%
        ,#1%
    }{}%
    \emph{#2}: \TipoVariable{#3}%
    \setboolean{EstructuraHayItems}{true}%
}

\newcommandx{\RepFc}[3][1={estr},2={e}]{%
  \tadOperacion{Rep}{#1}{bool}{}%
  \tadAxioma{Rep($#2$)}{#3}%
}%

\newcommandx{\Rep}[3][1={estr},2={e}]{%
  \tadOperacion{Rep}{#1}{bool}{}%
  \tadAxioma{Rep($#2$)}{true \ssi #3}%
}%

\newcommandx{\Abs}[5][1={estr},3={e}]{%
  \tadOperacion{Abs}{#1/#3}{#2}{Rep($#3$)}%
  \settominwidth{\hangindent}{Abs($#3$) \igobs #4: #2 $\mid$ }%
  \addtolength{\hangindent}{\parindent}%
  Abs($#3$) \igobs #4: #2 $\mid$ #5%
}%

\newcommandx{\AbsFc}[4][1={estr},3={e}]{%
  \tadOperacion{Abs}{#1/#3}{#2}{Rep($#3$)}%
  \tadAxioma{Abs($#3$)}{#4}%
}%


\newcommand{\DRef}{\ensuremath{\rightarrow}}

\begin{document}

% Estos comandos deben ir antes del \maketitle
\materia{Algoritmos y Estructuras de Datos II} % obligatorio
\submateria{Primer Cuatrimestre de 2016} % opcional
\tituloTP{Trabajo Pr�ctico 2} % obligatorio
\subtitulo{Dise�o} % opcional
\grupo{Grupo 15} % opcional 

\integrante{Alliani, Federico}{183/15}{fedealliani@gmail.com}
\integrante{Almada Canosa, Mat\'ias Ezequiel}{140/15}{matias.almada.canosa@gmail.com}
\integrante{Lancioni, Gian Franco}{234/15}{glancioni@dc.uba.ar}
\integrante{Raposeiras, Lucas Dami\'an}{034/15}{lucas.raposeiras@outlook.com}

\maketitle

% compilar 2 veces para actualizar las referencias
\tableofcontents

\pagebreak
%\newpage

\section{M�dulo Dato}

\begin{Interfaz}
  
  \textbf{se explica con}: \tadNombre{Dato}.

  \textbf{g�neros}: \TipoVariable{dato}.

  \Titulo{Operaciones b�sicas de dato}

  \InterfazFuncion{DatoString}{\In{s}{string}}{dato}
  [true]  
  {$res$ \igobs datoString($s$)}
  [$O\big(long(s)\big)$]
  [genera un dato con el string $s$.]
  
  \InterfazFuncion{DatoNat}{\In{n}{nat}}{dato}
  [true]
  {$res$ $\igobs$ datoNat($n$)}
  [$O(1)$]
  [genera un dato con el nat $n$.]

  \InterfazFuncion{Nat?}{\In{d}{dato}}{bool}
  [true]
  {$res$ $\igobs$ nat?($d$)}
  [$O(1)$]
  [devuelve \texttt{true} si el dato ingresado es del tipo nat.]
  
  \InterfazFuncion{ValorNat}{\In{d}{dato}}{nat}
  [nat?($d$)]
  {$res$ $\igobs$ valorNat($d$)}
  [$O(1)$]
  [devuelve el valor del nat del dato $d$.]
    
  \InterfazFuncion{ValorStr}{\In{d}{dato}}{string}
  [$\neg$nat?($d$)]
  {alias$\big(res$ $\igobs$ valorStr($d$)$\big)$}
  [$O(1)$]
  [devuelve el valor del string del dato $d$.]
  [devuelve el string por referencia. $res$ no es modificable.]
  
  \InterfazFuncion{MismoTipo?}{\In{d_1}{dato}, \In{d_2}{dato}}{bool}
  [true]
  {$res \igobs$ \big(nat?($d_1$) $\igobs$ nat?($d_2$)\big)}
  [$O(1)$]
  [devuelve \texttt{true} si $d_1$ y $d_2$ son del mismo tipo.]
  
  \InterfazFuncion{String?}{\In{d}{dato}}{bool}
  [true]
  {$res \igobs$ $\neg$nat?($d$)}
  [$O(1)$]
  [devuelve \texttt{true} si el dato ingresado es del tipo string.]
  
  \InterfazFuncion{Min}{\In{cs}{conj(dato)}}{dato}
  [$\neg$vac�a?($s$) $\land$ ($\forall$ $d_1$, $d_2$ : dato) $\Big(\big($est�?($d_1$, $s$) \ $\land$ est�?($d_2$, $s$)$\big)$ $\implies$ mismoTipo?($d_1$, $d_2$)$\Big)$]
  {$res$ \igobs min($cs$)}
  [$O(\#cs * L)$, donde $L$ es la longitud del string m�s largo.]
  [devuelve el m�nimo del conjunto de datos.]
  [$res$ no es modificable]
  
  \InterfazFuncion{Max}{\In{cs}{conj(dato)}}{dato}
  [$\neg$vac�a?($s$) $\land$ ($\forall$ $d_1$, $d_2$ : dato) $\Big(\big($est�?($d_1$, $s$) \ $\land$ est�?($d_2$, $s$)$\big)$ $\implies$ mismoTipo?($d_1$, $d_2$)$\Big)$]
  {$res$ \igobs max($cs$)}
  [$O(\#cs * L)$, donde $L$ es la longitud del string m�s largo.]
  [devuelve el m�ximo del conjunto de datos.]
  [$res$ no es modificable]
  
  \InterfazFuncion{$\argumento$ $\le$ $\argumento \ $}{\In{d_1}{dato}, \In{d_2}{dato}}{bool}
  [mismoTipo?($d_1$, $d_2$)]
  {$\Big($nat?($d_1$) $\implies$ $\big(res$ $\igobs$ (valorNat($d_1$) $\le_{nat}$ valorNat($d_2$))$\big)\Big)$ $\yluego$ \\
  $\Big($string?($d_1$) $\implies$ $\big(res$ $\igobs$ (valorStr($d_1$) $\le_{string}$ valorStr($d_2$))$\big)\Big)$}
  [$O\big(min\{|valorStr(d_1)|, |valorStr(d_2)|\}\big)$]
  [devuelve \texttt{true} si $d_1$ $\le$ $d_2$.]
  
  \InterfazFuncion{$\argumento$ = $\argumento \ $}{\In{d_1}{dato}, \In{d_2}{dato}}{bool}
  [true]
  {$\Big($nat?($d_1$) $\implies$ $\big(res$ $\igobs$ (valorNat($d_1$) $\igobs$ valorNat($d_2$))$\big)\Big)$ $\yluego$ \\
  $\Big($string?($d_1$) $\implies$ $\big(res$ $\igobs$ (valorStr($d_1$) $\igobs$ valorStr($d_2$)$)\big)\Big)$}
  [$O\big(min\{|valorStr(d_1)|, |valorStr(d_2)|\}\big)$]
  [devuelve \texttt{true} si $d_1$ = $d_2$.]

  \InterfazFuncion{Copiar}{\In{d}{dato}}{dato}
  [true]
  {$res$ $\igobs$ $d$}
  [nat?($d$) $\implies$ $O(1)$ \\
  $\neg$nat?($d$) $\implies$ $O(|valorStr(d)|)$]
  [devuelve una copia del elemento $d$.]
  

\end{Interfaz}

\begin{Representacion}
  
  \Titulo{Representaci�n de dato}

  \begin{Estructura}{dato}[estrDato]
    \begin{Tupla}[estrDato]
      \tupItem{nat?}{bool}%
      \tupItem{valorSrt}{string}%
      \tupItem{valorNat}{nat}%
    \end{Tupla}
  \end{Estructura}

  \Rep[estrDato][e]{$\Big(\big(e$.nat? $\implies$ ($e$.valorStr $\igobs$ "vacio")$\big)$ $\land$ $\big(\neg e$.nat? $\implies$ ($e$.valorNat $\igobs$ 0)$\big)\Big)$}

  \Statex \underline{Justificaci�n:} {Si bien podr�amos ser menos restrictivos, esto nos va a permitir considerar el costo de copiar datos nat como $O(1)$}
	
  ~
 
  \Abs[estrDato]{dato}[e]{d}{Nat?($d$) $\igobs$ $e$.nat? $\yluego$ $\big(e$.nat? $\implies$ (valorNat($d$) $\igobs$ $e$.valorNat) $\land$ \\ $\neg$($e$.nat?) $\implies$ (valorStr($d$) $\igobs$ $e$.valorStr)$\big)$}

  ~

\end{Representacion}

\bigskip

\begin{Algoritmos}

\medskip
	
 \Titulo{Algoritmos de dato}
  	\medskip
  
\begin{algorithm}[H]{\textbf{iDatoString}(\In{s}{string}) $\to$ $res$ : estrDato}
    	\begin{algorithmic}
       \State $res.nat? \gets false$         \Comment $O(1)$
       \State $res.valorStr \gets copiar(s)$ \comentariocompl{complejidad heredada de Copiar de Vector($\alpha$)}{$\displaystyle O\left(\sum_{i=1}^{long(s)}copy({s[i]})\right)$}
       \State $res.valorNat \gets 0$         \Comment $O(1)$

      \medskip
      \Statex \underline{Complejidad:} {$O\big(long(s)\big)$}
    \Statex \underline{Justificaci�n:} {$O(1)$ + $\displaystyle O\left(\sum_{i=1}^{long(s)}copy({s[i]})\right)$ + $O(1)$ = $\displaystyle O\left(\sum_{i=1}^{long(s)}copy({s[i]})\right)$ = $\displaystyle\sum_{i=1}^{long(s)}O\big(copy({s[i]})\big)$ \\
    \somequad = $\displaystyle\sum_{i=1}^{long(s)}O(1)$ = long($s$) * $O(1)$ = $O\big(long(s)\big)$}

      \end{algorithmic}
\end{algorithm}

\begin{algorithm}[H]{\textbf{iDatoNat}(\In{n}{nat}) $\to$ $res$ : estrDato}
  \begin{algorithmic}
    \State $res.nat? \gets true$           \Comment $O(1)$
    \State $res.valorStr \gets "vacio"$  \Comment $O\big(long("vacio")\big) = O(1)$
    \State $res.valorNat \gets n$          \Comment $O(1)$

    \medskip
    \Statex \underline{Complejidad:} {$O(1)$}
    \Statex \underline{Justificaci�n:} {Todas las funciones llamadas tienen complejidad $O(1)$.}

  \end{algorithmic}
\end{algorithm}

\begin{algorithm}[H]{\textbf{iNat?}(\In{e}{estrDato}) $\to$ $res$ : bool}
  \begin{algorithmic}
    \State $res \gets e.nat?$           \Comment $O(1)$
       
    \medskip
    \Statex \underline{Complejidad:} {$O(1)$}
    \Statex \underline{Justificaci�n:} {Todas las funciones llamadas tienen complejidad $O(1)$.}

  \end{algorithmic}
\end{algorithm}

\begin{algorithm}[H]{\textbf{iValorNat}(\In{e}{estrDato}) $\to$ $res$ : nat}
  \begin{algorithmic}
    \State $res \gets e.valorNat$           \Comment $O(1)$
       
    \medskip
    \Statex \underline{Complejidad:} {$O(1)$}
    \Statex \underline{Justificaci�n:} {Todas las funciones llamadas tienen complejidad $O(1)$.}

  \end{algorithmic}
\end{algorithm}

\begin{algorithm}[H]{\textbf{iValorStr}(\In{e}{estrDato}) $\to$ $res$ : string}
      \begin{algorithmic}
       \State $res \gets e.valorStr$           \Comment $O(1)$
       
      \medskip
      \Statex \underline{Complejidad:} {$O(1)$}
    \Statex \underline{Justificaci�n:} {Todas las funciones llamadas tienen complejidad $O(1)$.}

      \end{algorithmic}
\end{algorithm}

\begin{algorithm}[H]{\textbf{iMismoTipo?}(\In{e_1}{estrDato}, \In{e_2}{estrDato}) $\to$ $res$ : bool}
      \begin{algorithmic}
       \State $res \gets (e_1.nat? = e_2.nat?)$           \Comment $O(1)$
       
      \medskip
      \Statex \underline{Complejidad:} {$O(1)$}
    \Statex \underline{Justificaci�n:} {Todas las funciones llamadas tienen complejidad $O(1)$.}

      \end{algorithmic}
\end{algorithm}

\begin{algorithm}[H]{\textbf{iString?}(\In{e}{estrDato}) $\to$ $res$ : bool}
      \begin{algorithmic}
       \State $res \gets \NOT e.nat?$          \Comment $O(1)$
       
      \medskip
      \Statex \underline{Complejidad:} {$O(1)$}
    \Statex \underline{Justificaci�n:} {Todas las funciones llamadas tienen complejidad $O(1)$.}

      \end{algorithmic}
\end{algorithm}

\begin{algorithm}[H]{\textbf{iMin}(\In{cs}{conj(estrDato)}) $\to$ $res$ : estrDato}
  \begin{algorithmic}
    \State $it \gets CrearIt(cs)$           \Comment $O(1)$
    \State $res \gets Siguiente(it)$        \Comment $O(1)$
    \While{$HaySiguiente(it)$}              \Comment $O(\#cs * ...)$
      \State{\comentarioder{complejidad heredada de $\argumento$ $\le_{i}$ $\argumento$}}
      \If{$\NOT res \le_{i} Siguiente(it)$} \Comment $O\big(min\{|res.valorStr|, |Siguiente(it).valorStr|\}\big)$


        \State $res \gets Siguiente(it)$    \Comment $O(1)$
      \EndIf
      \State $Avanzar(it)$                  \Comment $O(1)$
    \EndWhile

    \medskip
    \Statex \underline{Complejidad:} {$O(\#cs * L)$, donde $L$ es la longitud del string m�s largo}
    \Statex \underline{Justificaci�n:} {Se recorre toda la lista $\big(O(\#cs)\big)$ y cada elemento se compara con el auxiliar $res$. En alg�n momento, $res$ se va a comparar con el string de longitud $L$. }
      \end{algorithmic}
\end{algorithm}


\begin{algorithm}[H]{\textbf{iMax}(\In{cs}{conj(estrDato)}) $\to$ $res$ : estrDato}
  \begin{algorithmic}
    \State $it \gets CrearIt(cs)$           \Comment $O(1)$
    \State $res \gets Siguiente(it)$        \Comment $O(1)$
    \While{$HaySiguiente(it)$}              \Comment $O(\#cs * ...)$
      \State{\comentarioder{complejidad heredada de $\argumento$ $\le_{i}$ $\argumento$}}
      \If{$res \le_{i} Siguiente(it)$} \Comment $O\big(min\{|res.valorStr|, |Siguiente(it).valorStr|\}\big)$
        \State $res \gets Siguiente(it)$    \Comment $O(1)$
      \EndIf
      \State $Avanzar(it)$                  \Comment $O(1)$
    \EndWhile

    \medskip
    \Statex \underline{Complejidad:} {$O(\#cs * L)$, donde $L$ es la longitud del string m�s largo}
    \Statex \underline{Justificaci�n:} {Se recorre toda la lista $\big(O(\#cs)\big)$ y cada elemento se compara con el auxiliar $res$. En alg�n momento, $res$ se va a comparar con el string de longitud $L$. }
      \end{algorithmic}
\end{algorithm}

\begin{algorithm}[H]{\textbf{$\argumento$ $\le_i$ $\argumento$\ }(\In{e_1}{estrDato}, \In{e_2}{estrDato}) $\to$ $res$ : bool}
  \begin{algorithmic}
    \If{$e_1.nat?$} \Comment $O(1)$
      \State $res \gets (e_1.valorNat \le e_2.valorNat)$ \Comment $O(1)$
    \Else
      \State $res \gets true$ \Comment $O(1)$
      \While{$res \ \AND i < min\big(longitud(e_1.valorStr), longitud(e_2.valorStr)\big)$} \Comment $O(1)$
        \If{$e_1.valorStr[i] > e_2.valorStr[i]$} \Comment $O(1)$
          \State $res \gets false$ \Comment $O(1)$
        \EndIf
      \EndWhile
      \State $res \gets (e_1.valorStr \le e_2.valorStr)$ \Comment $O\big(min\{|e_1.valorStr|, |e_2.valorStr|\}\big)$
    \EndIf

    \medskip
    \Statex \underline{Complejidad:} {$O\big(min\{|e_1.valorStr|, |e_2.valorStr|\}\big)$}
    \Statex \underline{Justificaci�n:} {Para determinar la desigualdad entre ambos vectores de strings realiza comparaciones entre chars $O(1)$ hasta el m�nimo de las longitudes }
  \end{algorithmic}
\end{algorithm}

\begin{algorithm}[H]{\textbf{$\argumento$ = $\argumento$\ }(\In{e_1}{estrDato}, \In{e_2}{estrDato}) $\to$ $res$ : bool}
  \begin{algorithmic}
    \If{$MismoTipo?(e_1, e_2)$}         \Comment $O(1)$
      \If{$e_1.nat?$}                   \Comment $O(1)$
        \State $res \gets (e_1.valorNat = e_2.valorNat)$ \Comment $O(1)$
      \Else
        \State $res \gets (e_1.valorStr = e_2.valorStr)$ \Comment $O\big(min\{|e_1.valorStr|, |e_2.valorStr|\}\big)$
      \EndIf
    \Else
      \State $res \gets false$ \Comment $O(1)$
    \EndIf

    \medskip
    \Statex \underline{Complejidad:} {$O\big(min\{|e_1.valorStr|, |e_2.valorStr|\}\big)$}
    \Statex \underline{Justificaci�n:} {$O(1)$ + $O(1)$ + $O(1)$ + $O\big(min\{|e_1.valorStr|, |e_2.valorStr|\}\big)$ = $O\big(min\{|e_1.valorStr|, |e_2.valorStr|\}\big)$}
  \end{algorithmic}
\end{algorithm}

\begin{algorithm}[H]{\textbf{iCopiar}(\In{e}{estrDato}) $\to$ $res$ : estrDato}
      \begin{algorithmic}
        \State $res \gets \big\langle e.nat?, e.valorStr, e.valorNat \big\rangle$         \Comment $O(1) + O(|e.valorStr|) + O(1)$
      \medskip
      \Statex \underline{Complejidad:} {$O(|e.valorStr|)$}
      \Statex \underline{Justificaci�n:} {La complejidad del algoritmo es la complejidad de copiar un string. La complejidad de copiar un string es $O\big(long(string)\big)$. Si esNat? es true, por invariante, el largo de valorStr est� acotado (valorStr = "vacio") y la complejidad es $O(1)$}

      \end{algorithmic}
\end{algorithm}



\end{Algoritmos}
\newpage
\section{M�dulo DiccionarioString(significado)}

\begin{Interfaz}

  \textbf{par�metros formales}\hangindent=2\parindent\\
  \parbox{1.7cm}{\textbf{g�neros}} significado\\
  \parbox[t]{1.7cm}{\textbf{funci�n}}\parbox[t]{\textwidth-2\parindent-1.7cm}{%
    \InterfazFuncion{Copiar}{\In{sig}{significado}}{significado}
    [true]
    {$res$ $\igobs$ $sig$}
    [$O\big(copy(sig)\big)$]
    [vuelve una copia del par�metro.]
  }

  \textbf{se explica con}: \tadNombre{Diccionario Extendido(String, Significado)}.

  \textbf{g�neros}: \TipoVariable{diccString(significado)}.

  \textbf{servicios exportados}: \tadNombre{Todos los de la interfaz}

  \textbf{servicios usados}\hangindent=2\parindent : Lista, itLista, Conjunto Lineal (c�tedra) \\

  \textbf{}

  \Titulo{Operaciones b�sicas de diccString}

  \InterfazFuncion{Vac�o}{}{diccString(significado)}
  [true]
  {$res$ $\igobs$ vacio}
  [$O(1)$]
  [genera un DiccionarioString vac�o.]

  \InterfazFuncion{Definir}{\In{clave}{string}, \In{significado}{significado}, \Inout{dicc}{diccString(significado)}}{}
  [$\neg$def?($clave$,$dicc$) $\land$ $dicc_0$ = $dicc$]
  {$dicc$ $\igobs$ definir($clave$, $significado$, $dicc_0$)}
  [$O\big(max\{long(clave), copy(significado)\}\big)$]
  [define la clave ingresada en el diccionario.]

  \InterfazFuncion{Def?}{\In{clave}{string}, \In{dicc}{diccString(significado)}}{bool}
  [true]
  {$res$ $\igobs$ def?($clave$, $dicc$)}
  [$O\big(long(clave)\big)$]
  [devuelve \texttt{true} si la clave est� definida en el diccionario.]

  \InterfazFuncion{Obtener}{\In{clave}{string}, \In{dicc}{diccString(significado)}}{significado}
  [def?($clave$, $dicc$)]
  {alias\big($res$ $\igobs$ obtener($clave$, $dicc$)\big)}
  [$O\big(long(clave)\big)$]
  [devuelve el significado correspondiente a la clave ingresada.]
  [se genera alias entre $res$ y el significado en el diccionario si el tipo significado no es primitivo. $res$ no es modificable.]

  \InterfazFuncion{Claves}{\In{dicc}{diccString(significado)}}{conj(string)}
  [true]
  {$res$ $\igobs$ claves($dicc$)}
  [$O\big(\#claves(dicc) * M\big)$, donde $M$ es la longitud del mayor string clave.]
  [devuelve por copia el conjunto de las claves del diccionario ingresado.]

  \InterfazFuncion{Borrar}{\In{clave}{string}, \Inout{dicc}{diccString(significado)}}{}
  [def?($clave$, $dicc$) $\land$ $dicc_0$ = $dicc$]
  {$dicc$ $\igobs$ borrar($clave$, $dicc_0$)}
  [$O\big(long(clave)\big)$]
  [borra la clave del diccionario.]

  \InterfazFuncion{VistaDicc}{\Inout{dicc}{diccString(significado)}}{itLista\big(tupla( Clave: string, significado: significado)\big)}
  [true]
  {alias\big($dicc$ $\igobs$ secuADicc\big(secuSuby($res$)\big)\big)}
  [$O(1)$]
  [devuelve un iterador a una lista de tuplas con las claves y sus significados.]
  [el iterador no es modificable.]


  \InterfazFuncion{Copiar}{\In{dicc}{diccString(significado)}}{diccString(significado)}
  [true]
  {$res$ $\igobs$ $dicc$}
  [$O\big(\#claves(dicc) * max\{k, s\}\big)$, donde $k$ es la longitud m�xima de cualquier clave en $dicc$ y $s$ el m�ximo costo de copiar un significado de $dicc$ de dicho tipo.]
  [devuelve una copia sin aliasing del diccionario de entrada.]

  \InterfazFuncion{Min}{\In{dicc}{diccString(significado)}}{string}
  [\#claves($e$) > 0]
  {alias($res$ $\igobs$ $m$) | $\big(\forall$ $n$ $\in$ claves($dicc$)$\big)$ ($m$ $\le$ $n$) $\land$ ($m$ $\in$ claves($dicc$))}
  [$O\big(long\big(|min(claves(dicc))|\big)\big)$]
  [Devuelve la clave m�nima del diccionario seg�n orden lexicogr�fico.]
  [res no es modificable]

  \InterfazFuncion{Max}{\In{dicc}{diccString(significado)}}{string}
  [\#claves($e$) > 0]
  {alias($res$ $\igobs$ $m$) | $\big(\forall$ $n$ $\in$ claves($dicc$)$\big)$ ($m$ $\ge$ $n$) $\land$ ($m$ $\in$ claves($dicc$))}
  [$O\big(long\big(|max(claves(dicc))|\big)\big)$]
  [Devuelve la clave m�xima del diccionario seg�n orden lexicogr�fico.]
  [res no es modificable]

\end{Interfaz}

\begin{Representacion}

  \Titulo{Representaci�n de diccString}

  \begin{Estructura}{diccString(significado)}[estrDiccString]

      \dondees{estrDiccString}{tupla}\big(\emph{trie}: \TipoVariable{nodoTrie},
        \emph{valores}: \TipoVariable{lista}\big(\TipoVariable{tupla<}\emph{clave}: \TipoVariable{string}, \emph{significado}: \TipoVariable{significado>}\big)\big)


    \begin{Tupla}[nodoTrie]
      \tupItem{valor}{puntero\big(itLista(tupla<string, significado>)\big)}%
      \tupItem{hijos}{arreglo[256] de puntero(nodoTrie)}%
      \tupItem{cantHijos}{nat}%
    \end{Tupla}
  \end{Estructura}

  \Rep[estrDiccString][e]{\\
  1) Un significado est� en la lista de valores si y s�lo si hay un nodo en $e$.trie que apunta a un iterador cuyo siguiente es ese valor. \\
  2) La clave de ese valor corresponde al string formado concatenando los valores char del �ndice de cada hijo que se recorre, dicho recorrido es �nico (p.e: $"Alas"$ solamente est� definido si el nodo correspondiente al recorrido A$\to$L$\to$A$\to$S apunta a un valor no nulo). Por lo tanto, el primer nodo no puede apuntar a un valor v�lido. \\
  3) cantHijos es igual a la cantidad de punteros no nulos en hijos. \\
  4) No existen dos nodos en la estructura recursiva que compartan alguno de sus hijos.\\


  }

  ~

  \Abs[estrDiccString]{diccString(significado)}[e]{d}{($\forall$ $c$ : string) def?($c$, $d$) $\ssi$ esClave?($c$, $e$.valores) $\yluego$ def?($c$, $d$) $\impluego$ obtener($c$, $d$) $\igobs$ significado($c$, $e$.valores)}



  \tadOperacion{esClave?}{string/s, secu$\big($tupla(string$\text{,}\alpha)\big)$/xs}{bool}{Rep($e$)}
  \tadAxioma{esClave?($s$, $xs$)}{$\neg$vacia?($xs$) $\yluego$ $\big(\Pi_1$(prim($xs$)) \ $\igobs$ $s$ $\lor$ esClave?($s$, fin($xs$))$\big)$}

  ~

  \tadOperacion{significado}{string/s, secu$\big($tupla(string$\text{,}\alpha)\big)$/xs}{$\alpha$}{esClave?($s$, $xs$)}
  \tadAxioma{significado($s$, $xs$)}{\IF $\Pi_1$(prim($xs$)) $\igobs$ $s$ THEN
  $\Pi_2$(prim($xs$))
ELSE
  significado$\big(s$, fin($xs$)$\big)$
FI}

\end{Representacion}

\bigskip

\begin{Algoritmos}

\medskip

 \Titulo{Algoritmos de diccString}
  	\medskip

\begin{algorithm}[H]{\textbf{iVacio}() $\to$ $res$ : estrDiccString}
      \begin{algorithmic}
       \State $res.valores \gets Vacia()$         \Comment $O(1)$
       \State $res.trie \gets iNuevoNodo()$       \Comment $O(1)$

      \medskip
      \Statex \underline{Complejidad:} {$O(1)$}
      \Statex \underline{Justificaci�n:} {Todas las funciones llamadas tienen complejidad $O(1)$.}

      \end{algorithmic}
\end{algorithm}

\begin{algorithm}[H]{\textbf{iNuevoNodo}() $\to$ $res$ : nodoTrie \comentariorandom{no se exporta}}
      \begin{algorithmic}
        \State $res \gets \big\langle NULL, CrearArreglo(256), 0\big\rangle$ \Comment $O(1)$
        \For{$i \gets 0 \ \TO 255$}                      \Comment $O(255 * ...) = O(1 * ...)$
          \State $res.hijos[i] \gets NULL$         \Comment $O(1)$
        \EndFor

        \medskip
        \Statex \underline{Complejidad:} {$O(1)$}
        \Statex \underline{Justificaci�n:} {Todas las funciones llamadas tienen complejidad $O(1)$.}
      \end{algorithmic}
\end{algorithm}

\begin{algorithm}[H]{\textbf{iDefinir}(\In{clave}{string}, \In{significado}{significado}, \Inout{e}{estrDiccString})}
      \begin{algorithmic}
        \State $entrada \gets \big\langle clave, significado \big\rangle$     \Comment $O(1)$
        \State $ $

        \comentario{agregamos a la lista}
        \State $iter \gets AgregarAdelante(e.valores, entrada)$       \Comment $O\big(copy(entrada)\big) = O\big(copy(clave)\big) + O\big(copy(significado)\big)$
        \comentario{iter tiene a entrada como siguiente}
        \State $ $

        \State $actual \gets \&e.trie$ \comentariorandom{actual es de tipo puntero(nodoTrie)}                             \Comment $O(1)$
        \For{$i \gets 0 \ \TO Longitud(clave) - 1$}     \Comment $O\big(long(clave) * ...\big)$
          \If{$\big(actual \to hijos\big)\big[ord(clave[i])\big] = NULL$} \Comment $O(1)$
            \State $\big(actual \to hijos\big)\big[ord(clave[i])\big] \gets \&\big(iNuevoNodo()\big)$   \Comment $O(1)$
            \State $(actual \to cantHijos) \gets (actual \to cantHijos) + 1$   \Comment $O(1)$
          \EndIf
          \State $actual \gets \big(actual \to hijos\big)\big[ord(clave[i])\big]$ \Comment $O(1)$
        \EndFor
        \State $(actual \to valor) \gets \&iter$   \Comment $O(1)$

        \medskip
        \Statex \underline{Complejidad:} {$O\big(Longitud(clave)\big)$ + $O\big(copy(clave)\big)$ + $O\big(copy(significado)\big)$ \\
        \somequad = $O\big(max\{Longitud(clave), copy(significado)\}\big)$}
        \Statex \underline{Justificaci�n:} {\\
\quad\quad Para definir creamos una tupla y copiamos la clave y el significado, por lo tanto tenemos en complejidad la copia del m�s grande de los dos, es decir, $O\big(copy(clave)\big)$ + $O\big(copy(significado)\big)$ = $O(L)$ + $O\big(copy(significado)\big)$. \\
\quad\quad Luego se hace un ciclo que se realiza $L$ veces y hace operaciones $O(1)$. \\
\quad\quad Por lo tanto la complejidad queda $O(L)$ + $O\big(copy(significado)\big)$ + $O(L)$. Por ser sumas queda la mayor de ellas, es decir $max\{2*O(L), O\big(copy(significado)\big)\}$ = $O\big(max\{L, copy(significado)\}\big)$ \\
\quad\quad $L$: Longitud de la clave.


        }
      \end{algorithmic}
\end{algorithm}

\begin{algorithm}[H]{\textbf{iDef?}(\In{clave}{string}, \In{e}{estrDiccString}) $\to$ $res$ : bool}
      \begin{algorithmic}
        \State $actual \gets \&e.trie$     \Comment $O(1)$
        \State $res \gets true$     \Comment $O(1)$
        \State $i \gets 0$     \Comment $O(1)$
        \While{$i < Longitud(clave)\ \AND res$} \Comment $O\big(long(clave) * ...\big)$
          \If{$actual \to cantHijos > 0$} \Comment $O(1)$
            \If{$(actual \to hijos)\big[ord(clave[i])\big] = NULL$} \Comment $O(1)$
              \State $res \gets false$ \Comment $O(1)$
            \Else
              \State $actual \gets \big(actual \to hijos\big[ord(clave[i])\big]\big)$ \Comment $O(1)$
            \EndIf
            \State $i++$  \Comment $O(1)$
          \Else
            \State $res \gets false$ \Comment $O(1)$
          \EndIf
        \EndWhile
        \State $ $

        \If{$res$}  \Comment $O(1)$
          \State $res \gets \NOT\big((actual \to valor) = NULL\big)$ \Comment $O(1)$
        \EndIf

        \medskip
        \Statex \underline{Complejidad:} {$O\big(long(clave)\big)$}
        \Statex \underline{Justificaci�n:} {El ciclo itera a lo sumo long($clave$) veces.}
      \end{algorithmic}
\end{algorithm}

\begin{algorithm}[H]{\textbf{iObtener}(\In{clave}{string}, \In{e}{estrDiccString}) $\to$ $res$ : significado}
      \begin{algorithmic}
        \State $actual \gets \&e.trie$     \Comment $O(1)$
        \For{$i \gets 0 \ \TO Longitud(clave) - 1$}     \Comment $O\big(long(clave) * ...\big)$
          \State $actual \gets \big(actual \to hijos\big[ord(clave[i])\big]\big)$ \Comment $O(1)$
        \EndFor
        \State $res \gets Siguiente\big(*(actual \to valor)\big).significado$ \Comment $O(1)$

        \medskip
        \Statex \underline{Complejidad:} {$O\big(long(clave)\big)$}
        \Statex \underline{Justificaci�n:} {El ciclo itera a lo sumo long($clave$) veces.}
      \end{algorithmic}
\end{algorithm}

\begin{algorithm}[H]{\textbf{iClaves}(\In{e}{estrDiccString}) $\to$ $res$ : Conj(string)}
      \begin{algorithmic}
        \State $it \gets CrearIt(e.valores)$     \Comment $O(1)$
        \State $res \gets Vacio()$                     \Comment $O(1)$
        \While{$HaySiguiente?(it)$} \Comment $O\big(long(e.valores) * ...\big)$
          \State $AgregarRapido(aux, Siguiente(it).clave)$ \Comment $O\big(copy(clave)\big)$
          \State $Avanzar(it)$ \Comment $O(1)$
        \EndWhile

        \medskip
        \Statex \underline{Complejidad:} {$O\big(long(e.valores) * M\big)$, donde $M$ es la longitud del mayor string clave en $e$.valores.}
        \Statex \underline{Justificaci�n:} {El ciclo itera toda la lista copiando la clave de cada elemento. Se acota el costo de copiado de todos los elementos por el del copiado de mayor longitud.}
      \end{algorithmic}
\end{algorithm}

\begin{algorithm}[H]{\textbf{iBorrar}(\In{clave}{string}, \Inout{e}{estrDiccString})}
      \begin{algorithmic}
        \State $actual \gets \&e.trie$ \Comment $O(1)$
        \State $listo \gets false$ \Comment $O(1)$

        \For{$i \gets 0 \ \TO Longitud(clave) - 1$}     \Comment $O\big(long(clave) * ...\big)$
          \State $temp \gets \big(actual \to hijos\big[ord(clave[i])\big]\big)$ \comentariorandom{guardamos a d�nde apunta} \Comment $O(1)$
          \If{$\big(actual \to hijos\big[ord(clave[i])\big]\big) \to cantHijos = 0$} \Comment $O(1)$
            \State $\big(actual \to hijos\big[ord(clave[i])\big]\big) \gets NULL$ \Comment $O(1)$
          \EndIf
          \State $actual \gets temp$ \Comment $O(1)$
          \comentario{seguimos recorriendo lo que antes era su nodo hijo para liberar el resto de memoria}
        \EndFor
        \State $EliminarSiguiente\big(*(actual \to valor)\big)$ \Comment $O(1)$
        \State $ $

        \comentario{crea un iterador uniendo la lista antes del elemento m�s la lista despu�s del elemento}
        \State $(actual \to valor) \gets NULL$ \Comment $O(1)$
        \State $actual \gets \&e.trie$ \Comment $O(1)$

        \State $i \gets 0$ \Comment $O(1)$
        \While{$i < Longitud(clave) - 1 \AND \NOT listo$} \Comment $O\big(long(clave) * ...\big)$
          \If{$\big(actual \to hijos\big[ord(clave[i])\big]\big) \to cantHijos > 0$} \Comment $O(1)$
            \State $actual \gets \big(actual \to hijos\big[ord(clave[i])\big]\big)$ \Comment $O(1)$
          \Else
            \State $\big(actual \to hijos\big[ord(clave[i])\big]\big) \gets NULL$ \Comment $O(1)$
            \State $listo \gets true$ \Comment $O(1)$
          \EndIf
          \State $i++$ \Comment $O(1)$
        \EndWhile

        \medskip
        \Statex \underline{Complejidad:} {$O\big(long(clave)\big)$}
        \Statex \underline{Justificaci�n:} {Ambos ciclos iteran a lo sumo long($clave$) veces.}
      \end{algorithmic}
\end{algorithm}


\begin{algorithm}[H]{\textbf{iVistaDicc}(\In{e}{estrDiccString}) $\to$ $res$ : itLista\big(tupla\big\langle clave: string, significado: significado\big\rangle\big)}
      \begin{algorithmic}
        \State $res \gets CrearIt(e.valores)$ \Comment $O(1)$

        \medskip
        \Statex \underline{Complejidad:} {$O(1)$}
        \Statex \underline{Justificaci�n:} {El algoritmo tiene una �nica llamada a una funci�n con costo $O(1)$.}
      \end{algorithmic}
\end{algorithm}

\begin{algorithm}[H]{\textbf{iCopiar}(\In{e}{estrDiccString}) $\to$ $res$ : estrDiccString}
      \begin{algorithmic}
        \State $it \gets CrearIt(e.valores)$ \Comment $O(1)$
        \State $res \gets Vacio()$ \Comment $O(1)$

        \While{$HaySiguiente(it)$} \Comment $O\big(long(e.valores) * ...\big)$
          \State $Definir\big(Siguiente(it).clave, Siguiente(it).significado, res\big)$ \Comment $O\big(max\{K, S\}\big)$
          \State $Avanzar(it)$ \Comment $O(1)$
        \EndWhile

        \medskip
        \Statex \underline{Complejidad:} {$O\big(long(e.valores) * max\{K, S\}\big)$, donde $K$ es la longitud m�xima de cualquier clave en $e$ y $S$ el m�ximo costo de copiar un significado de $e$ de dicho tipo.}
        \Statex \underline{Justificaci�n:} {El ciclo itera toda la lista definiendo cada clave en un nuevo diccionario.}
      \end{algorithmic}
\end{algorithm}

\begin{algorithm}[H]{\textbf{iMin}(\In{e}{estrDiccString}) $\to$ $res$ : string}
      \begin{algorithmic}
        \State $actual \gets \&e.trie$ \Comment $O(1)$
        \State $termine \gets false$ \Comment $O(1)$

        \While{$\NOT termine$} \Comment $O(L * ...)$
          \If{$(actual \to valor) = NULL$} \Comment $O(1)$
            \For{$i \gets 0 \ \TO 255$}     \Comment $O(255 * ...) = O(1 * ...)$
              \If{$\NOT\ actual \to hijos\big[ord(clave[i])\big] = NULL$} \Comment $O(1)$
                \State $actual \gets \big(actual \to hijos\big[ord(clave[i])\big]\big)$ \Comment $O(1)$
              \EndIf
            \EndFor
          \Else
            \State $termine \gets true$ \Comment $O(1)$
            \State $res \gets Siguiente\big(*(actual \to valor)\big).clave$ \Comment $O(1)$
          \EndIf

        \EndWhile

        \medskip
        \Statex \underline{Complejidad:} {$O\big(long\big(|min(claves(e))|\big)\big)$}
        \Statex \underline{Justificaci�n:} {\\
\quad\quad En el algoritmo hay un ciclo principal (while) y un ciclo interno (for) y luego fuera de los ciclos operaciones $O(1)$. El ciclo del while itera hasta que encuentra la primer palabra completa recorriendo desde el nodo del trie buscando siempre desde el primer char hasta el �ltimo. \\
\quad\quad Entonces iteramos $L$ veces, por lo tanto tenemos $O(L)$. Pero dentro del while hay un ciclo interno(for) que itera siempre 255 veces, por lo tanto tenemos de complejidad $O(L*255)$. La constante se puede sacar y queda $O(L)$. \\
\quad\quad Todas las operaciones internas de los ciclos son $O(1)$.\\
\\
\quad\quad $L$: Longitud de la clave m�nima del diccionario.

        }
      \end{algorithmic}
\end{algorithm}

\begin{algorithm}[H]{\textbf{iMax}(\In{e}{estrDiccString}) $\to$ $res$ : string}
      \begin{algorithmic}
        \State $actual \gets \&e.trie$ \Comment $O(1)$
        \State $termine \gets false$ \Comment $O(1)$

        \While{$\NOT termine$} \Comment $O(L * ...)$
          \comentario{Quiero que mientras haya hijos se meta en el �ndice m�s grande}
          \If{$(actual \to cantHijos) = 0$} \Comment $O(1)$
            \State $res \gets Siguiente\big(*(actual \to valor)\big).clave$ \Comment $O(1)$
            \State $termine \gets true $   \Comment $O(1)$
          \Else
            \State $i \gets 255$ \Comment $O(1)$
            \State $seguir \gets true$ \Comment $O(1)$
            \State $ $

            \While{$i \ge 0\ \AND seguir$}
              \If{$\NOT actual \to hijos\big[ord(clave[i])\big] = NULL$} \Comment $O(1)$
                \State $actual \gets \big(actual \to hijos\big[ord(clave[i])\big]\big)$ \Comment $O(1)$
                \State $seguir \gets false$ \Comment $O(1)$
              \EndIf
              \State $i--$ \Comment $O(1)$
            \EndWhile
          \EndIf
        \EndWhile

        \medskip
        \Statex \underline{Complejidad:} {$O\big(long\big(|min(claves(e))|\big)\big)$}
        \Statex \underline{Justificaci�n:} {\\
\quad\quad En el algoritmo hay un ciclo principal (while)  y un ciclo interno (for) y luego fuera de los ciclos operaciones $O(1)$.El ciclo del while itera hasta que encuentra la primer palabra completa recorriendo desde el nodo del trie buscando siempre desde el �ltimo char hasta el primero. \\
\quad\quad Entonces iteramos L veces por lo tanto tenemos $O(L)$. Pero dentro del while hay un ciclo interno(for) que itera siempre 255 veces, por lo tanto tenemos de complejidad $O(L*255)$. La constante se puede sacar y queda $O(L)$.\\
\quad\quad Todas las operaciones internas de los ciclos son $O(1)$.\\
\\
\quad\quad $L$: Longitud de la clave m�xima del diccionario.
        }
      \end{algorithmic}
\end{algorithm}

\end{Algoritmos}

\newpage
\section{M�dulo DiccionarioNat(significado)}

\begin{Interfaz}

  \textbf{par�metros formales}\hangindent=2\parindent\\
  \parbox{1.7cm}{\textbf{g�neros}} significado\\
  \parbox[t]{1.7cm}{\textbf{funci�n}}\parbox[t]{\textwidth-2\parindent-1.7cm}{%
    \InterfazFuncion{Copiar}{\In{sig}{significado}}{significado}
    [true]
    {$res$ $\igobs$ $sig$}
    [$O\big(copy(sig)\big)$]
    [vuelve una copia del par�metro.]
  }
  \textbf{se explica con}: \tadNombre{Diccionario Extendido(Nat, Significado)},\\ \tadNombre{Iterador Unidireccional\big(Tupla(Nat, Significado)\big)}.

  \textbf{g�neros}: \TipoVariable{diccNat(significado)}, \TipoVariable{itDiccNat(significado)}.

  \Titulo{Operaciones b�sicas de diccNat}

  \InterfazFuncion{Vac�o}{}{diccNat(significado)}
  [true]  
  {$res$ $\igobs$ vacio}
  [$O(1)$]
  [genera un DiccionarioNat vac�o.]
  
  \InterfazFuncion{Definir}{\In{clave}{nat}, \In{significado}{significado}, \Inout{dicc}{diccNat(significado)}}{}
  [$\neg$def?($clave$,$dicc$) $\land$ $dicc_0$ = $dicc$]
  {$dicc$ $\igobs$ definir($clave$, $significado$, $dicc_0$)}
  [$O\big(\#claves(dicc)\big)$ / $O\big(log\ \#claves(dicc)\big)$ asumiendo distribuci�n uniforme de claves.]
  [define la clave ingresada en el diccionario.]
  
  \InterfazFuncion{Def?}{\In{clave}{nat}, \In{dicc}{diccNat(significado)}}{bool}
  [true]
  {$res$ $\igobs$ def?($clave$, $dicc$)}
  [$O\big(\#claves(dicc)\big)$ / $O\big(log\ \#claves(dicc)\big)$ asumiendo distribuci�n uniforme de claves.]
  [devuelve \texttt{true} si la clave est� definida en el diccionario.]

  \InterfazFuncion{Obtener}{\In{clave}{nat}, \In{dicc}{diccNat(significado)}}{significado}
  [def?($clave$, $dicc$)]
  {alias\big($res$ $\igobs$ obtener($clave$, $dicc$)\big)}
  [$O\big(\#claves(dicc)\big)$ / $O\big(log\ \#claves(dicc)\big)$ asumiendo distribuci�n uniforme de claves.]
  [devuelve el significado correspondiente a la clave ingresada.]
  [se genera alias entre $res$ y el significado en el diccionario si el tipo significado no es primitivo. $res$ no es modificable.]

  \InterfazFuncion{Borrar}{\In{clave}{nat}, \Inout{dicc}{diccNat(significado)}}{}
  [def?($clave$, $dicc$) $\land$ $dicc_0$ = $dicc$]
  {$dicc$ $\igobs$ borrar($clave$, $dicc_0$)}
  [$O\big(\#claves(dicc)\big)$ / $O\big(log\ \#claves(dicc)\big)$ asumiendo distribuci�n uniforme de claves.]
  [borra la clave del diccionario.]

  \InterfazFuncion{Min}{\In{dicc}{diccNat(significado)}}{tupla(nat, significado)}
  [\#claves($dicc$) > 0]
  {alias$\big(\Pi_1$($res$) $\igobs$ min$\big($claves($dicc$)$\big)\big)$ $\yluego$ alias$\big(\Pi_2$($res$) $\igobs$ obtener$\big(\Pi_1$($res$), $dicc\big)\big)$}
  [$O\big(\#claves(dicc)\big)$ / $O\big(log\ \#claves(dicc)\big)$ asumiendo distribuci�n uniforme de claves.]
  [devuelve una tupla con la clave m�nima y su significado.]
  [$res$ no es modificable.]

  \InterfazFuncion{Max}{\In{dicc}{diccNat(significado)}}{tupla(nat, significado)}
  [\#claves($dicc$) > 0]
  {alias$\big(\Pi_1$($res$) $\igobs$ max$\big($claves($dicc$)$\big)\big)$ $\yluego$ alias$\big(\Pi_2$($res$) $\igobs$ obtener$\big(\Pi_1$($res$), $dicc\big)\big)$}
  [$O\big(\#claves(dicc)\big)$ / $O\big(log\ \#claves(dicc)\big)$ asumiendo distribuci�n uniforme de claves.]
  [devuelve una tupla con la clave m�xima y su significado.]
  [$res$ no es modificable.]

  \Titulo{Operaciones b�sicas del iterador}

  \InterfazFuncion{CrearIt}{\In{dicc}{diccNat(significado)}}{itDiccNat(significado)}
  [true]
  {alias$\big($secuADicc$\big($Siguientes($res$)$\big)$ $\igobs$ $dicc\big)$}
  [$O(1)$]
  [crea un iterador del conjunto.]
  [el iterador no puede realizar modificaciones y se indefine con la inserci�n y eliminaci�n de elementos en el diccionario.]

  \InterfazFuncion{Siguientes}{\In{it}{itDiccNat(significado)}}{lista\big(tupla(nat, significado)\big)}
  [true]
  {$res$ $\igobs$ siguientes($it$)}
  [$O\big(n*copy(x)\big)$, siendo $n$ la cantidad de claves del diccionario y $x$ el significado mas costoso de copiar.]
  [devuelve una lista con las claves siguientes y sus significados.]
  [no hay ya que se copian los elementos.]

  \InterfazFuncion{Avanzar}{\Inout{it}{itDiccNat(significado)}}{}
  [HayMas?($it$) $\land$ $it_0$ = $it$]
  {$it$ $\igobs$ Avanzar($it_0$)}
  [$O(1)$]
  [avanza a la posici�n siguiente del iterador.]

  \InterfazFuncion{HayMas?}{\In{it}{itDiccNat(significado)}}{bool}
  [true]
  {$res$ $\igobs$ HayMas?($it$)}
  [$O(1)$]
  [devuelve \texttt{true} si y s�lo si el en el iterador todav�a quedan elementos para avanzar.]

  \InterfazFuncion{Actual}{\In{it}{itDiccNat(significado)}}{tupla(nat, significado)}
  [HayMas?($it$)]
  {alias$\big(res$ $\igobs$ Actual($it$)$\big)$}
  [$O(1)$]
  [devuelve el elemento correspondiente a la posici�n actual del iterador.]
  [Genera aliasing. $res$ no es modificable.]

\end{Interfaz}

\begin{Representacion}
  
  \Titulo{Representaci�n de diccNat}

  \begin{Estructura}{diccNat(significado)}[estrDiccNat]

      \dondees{estrDiccNat}{puntero\big(nodoDiccNat(significado)\big)}\

      \begin{Tupla}[nodoDiccNat(significado)]
        \tupItem{clave}{nat}%
        \tupItem{significado}{puntero(significado)}%
        \tupItem{\\izq}{puntero(nodoDiccNat)}%
        \tupItem{der}{puntero(nodoDiccNat)}%
      \end{Tupla}

  \end{Estructura}

  \noindent\textbf{Invariante de representaci�n}
  
  \noindent\tab 1) Hijo izq menor estricto e hijo derecho mayor estricto. \\
\noindent\tab 2) Rep recursivo en sus hijos. \\
\noindent\tab 3) El significado no es nulo. \\
\noindent\tab 4) No hay repetidos. \\

  \Rep[estrDiccNat][e]{($e$ $\igobs$ NULL) $\oluego$ $\big($($e$ $\to$ izq $\nigobs$ NULL) $\implies$ ($e$ $\to$ izq $\to$ clave) <\ $e$ $\to$ clave$\big)$\\
    $\land$ $\big($($e$ $\to$ der $\nigobs$ NULL) $\implies$ ($e$ $\to$ der $\to$ clave) >\ $e$ $\to$ clave$\big)$\\
    $\land$ Rep($e$ $\to$ izq) $\land$ Rep($e$ $\to$ der) $\land$ ($e$ $\to$ significado $\nigobs$ NULL) $\land$ $\big(\forall$ $n$ : nat, esClave?($n$, $e$)$\big)$ ($n$ $\igobs$ $e$ $\to$ clave) $\implies$ $\big(\neg$esClave?($n$, $e$ $\to$ izq) $\land$ $\neg$esClave?($n$, $e$ $\to$ der)$\big)$ $\land$ $\big(e$ $\nigobs$ NULL $\yluego$ esClave?($n$, $e$ $\to$ izq)$\big)$ $\implies$ $\big(n$ $\nigobs$ $e$ $\to$ clave $\land$ $\neg$esClave?($n$, $e$ $\to$ der)$\big)$ $\land$ $\big(e$ $\nigobs$ NULL $\yluego$ esClave?($n$, $e$ $\to$ der)$\big)$ $\implies$ $\big(n$ $\nigobs$ $e$ $\to$ clave $\land$ $\neg$esClave?($n$, $e$ $\to$ izq)$\big)$}
	
  ~
 
  \Abs[estrDiccNat]{diccNat(significado)}[e]{d}{($\forall$ $c$ : nat) $\big($def?($c$, $d$) $\igobs$ esClave?($c$, $e$)$\big)$ $\land$ $\big($def?($c$, $d$) $\impluego$ obtener($c$, $d$) $\igobs$ suSignificado($c$, $d$)$\big)$}

  ~

  \tadOperacion{esClave?}{nat/n, estrDiccNat/e}{bool}{Rep($e$)}
  \tadAxioma{esClave?($n$, $e$)}{($e$ $\nigobs$ NULL) $\yluego$ $\big($($e$ $\to$ clave $\igobs$ $n$) $\lor$ (esClave?($n$, $e$ $\to$ der)) $\lor$ (esClave?($n$, $e$ $\to$ izq))$\big)$}

  ~

  \tadOperacion{suSignificado}{nat/n, estrDiccNat/e}{significado}{Rep($e$) $\yluego$ esClave?($n$, $e$)}
  \tadAxioma{suSignificado($n$, $e$)}
{\IF ($e$ $\to$ clave) $\igobs$ $c$ THEN
  *($e$ $\to$ significado) 
ELSE
  {\IF ($e$ $\to$ clave) <\ $c$ THEN
    suSignificado($c$, $e$ $\to$ izq) 
  ELSE
    suSignificado($c$, $e$ $\to$ der)
  FI}
FI}

\medskip

  \Titulo{Representaci�n del iterador}

  \begin{Estructura}{itDiccNat(significado)}[iter]

      \dondees{iter}{pila(estrDiccNat)}\

  \end{Estructura}

  \noindent\textbf{Invariante de representaci�n}
  
  \noindent\tab 1) Vale rep para cada elemento de la pila. \\
  \noindent\tab 2) No hay punteros nulos en la pila. \\
  \noindent\tab 3 La pila est� ordenada decrecientemente. \\
  \noindent\tab 4) No puede haber dos punteros apilados que compartan alg�n hijo (por lo tanto nunca va a haber hijos apilados para ning�n nodo). Tampoco por rep puede pasar que haya loops, por ejemplo: El nodo con clave 15 apunte a su izq al nodo con clave 9 y este a su der al nodo con clave 15 nuevamente. \\


  \Rep[iter][i]{$\big(\forall$ $n$ : estrDiccNat, estaEnPila?($n$, $i$)$\big)$ Rep($n$) $\yluego$ \\
  $\Big($noApilaNulos($i$) $\land$ ordenadaDec($i$) $\land$ $\big(\forall$ $n$ : estrDiccNat, estaEnPila?($n$, $i$)$\big)$ $\neg\big(\exists$ $n'$ : estrDiccNat, $n$ $\nigobs$ $n'$ $\land$ estaEnPila?($n'$, $i$)$\big)$ $\big(\exists$ $x$ : nat, esta?($x$, clavesDe($n$)) $\land$ esta?($x$, clavesDe($n'$))$\big)\Big)$}
  
  ~

\tadOperacion{noApilaNulos}{iter}{bool}{}
  \tadAxioma{noApilaNulos($i$)}
{\IF vacia?($i$) THEN
  true 
ELSE
  tope($i$) $\nigobs$ NULL $\land$ noApilaNulos$\big($desapilar($i$)$\big)$
FI}

 ~


\tadOperacion{clavesDe}{estrDiccNat}{secu(nat)}{}
  \tadAxioma{clavesDe($e$)}
{\IF $e$ $\nigobs$ NULL THEN
  clavesDe($e$ $\to$ izq) $\&$ $\big($($e$ $\to$ clave) $\argumento$ clavesDe($e$ $\to$ der)$\big)$ 
ELSE
  <>
FI}

 ~

\tadOperacion{ordenadaDec}{estrDiccNat}{bool}{}
  \tadAxioma{ordenadaDec($e$)}{$\big($tama�o($e$) <\ 2$\big)$ $\oluego$ $\big($(tope($e$) $\to$ clave) <\ tope$\big($desapilar($e$) $\to$ clave$\big)$ $\land$ ordenada$\big($desapilar($e$)$\big)\big)$}

 ~

\tadOperacion{estaEnPila?}{$\alpha$, pila($\alpha$)}{bool}{}
  \tadAxioma{estaEnPila?($e$, $s$)}
{\IF vacia?($s$) THEN
  false 
ELSE
  tope($s$) $\igobs$ $e$ $\lor$ estaEnPila?$\big(e$, desapilar($s$)$\big)$
FI}
 ~
 
  \Abs[iter]{itDiccNat(significado)}[i]{it}{Siguientes($it$) $\igobs$ secuDFS($i$)}

  ~

  \tadOperacion{secuDFS}{iter/i}{secu$\big($tupla(nat, significado)$\big)$}{Rep($i$)}
  \tadAxioma{secuDFS($i$)}
{\IF vac�a?($i$) THEN
  <> 
ELSE
  {\IF tope($i$) $\to$ der $\nigobs$ NULL $\land$ tope($i$) $\to$ izq $\nigobs$ NULL THEN
    secuDFS$\Big($apilar$\big($tope($i$) $\to$ izq, apilar$\big($tope($i$) $\to$ der, desapilar($i$)$\big)\big)\Big)$ $\circulito$ $\big<$tope($i$) $\to$ clave, *(tope($i$) $\to$ significado)$\big>$ 
  ELSE
    {\IF tope($i$) $\to$ der $\nigobs$ NULL THEN
      secuDFS$\Big($apilar$\big($tope($i$) $\to$ der, desapilar($i$)$\big)\Big)$ $\circulito$ $\big<$tope($i$) $\to$ clave, *(tope($i$) $\to$ significado)$\big>$ 
    ELSE
      {\IF tope($i$) $\to$ izq $\nigobs$ NULL THEN
        secuDFS$\Big($apilar$\big($tope($i$) $\to$ izq, desapilar($i$)$\big)\Big)$ $\circulito$ $\big<$tope($i$) $\to$ clave, *(tope($i$) $\to$ significado)$\big>$> 
      ELSE
        secuDFS$\big($desapilar($i$)$\big)$ $\circulito$ $\big<$tope($i$) $\to$ clave, *(tope($i$) $\to$ significado)$\big>$>
      FI}
    FI}
  FI}
FI}

  ~


\end{Representacion}

\bigskip

\begin{Algoritmos}

\medskip
	
 \Titulo{Algoritmos de diccNat}
  	\medskip
  
\begin{algorithm}[H]{\textbf{iVacio}() $\to$ $res$ : estrDiccNat}
      \begin{algorithmic}
       \State $res \gets NULL$         \Comment $O(1)$

      \medskip
      \Statex \underline{Complejidad:} {$O(1)$}
      \Statex \underline{Justificaci�n:} {Apuntar a $\texttt{NULL}$ un puntero es $O(1)$}

      \end{algorithmic}
\end{algorithm}

  
\begin{algorithm}[H]{\textbf{iDefinir}(\In{n}{nat}, \In{s}{significado}, \Inout{dicc}{estrDiccNat})}
  \begin{algorithmic}
    \State $diccAux \gets dicc$         \comentariocompl{Copiamos el PUNTERO}{$O(1)$}
    \State $termine \gets false$ \Comment $O(1)$
    \While{$\NOT termine$} \Comment $O\big(\#claves(dicc) * ...\big)\ /\ O\big(log\ \#claves(dicc) * ...\big)\ promedio\ con\ claves\ uniformes$
      \If{$diccAux = NULL$} \Comment $O(1)$
        \State $dicc \gets \&\big\langle n, \&s, NULL, NULL\big\rangle$ \Comment $O(1)$
        \State $termine \gets true$ \Comment $O(1)$
      \Else
        \If{$(diccAux \to clave) < n$} \Comment $O(1)$
          \If{$\NOT diccAux \to izq = NULL$} \Comment $O(1)$
            \State $diccAux \gets (diccAux \to izq)$ \comentariocompl{Copiamos el PUNTERO}{$O(1)$}
          \Else
            \State $(diccAux \to izq) \gets \&\big\langle n, \&s, NULL, NULL\big\rangle$ \Comment $O(1)$
             \State $termine \gets true$ \Comment $O(1)$
          \EndIf
        \Else
          \If{$\NOT diccAux \to der = NULL$} \Comment $O(1)$
            \State $diccAux \gets (diccAux \to der)$ \comentariocompl{Copiamos el PUNTERO}{$O(1)$}
          \Else
            \State $(diccAux \to der) \gets \&\big\langle n, \&s, NULL, NULL\big\rangle$ \Comment $O(1)$
            \State $termine \gets true$ \Comment $O(1)$
          \EndIf
        \EndIf
      \EndIf

    \EndWhile

    \medskip
    \Statex \underline{Complejidad:} {\\
\quad\quad En el peor caso: $O\big(\#claves(dicc)\big)$ \\
\quad\quad En promedio: $O\big(log\ \#claves(dicc)\big)$}
    \Statex \underline{Justificaci�n:} {\\
\quad\quad En el peor caso se definen todas las claves ordenadas y queda un �rbol derechista o izquierdista, por eso para definir tenemos que recorrer todas las claves y tenemos $O(n)$, donde $n$ es la cantidad de claves del diccionario.\\
\quad\quad Pero como el enunciado del trabajo pr�ctico dice que los valores nat se insertan con probabilidad uniforme, cada vez que bajemos un nivel en la altura del �rbol nos vamos a quedar con la mitad de claves, por lo tanto tenemos una complejidad de $O(log\ n)$ en promedio, donde $n$ es la cantidad de claves del diccionario.}

  \end{algorithmic}
\end{algorithm}

\begin{algorithm}[H]{\textbf{iDef?}(\In{n}{nat}, \In{dicc}{estrDiccNat}) $\to$ $res$ : bool}
  \begin{algorithmic}
    \State $diccAux \gets dicc$         \comentariocompl{Copiamos el PUNTERO}{$O(1)$}
    \State $termine \gets false$ \Comment $O(1)$
    \While{$\NOT termine$} \Comment $O\big(\#claves(dicc) * ...\big)\ /\ O\big(log\ \#claves(dicc) * ...\big)\ promedio\ con\ claves\ uniformes$
      \If{$diccAux = NULL$} \Comment $O(1)$
        \State $res \gets false$ \Comment $O(1)$
        \State $termine \gets true$ \Comment $O(1)$
      \Else
        \If{$(diccAux \to clave) = n$} \Comment $O(1)$
          \State $termine \gets true$ \Comment $O(1)$
          \State $res \gets true$ \Comment $O(1)$
        \Else
          \If{$(diccAux \to clave) < n$} \Comment $O(1)$
            \State $diccAux \gets (diccAux \to izq)$ \comentariocompl{Copiamos el PUNTERO}{$O(1)$}
          \Else
            \State $diccAux \gets (diccAux \to der)$ \comentariocompl{Copiamos el PUNTERO}{$O(1)$}
          \EndIf
        \EndIf
      \EndIf

    \EndWhile

    \medskip
    \Statex \underline{Complejidad:} {\\
\quad\quad En el peor caso: $O\big(\#claves(dicc)\big)$ \\
\quad\quad En promedio: $O\big(log\ \#claves(dicc)\big)$}
    \Statex \underline{Justificaci�n:} {\\
\quad\quad En el peor caso se definen todas las claves ordenadas y queda un �rbol derechista o izquierdista, por eso para buscar si est� definido, tenemos que recorrer todas las claves y tenemos $O(n)$, donde $n$ es la cantidad de claves del diccionario.\\
\quad\quad Pero como el enunciado del trabajo pr�ctico dice que los valores nat se insertan con probabilidad uniforme, cada vez que bajemos un nivel en la altura del �rbol nos vamos a quedar con la mitad de claves, por lo tanto tenemos una complejidad de $O(log\ n)$ en promedio, donde $n$ es la cantidad de claves del diccionario.}

  \end{algorithmic}
\end{algorithm}

\begin{algorithm}[H]{\textbf{iObtener}(\In{n}{nat}, \In{dicc}{estrDiccNat}) $\to$ $res$ : significado}
  \begin{algorithmic}
    \State $diccAux \gets dicc$         \comentariocompl{Copiamos el PUNTERO}{$O(1)$}
    \State $termine \gets false$ \Comment $O(1)$
    \While{$\NOT termine$} \Comment $O\big(\#claves(dicc) * ...\big)\ /\ O\big(log\ \#claves(dicc) * ...\big)\ promedio\ con\ claves\ uniformes$
      \If{$(diccAux \to clave) < n$} \Comment $O(1)$
        \State $diccAux \gets (diccAux \to izq)$ \comentariocompl{Copiamos el PUNTERO}{$O(1)$}
      \Else
        \If{$(diccAux \to clave) = n$} \Comment $O(1)$
          \State $res \gets (diccAux \to significado)$ \comentariocompl{lo pasamos por referencia}{$O(1)$}
        \Else
          \State $diccAux \gets (diccAux \to der)$ \comentariocompl{Copiamos el PUNTERO}{$O(1)$}
        \EndIf
      \EndIf
    \EndWhile

    \medskip
    \Statex \underline{Complejidad:} {\\
\quad\quad En el peor caso: $O\big(\#claves(dicc)\big)$ \\
\quad\quad En promedio: $O\big(log\ \#claves(dicc)\big)$}
    \Statex \underline{Justificaci�n:} {\\
\quad\quad En el peor caso se definen todas las claves ordenadas y queda un �rbol derechista o izquierdista, por eso para obtener el significado, tenemos que recorrer todas las claves y tenemos $O(n)$, donde $n$ es la cantidad de claves del diccionario.\\
\quad\quad Pero como el enunciado del trabajo pr�ctico dice que los valores nat se insertan con probabilidad uniforme, cada vez que bajemos un nivel en la altura del �rbol nos vamos a quedar con la mitad de claves, por lo tanto tenemos una complejidad de $O(log\ n)$ en promedio, donde $n$ es la cantidad de claves del diccionario.}

  \end{algorithmic}
\end{algorithm}


\begin{algorithm}[H]{\textbf{iBorrar}(\In{n}{nat}, \Inout{dicc}{estrDiccNat})}
  \begin{algorithmic}
    \State $diccAux \gets dicc$         \comentariocompl{Copiamos el PUNTERO}{$O(1)$}
    \State $termine \gets false$ \Comment $O(1)$
    \State $padre \gets NULL$ \Comment $O(1)$
    \While{$\NOT termine$} \Comment $O\big(\#claves(dicc) * ...\big)\ /\ O\big(log\ \#claves(dicc) * ...\big)\ promedio\ con\ claves\ uniformes$
      \If{$(diccAux \to clave) < n$} \Comment $O(1)$
        \State $padre \gets diccAux$ \Comment $O(1)$
        \State $diccAux \gets (diccAux \to izq)$ \comentariocompl{Copiamos el PUNTERO}{$O(1)$}
      \Else
        \If{$(diccAux \to clave) = n$} \Comment $O(1)$
          \State $termine \gets true$ \Comment $O(1)$
        \Else
          \State $padre \gets diccAux$ \Comment $O(1)$
          \State $diccAux \gets (diccAux \to der)$ \comentariocompl{Copiamos el PUNTERO}{$O(1)$}
        \EndIf
      \EndIf
    \EndWhile
    \BState

    \comentario{Caso hoja}
    \If{$(diccAux \to izq) = NULL\ \AND (diccAux \to der) = NULL$} \Comment $O(1)$
      \If{(padre \to izq) = diccAux} \Comment $O(1)$
        \State $(padre \to izq) \gets NULL$ \Comment $O(1)$
      \Else
        \State $(padre \to der) \gets NULL$ \Comment $O(1)$
      \EndIf

    \comentario{Caso un s�lo hijo (derecho)}
    \ElsIf{$(diccAux \to izq) = NULL\ \AND \NOT (diccAux \to der) = NULL$} \Comment $O(1)$
      \If{(padre \to izq) = diccAux} \Comment $O(1)$
        \State $(padre \to izq) \gets (diccAux \to der)$ \Comment $O(1)$
      \Else
        \State $(padre \to der) \gets (diccAux \to der)$ \Comment $O(1)$
      \EndIf

    \comentario{Caso un s�lo hijo (izquierdo)}
    \ElsIf{$\NOT (diccAux \to izq) = NULL\ \AND (diccAux \to der) = NULL$} \Comment $O(1)$
      \If{(padre \to izq) = diccAux} \Comment $O(1)$
        \State $(padre \to izq) \gets (diccAux \to izq)$ \Comment $O(1)$
      \Else
        \State $(padre \to der) \gets (diccAux \to izq)$ \Comment $O(1)$
      \EndIf

    \comentario{Caso dos hijos}
    \ElsIf{$\NOT (diccAux \to izq) = NULL\ \AND \NOT (diccAux \to der) = NULL$} \Comment $2*O\big(log\ \#claves(dicc)\big)$
      \State $temp \gets Min(diccAux \to der)$\\ \Comment $O\big(\#claves(dicc) * ...\big)\ /\ O\big(log\ \#claves(dicc) * ...\big)\ promedio\ con\ claves\ uniformes$
      \State $Borrar(temp.clave, dicc)$ \comentariorandom{entra en caso hoja o caso un s�lo hijo por caracter�sticas de m�nimo}
      \BState

      \State $(diccAux \to clave) \gets temp.clave$ \Comment $O(1)$
      \State $(diccAux \to significado) \gets temp.significado$ \Comment $O(1)$
      
    \EndIf

    \medskip
    \Statex \underline{Complejidad:} {\\
\quad\quad En el peor caso: $O\big(\#claves(dicc)\big)$ \\
\quad\quad En promedio: $O\big(log\ \#claves(dicc)\big)$}
    \Statex \underline{Justificaci�n:} {\\
\quad\quad En el peor caso se definen todas las claves ordenadas y queda un �rbol derechista o izquierdista, por eso para borrar la clave, tenemos que recorrer todas las claves y tenemos $O(n)$, donde $n$ es la cantidad de claves del diccionario.\\
\quad\quad Pero como el enunciado del trabajo pr�ctico dice que los valores nat se insertan con probabilidad uniforme, cada vez que bajemos un nivel en la altura del �rbol nos vamos a quedar con la mitad de claves, por lo tanto tenemos una complejidad de $O(log\ n)$ en promedio, donde $n$ es la cantidad de claves del diccionario.}

  \end{algorithmic}
\end{algorithm}

\begin{algorithm}[H]{\textbf{iMin}(\In{dicc}{estrDiccNat}) $\to$ $res$ : tupla\big\langle nat, significado\big\rangle}
  \begin{algorithmic}
    \State $diccAux \gets dicc$ \Comment $O(1)$
    \While{$\NOT (diccAux \to izq) = NULL$}\\ \Comment $O\big(\#claves(dicc) * ...\big)\ /\ O\big(log\ \#claves(dicc) * ...\big)\ promedio\ con\ claves\ uniformes$
      \State $diccAux \gets (diccAux \to izq)$ \Comment $O(1)$
    \EndWhile
    \State $res \gets \big\langle diccAux \to clave, diccAux \to significado\big\rangle$ \Comment $O(1)$
     
    \medskip
    \Statex \underline{Complejidad:} {\\
\quad\quad En el peor caso: $O\big(\#claves(dicc)\big)$ \\
\quad\quad En promedio: $O\big(log\ \#claves(dicc)\big)$}
    \Statex \underline{Justificaci�n:} {\\
\quad\quad En el peor caso se definen todas las claves ordenadas y queda un �rbol derechista o izquierdista, por eso para encontrar la clave m�nima, tenemos que recorrer todas las claves y tenemos $O(n)$, donde $n$ es la cantidad de claves del diccionario.\\
\quad\quad Pero como el enunciado del trabajo pr�ctico dice que los valores nat se insertan con probabilidad uniforme, cada vez que bajemos un nivel en la altura del �rbol nos vamos a quedar con la mitad de claves, por lo tanto tenemos una complejidad de $O(log\ n)$ en promedio, donde $n$ es la cantidad de claves del diccionario.}

  \end{algorithmic}
\end{algorithm}

\begin{algorithm}[H]{\textbf{iMax}(\In{dicc}{estrDiccNat}) $\to$ $res$ : tupla\big\langle nat, significado\big\rangle}
  \begin{algorithmic}
    \State $diccAux \gets dicc$ \Comment $O(1)$
    \While{$\NOT (diccAux \to der) = NULL$}\\ \Comment $O\big(\#claves(dicc) * ...\big)\ /\ O\big(log\ \#claves(dicc) * ...\big)\ promedio\ con\ claves\ uniformes$
      \State $diccAux \gets (diccAux \to der)$ \Comment $O(1)$
    \EndWhile
    \State $res \gets \big\langle diccAux \to clave, diccAux \to significado\big\rangle$ \Comment $O(1)$
     
    \medskip
    \Statex \underline{Complejidad:} {\\
\quad\quad En el peor caso: $O\big(\#claves(dicc)\big)$ \\
\quad\quad En promedio: $O\big(log\ \#claves(dicc)\big)$}
    \Statex \underline{Justificaci�n:} {\\
\quad\quad En el peor caso se definen todas las claves ordenadas y queda un �rbol derechista o izquierdista, por eso para encontrar la clave m�xima, tenemos que recorrer todas las claves y tenemos $O(n)$, donde $n$ es la cantidad de claves del diccionario.\\
\quad\quad Pero como el enunciado del trabajo pr�ctico dice que los valores nat se insertan con probabilidad uniforme, cada vez que bajemos un nivel en la altura del �rbol nos vamos a quedar con la mitad de claves, por lo tanto tenemos una complejidad de $O(log\ n)$ en promedio, donde $n$ es la cantidad de claves del diccionario.}

  \end{algorithmic}
\end{algorithm}


\Titulo{Algoritmos del iterador}
    \medskip
  
\begin{algorithm}[H]{\textbf{iCrearIt}(\In{dicc}{estrDiccNat}) $\to$ $res$ : itDiccNat}
  \begin{algorithmic}
    \State $res \gets Vacia()$ \Comment $O(1)$
    \If{$\NOT dicc = NULL$} \Comment $O(1)$
      \State $Apilar(res, *dicc)$ \Comment $O\big(copy(*dicc)\big) = O(1)$
    \EndIf

    \medskip
    \Statex \underline{Complejidad:} {$O(1)$}
    \Statex \underline{Justificaci�n:} {Se llama �nicamente a la funci�n apilar del m�dulo pila del apunte de m�dulos b�sicos. Como se copia un nat la complejidad es $O(1)$.}
  \end{algorithmic}
\end{algorithm}


\begin{algorithm}[H]{\textbf{iSiguientes}(\In{it}{iter}) $\to$ $res$ : lista\big(tupla\big\langle nat, significado\big\rangle\big)}
  \begin{algorithmic}
    \comentario{DFS}
    \State $res \gets Vacia()$ \Comment $O(1)$
    \State $iterador \gets Copiar(it)$
    \While{$\NOT EsVacia?(iterador)$} \Comment $O\big(n * copy(significado\ mas\ costoso))$
      \State $prox \gets Desapilar(iterador)$ \Comment $O(1)$
      \State $AgregarAtras\big(res, \big\langle prox \to clave, prox \to significado\big\rangle\big)$ \Comment $O\big(copy(tupla\langle nat, significado\big\rangle)\big)$
      \If{$\NOT prox \to der = NULL$} \Comment $O(1)$
        \State $Apilar(iterador, prox \to der)$ \Comment $O(1)$
      \EndIf

      \If{$\NOT prox \to izq = NULL$} \Comment $O(1)$
        \State $Apilar(iterador, prox \to izq)$ \Comment $O(1)$
      \EndIf

    \EndWhile

    \medskip
    \Statex \underline{Complejidad:} {$O(n*copy(x))$ con $n$ siendo la cantidad de claves del diccionario y $x$ siendo el significado m�s costoso de copiar.
    }
    \Statex \underline{Justificaci�n:} {Tiene un ciclo principal y todas las dem�s operaciones $O(1)$. En el ciclo principal se itera $n$ veces y se copia en cada iteraci�n una tupla<nat, significado>. Como copiar un nat es $O(1)$, nos queda una complejidad de $O\big(n * copy(significado\ mas\ costoso)\big)$}
  \end{algorithmic}
\end{algorithm}

\begin{algorithm}[H]{\textbf{iAvanzar}(\Inout{it}{iter})}
  \begin{algorithmic}
    \State $prox \gets Desapilar(it)$ \Comment $O(1)$
   
    \If{$\NOT prox \to der = NULL$} \Comment $O(1)$
      \State $Apilar(it, prox \to der)$ \Comment $O(copy(puntero)) = O(1)\ (al\ menos\ una\ sola\ vez)$
    \EndIf

    \If{$\NOT prox \to izq = NULL$} \Comment $O(1)$
      \State $Apilar(it, prox \to izq)$ \Comment $O(1)$
    \EndIf

    \medskip
    \Statex \underline{Complejidad:} {$O(1)$}
    \Statex \underline{Justificaci�n:} {La funcion desapilar es $O(1)$, luego se llama dos veces a la funci�n apilar del modulo basico pila que tiene una complejidad de $O\big(copy(\alpha)\big)$, pero como $\alpha$ en este algoritmo es un puntero, y copiar un puntero es $O(1)$, entonces la complejidad del algoritmo es $O(1)$.}
  \end{algorithmic}
\end{algorithm}

\begin{algorithm}[H]{\textbf{iHayMas?}(\In{it}{iter}) $\to$ $res$ : bool}
  \begin{algorithmic}
    \State $res \gets \big(\NOT EsVacia?(it)\big)$ \Comment $O(1)$
    
    \medskip
    \Statex \underline{Complejidad:} {$O(1)$}
    \Statex \underline{Justificaci�n:} {Se llama �nicamente a la funci�n EsVacia del m�dulo Pila del apunte de m�dulos b�sicos, que tiene complejidad $O(1)$.}
  \end{algorithmic}
\end{algorithm}


\begin{algorithm}[H]{\textbf{iActual}(\In{it}{iter}) $\to$ $res$ : tupla\big\langle nat, significado\big\rangle}
  \begin{algorithmic}
    \State $res \gets \big\langle Tope(it) \to clave, Tope(it) \to significado\big\rangle)$

    \medskip
    \Statex \underline{Complejidad:} {$O(1)$}
    \Statex \underline{Justificaci�n:} {No se copia la tupla actual del iterador, si no que se pasa por referencia. Por lo tanto es $O(1)$.}
  \end{algorithmic}
\end{algorithm}

\end{Algoritmos}



\newpage
\section{M�dulo Tabla}

\begin{Interfaz}

  \textbf{se explica con}: \tadNombre{Tabla}.

  \textbf{g�neros}: \TipoVariable{tabla}.

  \Titulo{Operaciones b�sicas de tabla}

  \InterfazFuncion{Nombre}{\In{t}{tabla}}{string}
  [true]
  {alias$\big(res$ $\igobs$ nombre($t$)$\big)$}
  [$O(1)$]
  [devuelve el nombre de la tabla indicada.]
  [se pasa por referencia. No es modificable (const).]

  \InterfazFuncion{Claves}{\In{t}{tabla}}{itBi(campo)}
  [true]
  {alias$\big(res$ $\igobs$ crearIt(claves($t$))$\big)$}
  [O(1)]
  [devuelve un iterador al conjunto de campos claves de la tabla indicada.]
  [se pasa por referencia. No es modificable (const)]

  \InterfazFuncion{Buscar}{\In{c}{campo}, \In{d}{dato}, \In{t}{tabla}}{secu(registro)}
  [$c$ $\in$ campos($t$) $\yluego$ $\big($tipoCampo($c$, $t$) $\igobs$ nat?($d$)$\big)$]
  {($\forall$ $r$ : registro) def?($c$, $r$) $\implies$ $\big($($r$ $\in$ registros($t$) $\land$ obtener($c$, $r$) $\igobs$ $d$) $\ssi$ esta?($r$, $res$)$\big)$}
  [\\
  \tab Campo indexado nat y clave $\implies$ $O\big(log\ n\ +\ |L|\big)$ promedio. \\
  \tab Campo indexado nat y no clave $\implies$ $O\big(log\ n\ +\ n * |L|\big)$ promedio. \\
  \\
  \tab Campo indexado String y clave $\implies$ $O(|L|\ +\ |L|)$ = $O(|L|)$. \\
  \tab Campo indexado String y no clave $\implies$ $O\big(|L|\ +\ n * |L|\big)$ = $O(n * |L|)$. \\
  \\
  \tab Campo NO indexado $\implies$ $O(n * |L|)$. \\
  \\
  \tab Donde $n$ es la cantidad de registros de la tabla pasada por argumento y |$L$| corresponde a la longitud m�xima de cualquier valor string de datos de la tabla.
  ]
  [Busca en todos los registros de la tabla los que tengan el dato $d$ en el campo $c$, esos registros los devuelve en una secuencia.]
  [no hay ya que se copian los registros.]

  \InterfazFuncion{Indices}{\In{t}{tabla}}{conj(campo)}
  [true]
  {$res$ $\igobs$ indices($t$)}
  [$O(1)$]
  [devuelve el conjunto de campos con �ndice de la tabla indicada.]
  [no hay aliasing, se devuelve por copia.]

  \InterfazFuncion{Campos}{\In{t}{tabla}}{itBi(campo)}
  [true]
  {$res$ $\igobs$ crearIt(campos($t$))}
  [$O(1)$]
  [devuelve un iterador al conjunto de campos (devuelto por copia) de la tabla indicada.]


  \InterfazFuncion{TipoCampo}{\In{c}{campo}, \In{t}{tabla}}{bool}
  [$c$ $\in$ campos($t$)]
  {$res$ $\igobs$ tipoCampo($c$, $t$)}
  [$O(1)$]
  [devuelve \texttt{true} si el tipo de campo es nat y \texttt{false} si el tipo de campo es string.]

  \InterfazFuncion{Registros}{\In{t}{tabla}}{conj(registro)}
  [true]
  {$res$ $\igobs$ crearIt(registros($t$))}
  [$O(1)$]
  [devuelve un iterador al conjunto de registros de la tabla indicada.]
  [hay aliasing, pero no es modificable.]

  \InterfazFuncion{CantidadDeAccesos}{\In{t}{tabla}}{nat}
  [true]
  {$res$ $\igobs$ cantidadDeAccesos($t$)}
  [$O(1)$]
  [devuelve la cantidad de accesos de la tabla indicada.]


  \InterfazFuncion{NuevaTabla}{\In{nombre}{string}, \In{claves}{conj(campo)}, \In{columnas}{registro}}{tabla}
  [$claves$ $\nigobs$ $\emptyset$ $\land$ $claves$ $\incluido$ claves($columnas$)]
  {$res$ $\igobs$ nuevaTabla($nombre$, $claves$, $columnas$)}
  [$O(1)$]
  [genera una tabla con los valores ingresados.]
  [hay aliasing, tabla es modificable]

  \InterfazFuncion{AgregarRegistro}{\In{r}{registro}, \Inout{t}{tabla}}{}
  [campos($r$) $\igobs$ campos($t$) $\land$ puedoInsertar?($r$, $t$) $\land$ $t_{0}$ = $t$]
  {$t$ $\igobs$ agregarRegistro($r$, $t_0$)}
  [\\
  \tab Campo indexado $\implies$ En caso promedio $O(|L| + log\ n)$, donde $n$ es la cantidad de registros($t$) y $L$ es el string m�s largo de $r$.\\
  \tab Campo no indexado $\implies$ $O(|S|)$, donde $S$ es el string m�s largo de $r$.]
  [agrega un registro a la tabla.]

  \InterfazFuncion{BorrarRegistro}{\In{criterio}{registro}, \Inout{t}{tabla}}{}
  [\#campos($criterio$) $\igobs$ 1 $\yluego$ dameUno$\big($campos($criterio$)$\big)$ $\in$ claves($t$) $\land$ $t_0$ = $t$]
  {$t$ $\igobs$ borrarRegistro($criterio$, $t_0$)}
  [\\
  \tab Criterio sobre campo indexado $\implies$ $O(log\ n\ +\ L)$. \\
  \tab Criterio sobre campo no indexado $\implies$ $O(n * |L|)$. \\
  \tab Donde $n$ es la cantidad total de registros de la tabla y $L$ el valor string m�s largo de todos los datos comparados.]
  [borra un registro de la tabla.]

  \InterfazFuncion{Indexar}{\In{c}{campo}, \Inout{t}{tabla}}{}
  [puedeIndexar($c$, $t$)]
  {$t$ $\igobs$ indexar($c$, $t$)}
  [$O\big(|registros| * L * (L\ +\ log\ |registros|)\big)$, donde $L$ es el m�ximo string para el campo $c$ en cualquier registro.]
  [indexa un campo de la tabla.]

  \InterfazFuncion{Minimo}{\In{c}{campo}, \In{t}{tabla}}{dato}
  [$\neg$vacio?(registros($t$)) $\land$ $c$ $\in$ indices($t$)]
  {alias($res$ $\igobs$ $m$) | nat?($m$) $\implies$ valorNat($m$) $\igobs$ valorNat$\big($minimo($c$, $t$)$\big)$ $\land$ $\neg$nat?($m$) $\implies$ valorStr($m$) $\igobs$ valorStr$\big($minimo($c$, $t$)$\big)$}
  [$O(1)$]
  [devuelve el minimo de una tabla por referencia de un campo indexado. $res$ no es modificable]
  [$res$ no es modificable.]

  \InterfazFuncion{Maximo}{\In{c}{campo}, \In{t}{tabla}}{dato}
  [$\neg$vacio?(registros($t$)) $\land$ $c$ $\in$ indices($t$)]
  {alias($res$ $\igobs$ $m$) | tipoCampo($c$, $m$) $\implies$ $\Big($nat?($m$) $\land$ $\big($valorNat($m$) $\igobs$ valorNat(maximo$\big(c$, $t$)$\big)\big)\Big)$ $\land$ $\neg$tipoCampo($c$, $m$) $\implies$ $\Big(\neg$nat?($m$) $\land$ $\big($valorStr($m$) $\igobs$ valorStr(maximo$\big(c$, $t$)$\big)\big)\Big)$}
  [$O(1)$]
  [devuelve el maximo de una tabla por referencia de un campo indexado. $res$ no es modificable]
  [$res$ no es modificable.]




\end{Interfaz}

\begin{Representacion}

  \Titulo{Representaci�n de tabla}

  \begin{Estructura}{tabla}[estrTabla]
    \begin{flushright}
      \dondees{estrTabla}{tupla}$\Big($
      \emph{indicesString}: \TipoVariable{diccString\big(conj(itConj(registro))\big)}, \veryquad \\
        \emph{indicesNat}: \TipoVariable{diccNat\big(conj(itConj(registro))\big)},
        \emph{registros}: \TipoVariable{conj(registro)}, \veryquad \\
        \emph{nombre}: \TipoVariable{string},
        \emph{campos}: \TipoVariable{diccString(bool esNat?)},
        \emph{claves}: \TipoVariable{conj(campo)}, \veryquad \\
        \emph{campoIndexadoNat}: \TipoVariable{lista(tupla<}\emph{nombre}: \TipoVariable{campo}, \emph{max}: \TipoVariable{dato}, \emph{min}: \TipoVariable{dato}, \emph{vacio?}: \TipoVariable{bool>)}, \veryquad \\
        \emph{campoIndexadoString}: \TipoVariable{lista(tupla<}\emph{nombre}: \TipoVariable{campo}, \emph{max}: \TipoVariable{dato}, \emph{min}: \TipoVariable{dato}, \emph{vacio?}: \TipoVariable{bool>)},
        \emph{cantAccesos}: \TipoVariable{nat}
        $\Big)$ \veryquad
      \end{flushright}
     \dondees{registro}{diccString(dato)}y se explica con \tadNombre{Registro}, y \TipoVariable{conj} corresponde al conjunto lineal de la c�tedra.
  \end{Estructura}

  \noindent\textbf{Invariante de representaci�n}

  \noindent\tab 1) Las claves de indicesString corresponden al valor del campo indexado para cada registro que est� en sus significados. \\
  \noindent\tab 2) Las claves de indicesNat corresponden al valor del campo indexado para cada registro que est� en sus significados. \\
  \noindent\tab 3) Los significados de indicesString e indicesNat pertenecen a registros. \\
  \noindent\tab 4) Todos los registros estan indexados. \\
  \noindent\tab 5) Claves esta entre los campos y no es vacio. \\
  \noindent\tab 6) Todos los valores de los registros son menores o iguales al m�ximo y mayor o iguales al m�nimo para cada campo indexado. \\
  \noindent\tab 7) Para cada campo indexado, hay un registro cuyo valor en ese campo es el maximo y un registro cuyo valor es el minimo. \\
  \noindent\tab 8) Si un campo es clave no puede haber dos registros con mismo dato en ese campo. \\
  \noindent\tab 9) El tipo de dato en registro corresponde al tipo de dato en campos y las claves de los registros son los campos de la tabla. \\
  \noindent\tab 10) El campo indexado pertenece a campos. \\
  \noindent\tab 11) cantAccesos es mayor o igual a la cantidad de registros. \\
  \noindent\tab 12) Tama�o de las listas 'campoIndexado...' es menor o igual a 1. \\
  \noindent\tab 13) El bool 'vacio?' de las tuplas de campoIndexado valen true si y solo si sus respectivos diccionarios est�n vac�os. \\
  \noindent\tab 14) Si no hay un campoIndexado de cierto tipo, el diccionario correspodiente, esta vac�o.\\
  \noindent\tab 15) No hay dos registros con mismo valor para un campo clave. \\

  \Rep[estrTabla][e]{}
    \noindent\tab\tab\textbf{1)} $\big(\neg$vacia?($e$.campoIndexadoString)$\big)$ $\implies$\\
    \noindent\tab\tab\bigpar{$\big(\forall$ $c$ : string, $c$ $\in$ claves($e$.indicesString)$\big)$
    $\big(\forall$ $r$ : itConj(registro), $r$ $\in$ obtener($c$, $e$.indicesString)$\big)$\\
    $\Big($valorStr$\big($obtener$\big($campoIndexString, siguiente($r$)$\big)\big)$ = $c\Big)$
    }\\
    \noindent\tab\tab\textbf{2)} $\big(\neg$vacia?($e$.campoIndexadoNat)$\big)$ $\implies$\\
    \noindent\tab\tab\bigpar{$\big(\forall$ $c$ : nat, $c$ $\in$ claves($e$.indicesNat)$\big)$
    $\big(\forall$ $r$ : itConj(registro), $r$ $\in$ obtener($c$, $e$.indicesNat)$\big)$\\
    $\Big($valorNat$\big($obtener$\big($campoIndexNat, siguiente($r$)$\big)\big)$ = $c\Big)$
    }\\
    \noindent\tab\tab\textbf{3)} $\big(\neg$vacia?($e$.campoIndexadoString)$\big)$ $\implies$\\
    \noindent\tab\tab\bigpar{$\big(\forall$ $c$ : string, $c$ $\in$ claves($e$.indicesString)$\big)$
    $\big(\forall$ $r$ : itConj(registro), $r$ $\in$ obtener($c$, $e$.indicesString)$\big)$\\
    $\big($siguiente($r$) $\in$ $e$.registros$\big)$
    }\\
    \noindent\tab\tab\textbf{3 bis)} $\big(\neg$vacia?($e$.campoIndexadoNat)$\big)$ $\implies$\\
    \noindent\tab\tab\bigpar{$\big(\forall$ $c$ : nat, $c$ $\in$ claves($e$.indicesNat)$\big)$
    $\big(\forall$ $r$ : itConj(registro), $r$ $\in$ obtener($c$, $e$.indicesNat)$\big)$\\
    $\big($siguiente($r$) $\in$ $e$.registros$\big)$
    }\\
    \noindent\tab\tab\textbf{4)} $\big(\neg$vacia?($e$.campoIndexadoString)$\big)$ $\implies$\\
    \noindent\tab\tab\bigpar{$\big(\forall$ $r$ : registro, $r$ $\in$ $e$.registros$\big)$
    $\big(\exists$ $it$ : itConj(registro), siguiente($it$) $\igobs$ $r\big)$\\
    $it$ $\in$ obtener$\big($valorStr$\big($obtener(campoIndexString, $r$)$\big)$, $e$.indicesString$\big)$
    }\\
    \noindent\tab\tab\textbf{4 bis)} $\big(\neg$vacia?($e$.campoIndexadoNat)$\big)$ $\implies$\\
    \noindent\tab\tab\bigpar{$\big(\forall$ $r$ : registro, $r$ $\in$ $e$.registros$\big)$
    $\big(\exists$ $it$ : itConj(registro), siguiente($it$) $\igobs$ $r\big)$\\
    $it$ $\in$ obtener$\big($valorNat$\big($obtener(campoIndexNat, $r$)$\big)$, $e$.indicesNat$\big)$
    }\\
    \noindent\tab\tab\textbf{5)} $\big(\forall$ $c$ : campo, $c$ $\in$ $e$.claves$\big)$ $\big(c$ $\in$ claves($e$.campos) $\land$ \#$e$.claves > 0$\big)$\\
    \noindent\tab\tab\textbf{6)} $\Big(\big(\neg$vacia?($e$.campoIndexadoNat)$\big)$ $\yluego$ $\neg\big(\big($prim($e$.campoIndexadoNat)$\big)$.vacio?$\big)\Big)$ $\implies$ \\
    \noindent\tab\tab\bigpar{$\big(\forall$ $r$ : registro, $r$ $\in$ $e$.registros$\big)$\\
    $\Big(\big($prim($e$.campoIndexadoNat)$\big)$.min $\le$ obtener($campoIndexNat$, $r$) $\le$ $\big($prim($e$.campoIndexadoNat)$\big)$.max$\Big)$
    }\\
    \noindent\tab\tab\textbf{6 bis)} $\Big(\big(\neg$vacia?($e$.campoIndexadoString)$\big)$ $\yluego$ $\neg\big(\big($prim($e$.campoIndexadoString)$\big)$.vacio?$\big)\Big)$ $\implies$ \\
    \noindent\tab\tab\bigpar{$\big(\forall$ $r$ : registro, $r$ $\in$ $e$.registros$\big)$\\
    $\Big(\big($prim($e$.campoIndexadoString)$\big)$.min $\le$ obtener($campoIndexString$, $r$) $\le$ $\big($prim($e$.campoIndexadoString)$\big)$.max$\Big)$
    }\\
    \noindent\tab\tab\textbf{7)} $\Big(\big(\neg$vacia?($e$.campoIndexadoString)$\big)$ $\yluego$ $\neg\big(\big($prim($e$.campoIndexadoString)$\big)$.vacio?$\big)\Big)$ $\implies$ \\
    \noindent\tab\tab\bigpar{$\big(\exists$ $r$, $r'$ : registro, $r$ $\in$ $e$.registros $\land$ $r'$ $\in$ $e$.registros$\big)$\\
    $\Big($obtener(campoIndexString, $r$) $\igobs$ $\big($prim($e$.campoIndexadoString)$\big)$.max $\land$ \\
    obtener(campoIndexString, $r'$) $\igobs$ $\big($prim($e$.campoIndexadoString)$\big)$.min$\Big)$
    }\\
    \noindent\tab\tab\textbf{7 bis)} $\Big(\big(\neg$vacia?($e$.campoIndexadoNat)$\big)$ $\yluego$ $\neg\big(\big($prim($e$.campoIndexadoNat)$\big)$.vacio?$\big)\Big)$ $\implies$ \\
    \noindent\tab\tab\bigpar{$\big(\exists$ $r$, $r'$ : registro, $r$ $\in$ $e$.registros $\land$ $r'$ $\in$ $e$.registros$\big)$\\
    $\Big($obtener(campoIndexNat, $r$) $\igobs$ $\big($prim($e$.campoIndexadoNat)$\big)$.max $\land$ \\
    obtener(campoIndexNat, $r'$) $\igobs$ $\big($prim($e$.campoIndexadoNat)$\big)$.min$\Big)$
    }\\
    \noindent\tab\tab\textbf{8)} $\big(\forall$ $c$ : campo, $c$ $\in$ $e$.claves$\big)$ $\big(\forall$ $x$, $y$ : registro, $x$ $\in$ $e$.registros $\land$ $y$ $\in$ $e$.registros $\land$ ($x$ $\nigobs$ $y$)$\big)$ \\
    \noindent\tab\tab obtener($c$, $y$) $\nigobs$ obtener($c$, $x$)
    \\
    \noindent\tab\tab\textbf{9)} $\big(\forall$ $r$ : registro, $r$ $\in$ $e$.registros$\big)$ $\big(\forall$ $c$ : campo$\big)$ $\big(c$ $\in$ claves($e$.campos)$\big)$ $\ssi$ \\
    \noindent\tab\tab $\big(c$ $\in$ claves($r$) $\yluego$ obtener($c$, $e$.campos) $\igobs$ tipo?(obtener($c$,$r$))$\big)$
    \\
    \noindent\tab\tab\textbf{10)} $\neg\big($vacia?($e$.campoIndexadoString)$\big)$ $\implies$ $\big($prim($e$.campoIndexadoString)$\big)$.nombre $\in$ claves($e$.campos)\\
    \noindent\tab\tab\textbf{10 bis)} $\neg\big($vacia?($e$.campoIndexadoNat)$\big)$ $\implies$ $\big($prim($e$.campoIndexadoNat)$\big)$.nombre $\in$ claves($e$.campos)\\
    \noindent\tab\tab\textbf{11)} $e$.cantAccesos $\ge$ \#$e$.registros\\
    \noindent\tab\tab\textbf{12)} long($e$.campoIndexadoNat) $\le$ 1 $\land$ long($e$.campoIndexadoNat) $\le$ 1\\
    \noindent\tab\tab\textbf{13)} $\neg\big($vacia?($e$.campoIndexadoNat)$\big)$ $\implies$\\
    \noindent\tab\tab$\big($prim($e$.campoIndexadoNat)$\big)$.vacio? $\ssi$ $\big(\#$claves($e$.indicesNat) $\igobs$ 0$\big)$ $\land$ \\
    \noindent\tab\tab $\neg\big($vacia?($e$.campoIndexadoString)$\big)$ $\implies$\\
    \noindent\tab\tab $\big($prim($e$.campoIndexadoString)$\big)$.vacio? $\ssi$ $\big(\#$claves($e$.indicesString) $\igobs$ 0$\big)$\\
    \noindent\tab\tab\textbf{14)} $e$.campoIndexadoNat $\igobs$ <> $\implies$ claves($e$.indicesNat) $\igobs$ $\emptyset$ $\land$ \\
    \noindent\tab\tab $e$.campoIndexadoString $\igobs$ <> $\implies$ claves($e$.indicesString) $\igobs$ $\emptyset$ \\
    \noindent\tab\tab\textbf{15)} ($\forall $ $r_1$, $r_2$ : registro, \{$r_1$, $r_2$\} $\subset$ $e$.registros) \\
    \noindent\tab\tab $\neg$ ($\exists$ $c$ : campo, $c$ $\in$ $e$.claves)\ $\big($obtener($c$, $r_1$) $\igobs$ obtener($c$, $r_2$)$\big)$
    \\

\medskip

  \noindent\textbf{Auxiliares sint�cticos}
  \tadAuxiliar{campoIndexString}{$\Pi_1\big($prim($e$.campoIndexString)$\big)$}
  \tadAuxiliar{campoIndexNat}{$\Pi_1\big($prim($e$.campoIndexadoNat)$\big)$}

  ~

  ~

  \Abs[estrTabla]{tabla}[e]{t}{nombre($t$) $\igobs$ $e$.nombre $\land$ claves($t$) $\igobs$ $e$.claves $\land$ campos($t$) $\igobs$ claves($e$.campos) $\yluego$ $\big(\forall$ $c$ : campo, $c$ $\in$ campos($t$)$\big)$ tipoCampo($c$, $t$) $\igobs$ obtener($c$, $e$.campos) $\land$ registros($t$) $\igobs$ $e$.registros $\land$ cantidadDeAccesos($t$) $\igobs$ $e$.cantAccesos $\land$ $\big(\forall$ $i$ : campo, $i$ $\in$ indices($t$)$\big)$ $\Big($tipoCampo($i$, $t$) $\implies$ $\big(i$ $\igobs$ $\big($prim($e$.campoIndexadoNat)$\big)$.nombre$\big)$ $\land$ \\
  $\neg\big($tipoCampo($i$, $t$)$\big)$ $\implies$ $\big(i$ $\igobs$ $\big($prim($e$.campoIndexadoString)$\big)$.nombre$\big)\Big)$}

  ~

\end{Representacion}

\bigskip

\begin{Algoritmos}

\medskip

 \Titulo{Algoritmos de tabla}
  	\medskip

\begin{algorithm}[H]{\textbf{iNombre}(\In{e}{estrTabla}) $\to$ $res$ : string}
    	\begin{algorithmic}
        \State $res \gets e.nombre$         \Comment $O(1)$

        \medskip
        \Statex \underline{Complejidad:} {$O(1)$}
        \Statex \underline{Justificaci�n:} {El algoritmo tiene una �nica llamada a una funci�n con costo $O(1)$.}
      \end{algorithmic}
\end{algorithm}

\begin{algorithm}[H]{\textbf{iClaves}(\In{e}{estrTabla}) $\to$ $res$ : itConj(campo)}
      \begin{algorithmic}
        \State $res \gets CrearIt(e.claves)$         \Comment $O(1)$

        \medskip
        \Statex \underline{Complejidad:} {$O(1)$}
        \Statex \underline{Justificaci�n:} {El algoritmo tiene una �nica llamada a una funci�n con costo $O(1)$.}
      \end{algorithmic}
\end{algorithm}

\begin{algorithm}[H]{\textbf{iIndices}(\In{e}{estrTabla}) $\to$ $res$ : conj(campo)}
      \begin{algorithmic}
        \State $res \gets Vacio()$                                  \Comment $O(1)$
        \If{$Longitud(e.campoIndexadoNat) > 0$}                     \Comment $O(1)$
          \State $AgregarRapido\big(Primero((e.campoIndexadoNat).nombre), aux\big)$\\ \Comment $O\big(copy((e.campoIndexadoNat).nombre)\big)$
        \EndIf
        \If{$Longitud(e.campoIndexadoString) > 0$}                     \Comment $O(1)$
          \State $AgregarRapido\big(Primero((e.campoIndexadoString).nombre), aux\big)$\\ \Comment $O\big(copy((e.campoIndexadoString).nombre)\big)$
        \EndIf

        \medskip
        \Statex \underline{Complejidad:} {$O(1)$}
        \Statex \underline{Justificaci�n:} {El algoritmo utiliza la funci�n agregarRapido del M�dulo Conjunto lineal 2 veces, entonces la complejidad es la de copiar el string del campo m�s largo de los dos. Como los strings de los campos estan acotados por una constante, entonces la complejidad queda $O(1)$.}
      \end{algorithmic}
\end{algorithm}

\begin{algorithm}[H]{\textbf{iCampos}(\In{e}{estrTabla}) $\to$ $res$ : itConj(campo)}
      \begin{algorithmic}
        \State $res \gets CrearIt(e.campos)$         \Comment $O\big(\#claves(e.campos) * L\big)$

        \medskip
        \Statex \underline{Complejidad:} {$O\big(\#claves(e.campos)\big)$}
        \Statex \underline{Justificaci�n:} {Por m�dulo diccString, la operaci�n Claves exporta complejidad $O\big(\#claves(e.campos) * L\big)$ siendo $L$ la longitud del mayor string en claves. Dado que los nombres de los campos est�n acotados, la complejidad final es $O\big(\#claves(e.campos)\big)$.
}
      \end{algorithmic}
\end{algorithm}

\begin{algorithm}[H]{\textbf{iTipoCampo}(\In{c}{campo}, \In{e}{estrTabla}) $\to$ $res$ : bool}
      \begin{algorithmic}
      \State $res \gets Obtener(c, e.campos)$         \Comment $O(1)$
      \medskip
      \Statex \underline{Complejidad:} {$O(1)$}
      \Statex \underline{Justificaci�n:} {Como la longitud de los campos es acotada, buscar en un diccString pasa de ser orden de longitud de la clave m�s larga a $O(1)$.}
      \end{algorithmic}
\end{algorithm}

\begin{algorithm}[H]{\textbf{iRegistros}(\In{e}{estrTabla}) $\to$ $res$ : conj(registros)}
      \begin{algorithmic}
      \State $res \gets (e.registros)$         \Comment $O(1)$
      \medskip
      \Statex \underline{Complejidad:} {$O(1)$}
      \Statex \underline{Justificaci�n:} {Devolver el conjunto por referencia es $O(1)$.}
      \end{algorithmic}
\end{algorithm}

\begin{algorithm}[H]{\textbf{iCantidadDeAccesos}(\In{e}{estrTabla}) $\to$ $res$ : nat}
      \begin{algorithmic}
      \State $res \gets e.cantAccesos$         \Comment $O(1)$
      \medskip
      \Statex \underline{Complejidad:} {$O(1)$}
      \Statex \underline{Justificaci�n:} {El algoritmo tiene una �nica llamada a una funci�n con costo $O(1)$.}
      \end{algorithmic}
\end{algorithm}

\begin{algorithm}[H]{\textbf{iBorrarRegistro}(\In{criterio}{registro}, \Inout{e}{estrTabla})}
      \begin{algorithmic}
      \State $it \gets CrearIt\big(VistaDicc(criterio)\big)$    \Comment $O(1)$
      \State $clave \gets Siguiente(it).clave$                  \Comment $O(1)$
      \State $dato \gets Siguiente(it).significado$             \Comment $O(1)$
      \State $ $
      \comentario{Si el criterio es un �ndice tenemos que recorrer el conjunto de iteradores a registros con un iterador borrando todos, si no hay que recorrer registros linealmente}

      \If{$\big(Primero(e.campoIndexadoNat)\big).nombre = clave$} \Comment $O(|L|)$
        \If{$Def?(ValorNat(dato), e.indicesNat)$}                 \Comment $O(log\ n) \ promedio$
          \State $iterador \gets CrearIt\big(Obtener(ValorNat(dato), e.indicesNat)\big)$                           \Comment $O(log\ n) \ promedio$
          \comentario{es clave, por lo tanto es el �nico en el conjunto, lo borro de e.registros:}
          \State $EliminarSiguiente\big(Siguiente(iterador)\big)$ \Comment $O(1)$
          \State $e.cantAccesos++$                                \Comment $O(1)$
          \State $Borrar\big(ValorNat(dato), e.indicesNat\big)$   \Comment $O(log\ n) \ promedio$
          \State $temp \gets CrearIt(e.indicesNat)$ \Comment $O(1)$
          \If{$\NOT HaySiguiente(temp)$} \Comment $O(1)$
            \State $Primero(e.campoIndexadoNat).vacio? \gets true$ \Comment $O(1)$
          \Else
            \comentario{comparamos por valorNat porque es O(1) vs comparar por dato}
            \If{$ValorNat(dato) = ValorNat\big(Primero(e.campoIndexadoNat).max\big)$} \Comment $O(1)$
                \State $Primero(e.campoIndexadoNat).max \gets DatoNat\big(\Pi_1\big(Max(e.indicesNat)\big)\big)$ \Comment $O(log\ n)$
            \EndIf
            \If{$ValorNat(dato) = ValorNat\big(Primero(e.campoIndexadoNat).min\big)$} \Comment $O(1)$
                \State $Primero(e.campoIndexadoNat).min \gets DatoNat\big(\Pi_1\big(Min(e.indicesNat)\big)\big)$ \Comment $O(log\ n)$
            \EndIf
          \EndIf

        \EndIf
      \ElsIf{$\big(Primero(e.campoIndexadoString)\big).nombre = clave$}  \Comment $O(|L|)$
        \If{$Def?\big(ValorString(dato), e.indicesString\big)$}   \Comment $O(|L|)$
          \State $iterador \gets CrearIt\big(Obtener(ValorStr(dato), e.indicesString)\big)$ \Comment $O(|L|)$
          \State $EliminarSiguiente\big(Siguiente(iterador)\big)$ \Comment $O(1)$
          \State $e.cantAccesos++$                                \Comment $O(1)$
          \State $Borrar\big(ValorStr(dato), e.indicesString\big)$ \Comment $O(|L|)$
          \State $temp \gets CrearIt(e.indicesString)$ \Comment $O(1)$
          \If{$\NOT HaySiguiente(temp)$} \Comment $O(1)$
            \State $Primero(e.campoIndexadoString).vacio? \gets true$ \Comment $O(1)$
          \Else
            \If{$dato = Primero(e.campoIndexadoString).max$} \Comment $O(L)$
                \State $Primero(e.campoIndexadoString).max \gets DatoString\big(\Pi_1\big(Max(e.indicesString)\big)\big)$\\ \Comment $O(L) \ por\ ref$
            \EndIf
            \If{$dato = Primero(e.campoIndexadoString).min$} \Comment $O(L)$
                \State $Primero(e.campoIndexadoString).min \gets DatoString\big(\Pi_1\big(Min(e.indicesString)\big)\big)$ \Comment $O(L)$
            \EndIf
          \EndIf



        \EndIf
      \Else

\algstore{myalg}
  \end{algorithmic}
\end{algorithm}

\begin{algorithm}
  \begin{algorithmic}
      \algrestore{myalg}


        \State $iter \gets CrearIt(e.registros)$                  \Comment $O(1)$
        \While{$HaySiguiente(iter)$}                              \Comment $O(n * ...)$
          \If{$Obtener\big(clave, Siguiente(iter)\big) = dato$}   \Comment $O(|L|)$
          \comentario{Borro de los �ndices, si hay}
            \If{$\NOT Vacia?(e.campoIndexadoNat)$}    \Comment $O(1)$
              \State $regi \gets Siguiente(iter) $ \Comment $O(1)$
              \State $valorIndex \gets ValorNat\big(Obtener(Primero(e.campoIndexadoNat)).nombre, regi\big) $ \Comment $O(1)$
              \State $conjIters \gets Obtener(valorIndex, e.indicesNat)$                          \Comment $O(log\ n)$
              \State $itDeIters \gets CrearIt(conjIters)$                                         \Comment $O(1)$
              \While{$HaySiguiente(itDeIters) $}                                                   \Comment $O(n * ...)$
                \comentario{Si apunto al registro que quiero borrar, actualizo el �ndice}
                \If{$Siguiente\big(Siguiente(itDeIters)\big) = regi$}                                     \Comment $O(1)$
                  \State $EliminarSiguiente(itDeIters)$                                           \Comment $O(1)$
                \EndIf
              \EndWhile

            \ElsIf{$\NOT Vacia?(e.campoIndexadoString)$}                                  \Comment $O(1)$
              \State $regi \gets Siguiente(iter)$                       \Comment $O(1)$
              \State $valorIndex \gets ValorStr\big(Obtener(Primero(e.campoIndexadoString)\big).nombre, regi)$  \Comment $O(L)$
              \State $conjIters \gets Obtener(valorIndex, e.indicesString)$                             \Comment $O(L)$
              \State $itDeIters \gets CrearIt(conjIters)$                                           \Comment $O(1)$
              \While{$HaySiguiente(itDeIters)$}                                         \Comment $O(n * ...)$
                \comentario{Si apunto al registro que quiero borrar, actualizo el �ndice}
                \If{$Siguiente\big(Siguiente(itDeIters)\big) = regi$}                     \Comment $O(1)$
                  \State $EliminarSiguiente(itDeIters)$                           \Comment $O(1)$
                \EndIf
              \EndWhile
            \EndIf
            \comentario{Ahora s� lo borro del conj}
            \State $EliminarSiguiente(iter)$                      \Comment $O(1)$
          \EndIf
          \State $Avanzar(iter)$                                  \Comment $O(1)$
        \EndWhile
      \EndIf



      \medskip
      \Statex \underline{Complejidad:} {\\
\quad\quad Criterio sobre campo indexado $\implies$ $O(log\ n + L)$. \\
\quad\quad Criterio sobre campo no indexado $\implies$ $O(n*|L|)$. \\
\textcolor{red}{LAN: COMPLETAR CON LA COMPLEJIDAD/JUSTIFIACION POSTA!!!!!!!!!!!!!!!}. \\
\quad\quad Siendo $n$ la cantidad total de registros de la tabla y $L$ el valor string m�s largo de todos los datos comparados.}

      \Statex \underline{Justificaci�n:} {\\
\quad\quad En peor caso sobre campo indexado recorre el diccNat o el diccString para encontrar el iterador (el conjunto significado tiene longitud 1 por ser un campo clave) al conjunto y eliminarlo. Buscar en diccNat en promedio es $O(log\ n)$ dado que se inserta con probabilidad uniforme. Buscar en diccString es $O(L)$ en el peor caso. \\
\quad\quad Se agrega el costo de actualizar el m�ximo, que es $O(log\ n)$ para �ndice nat y $O(L\ +\ L)$ = $O(L)$ (por comparar con m�ximo y m�nimo actual y luego buscar m�ximo y m�nimo respectivamente) para �ndice string, la inserci�n es por referencia a�n as�. Por lo tanto el m�ximo y m�nimo se actualiza en $O(log\ n\ +\ L)$. \\
\quad\quad En el peor caso sobre campo no indexado debe recorrer todo el conjunto de registros de la tabla preguntando si el dato de cada registro coincide con el criterio para eliminarlo.}
      \end{algorithmic}
\end{algorithm}

\begin{algorithm}[H]{\textbf{iNuevaTabla}(\In{nombre}{string}, \In{claves}{conj(campo)}, \In{columnas}{registro}) $\to$ $res$ : estrTabla}
  \begin{algorithmic}

    \State $res.indicesString \gets Vacio()$            \Comment $O(1)$
    \State $res.indicesNat \gets Vacio()$               \Comment $O(1)$
    \State $res.registros \gets Vacio()$                \Comment $O(1)$
    \State $res.nombre \gets Copiar(nombre)$            \Comment $O(|nombre|) = O(1)$
    \State $res.campos \gets Vacio()$                   \Comment $O(1))$
    \State $iter \gets VistaDicc(columnas)  $           \Comment $O(1)$
    \While{$HaySiguiente(iter)$}                        \Comment $O\big(\#campos(columnas) * ...\big) = O(1 * ...)$
      \State $Definir\big(Siguiente(iter).clave, Nat?(Siguiente(iter).significado, res.campos)\big)$
      \State $ $                                        \Comment $O\big(|max\ nombre\ de\ campo|\big) = O(1)$
      \State $Avanzar(iter) $                          \Comment $O(1)$
    \EndWhile
    \State $res.claves \gets Copiar(claves)$            \Comment $O(\#claves * L)$
    \State $res.campoIndexadoNat \gets Vacio()$         \Comment $O(1)$
    \State $res.campoIndexadoString \gets Vacio()$      \Comment $O(1)$
    \State $res.cantAccesos \gets 0$                    \Comment $O(1)$

    \medskip
    \Statex \underline{Complejidad:} {$O(1)$}
    \Statex \underline{Justificaci�n:} {\\
\quad\quad Todas las asignaciones que no usen copias son $O(1)$. Copiar el nombre es cte. porque, por enunciado los nombres de las tablas son acotados. \\
\quad\quad Copiar claves tiene complejidad $O(\#claves * L)$, donde $L$ es el nombre m�s largo de cualquier clave, que se reduce a $O(1)$ porque por enunciado los nombres de los campos tambi�n son acotados y tambi�n la cantidad de campos por tabla (es decir \#claves <\ $n$, para alg�n $n$ natural).\\
\quad\quad Vale lo mismo para definir las columnas, que pasa de ser $O(\#claves(columnas) * M)$ siendo $M$ el nombre de la clave m�s larga del diccionario (los significados de tipo bool para Nat? se consultan y copian en $O(1)$) a ser $O(1)$ por los factores ya mencionados.
  }
  \end{algorithmic}
\end{algorithm}


\begin{algorithm}[H]{\textbf{iAgregarRegistro}(\In{r}{registro}, \Inout{e}{estrTabla})}
  \begin{algorithmic}
    \comentario{Aumento la cantidad de accesos}
    \State $e.cantAccesos++$                                          \Comment $O(1)$


    \comentario{Agrego el registro al conjunto e.registros}
    \State $it \gets AgregarRapido(r, e.registros)$                   \Comment $O\big(copy(r)\big)$
    \comentario{Se paga una cantidad de veces acotada por copiar datos acotados en costo por L, que es el dato string m�s largo}
    \State $ $

    \comentario{Me fijo si tengo un campo nat indexado (me fijo en e.campoIndexadoNat)}
    \If{$\NOT Vacia?(e.campoIndexadoNat)$}                            \Comment $O(1)$
      \If{$\big(Primero(e.campoIndexadoNat)\big).vacio?$}                        \Comment $O(1)$
        \State $\big(Primero(e.campoIndexadoNat)\big).min \gets Obtener\big(\big(Primero(e.campoIndexadoNat)\big).nombre, r\big)$                       \Comment $O(1)$
        \State $\big(Primero(e.campoIndexadoNat)\big).max \gets \big(Primero(e.campoIndexadoNat)\big).min$                                   \Comment $O(1)$
        \State $\big(Primero(e.campoIndexadoNat)\big).vacio? \gets false$ \Comment $O(1)$
      \Else
        \State $nPaMinMax \gets Obtener\big(\big(Primero(e.campoIndexadoNat)\big).nombre, r\big)$                                                       \Comment $O(1)$
        \If{$nPaMinMax < \big(Primero(e.campoIndexadoNat)\big).min$}
          \State $\big(Primero(e.campoIndexadoNat)\big).min \gets nPaMinMax$
                                                                      \Comment $O(1)$
        \EndIf
        \If{$nPaMinMax > \big(Primero(e.campoIndexadoNat)\big).max$}
          \State $\big(Primero(e.campoIndexadoNat)\big).max \gets nPaMinMax$
                                                                      \Comment $O(1)$
        \EndIf
      \EndIf

      \State $aux \gets Obtener\big(\big(Primero(e.campoIndexadoNat)\big).nombre, r\big)$                                                                \Comment $O(1)$
      \comentario{Si lo tengo indexado y est� definido}
      \If{$Def?(aux, e.indicesNat)$}                                  \Comment $O(log\ n)$
        \State $AgregarRapido\big(it, Obtener(aux, e.indicesNat)\big)$ \Comment $O\big(copy(it)\big)$
      \Else
        \State $Definir\big(aux, AgregarRapido(it, Vacio()), e.indicesNat\big)$                                                              \Comment $O(log\ n)$
      \EndIf
    \EndIf
    \State $ $

    \If{$\NOT Vacia?(e.campoIndexadoString)$}                         \Comment $O(1)$
      \If{$\big(Primero(e.campoIndexadoString)\big).vacio?$} \Comment $O(1)$
        \State $\big(Primero(e.campoIndexadoString)\big).min \gets Obtener\big(\big(Primero(e.campoIndexadoString)\big).nombre, r\big)$\\ \Comment $O(1)$
        \State $\big(Primero(e.campoIndexadoString)\big).max \gets \big(Primero(e.campoIndexadoString)\big).min$                                \Comment $O(1)$
        \State $\big(Primero(e.campoIndexadoNat)\big).vacio? \gets false$ \Comment $O(1)$
      \Else
        \State $sPaMinMax \gets Obtener\big(\big(Primero(e.campoIndexadoString)\big).nombre, r\big)$                                               \Comment $O(1)$
        \If{$sPaMinMax < \big(Primero(e.campoIndexadoString)\big).min$}
                                                                      \Comment $O(L)$
          \State $\big(Primero(e.campoIndexadoString)\big).min \gets sPaMinMax$
                                                                      \Comment $O(1)$
        \EndIf
        \If{$sPaMinMax > \big(Primero(e.campoIndexadoString)\big).max$}                                                              \Comment $O(L)$
          \State $\big(Primero(e.campoIndexadoString)\big).max \gets sPaMinMax$
                                                                      \Comment $O(1)$
        \EndIf
      \EndIf

      \State $aux \gets Obtener\big(\big(Primero(e.campoIndexadoString)\big).nombre, r\big)$                                                              \Comment $O(1)$

      \If{$Def?(aux, e.indicesString)$}                               \Comment $O(Max\ string)$
        \State $AgregarRapido\big(it, Obtener(aux, e.indicesString)\big)$ \Comment $O\big(copy(it)\big)$
      \Else
        \State $Definir\big(aux, AgregarRapido(it, Vacio()), e.indicesString\big)$                                                              \Comment $O(Max\ string)$
      \EndIf
    \EndIf

    \medskip
    \Statex \underline{Complejidad:} {\\
\quad\quad Campo indexado: $O(|L| +log\ n)$\\
\quad\quad Campo no indexado: $O(|S|)$\\
\quad\quad Donde $n$ es la cantidad de claves del diccNat $e$.indicesNat (acotada por la cantidad de registros de la tabla) y $L$ es el string m�s largo de cualquier registro en la tabla. $S$ es el string m�s largo del registro a agregar.
}

    \algstore{myalg}
  \end{algorithmic}
\end{algorithm}

\clearpage

\begin{algorithm}
  \begin{algorithmic}
      \algrestore{myalg}

    \Statex \underline{Justificaci�n:} {\\
\quad\quad Campo indexado: Agregar el registro al conjunto cuesta $O\big(copy (r)\big)$; al ser un diccString eso ser�a $O\big(\#claves(r) * Max\{K,S\}\big)$, siendo $K$ la clave de m�ximo costo para copiar y $S$ lo mismo pero para significados. Como la cantidad de claves est� acotada por haber una cantidad acotada de campos (por enunciado) y la longitud de los nombres de campos tambi�n, vale que $O\big(\#claves(r) * K\big)$ = $O(1)$. Por lo tanto, el peor caso es pagar por el copiado del significado m�s costoso, que corresponde a la longitud m�xima de cualquier string del registro (que acotamos por la m�xima longitud de cualquier string de cualquier registro de la tabla, y lo denominamos |$L$|). \\
\quad\quad Adem�s se agrega el costo de agregar en el diccionario de �ndices nat (logar�tmico en la cantidad $n$ de registros de la tabla) y el de agregar en el diccionario de �ndices string $\big($nuevamente, longitud m�xima de cualquier string de cualquier registro de la tabla, es decir $O(|L|)\big)$. \\
\quad\quad Por lo tanto, la complejidad final queda $O\big(|L| + log\ n + |L|\big)$ = $O\big(|L| + log\ n\big)$.\\
\\
\quad\quad Campo no indexado: Agregar el registro al conjunto cuesta $O\big(copy(r)\big)$, esto es igual a copiar el string m�s largo ya que copiar los nat es $O(1)$. Luego el algoritmo si no hay campos indexados no hace mas operaciones que sean mayores a $O(1)$. Por lo tanto el algoritmo tiene complejidad $O(|S|)$, siendo $S$ el string m�s largo del registro a agregar.

}
  \medskip
  \end{algorithmic}
\end{algorithm}

\begin{algorithm}[H]{\textbf{iIndexar}(\In{c}{campo}, \Inout{e}{estrTabla})}
  \begin{algorithmic}
    \comentario{si es tipo nat...}
    \If{$TipoCampo(c, e)$} \Comment $O(1)$
      \State $dato \gets DatoNat(0)$ \Comment $O(1)$
      \comentario{Agrego adelante de la lista de campoIndexadoNat}
      \State $AgregarAdelante\big(e.campoIndexadoNat,  \big\langle c, dato, dato, true\big\rangle\big)$ \Comment $O\big(copy(\langle c, dato, dato, bool\rangle)\big) \in O(1)$
      \State $ $

      \State $it \gets CrearIt(e.registros)$ \Comment $O(1)$
      \comentario{Si hay algun registro entonces lo seteo como maximo y minimo y en el while pregunto}
      \If{$HaySiguiente(it)$} \Comment $O(1)$
        \State $\big(Primero(e.campoIndexadoNat)\big).vacio? \gets false$ \Comment $O(1)$
        \State $\big(Primero(e.campoIndexadoNat)\big).max? \gets Obtener(c, Siguiente(it))$ \Comment $O(1)$
        \State $\big(Primero(e.campoIndexadoNat)\big).min? \gets Obtener(c, Siguiente(it))$ \Comment $O(1)$
      \EndIf
      \While{$HaySiguiente(it)$} \Comment $O\big(\#registros(e) * ...\big)$
        \State $temp \gets ValorNat\big(Obtener(c, Siguiente(it))\big)$ \Comment $O(1)$
        \If{$\NOT Def?\big(Obtener(temp, e.indicesNat)\big)$} \Comment $O\big(log\ \#registros\big)$
            \State $Definir\big(temp, Vacio(), e.indicesNat\big)$ \Comment $O\big(log\ \#registros\big)$
        \EndIf
        \State $AgregarRapido\big(it, Obtener(temp, e.indicesNat)\big)$ \Comment $O\big(copy(it)\ +\ log\ \#registros\big)$
        \If{$Obtener\big(c, Siguiente(it)\big) > \big(Primero(e.campoIndexadoNat)\big).max$} \Comment $O(1)$
          \State $\big(Primero(e.campoIndexadoNat)\big).max \gets Obtener\big(c, Siguiente(it)\big)$ \Comment $O(1)$
        \EndIf
        \If{$Obtener\big(c, Siguiente(it)\big) < \big(Primero(e.campoIndexadoNat)\big).min$} \Comment $O(1)$
          \State $\big(Primero(e.campoIndexadoNat)\big).min \gets Obtener\big(c, Siguiente(it)\big)$ \Comment $O(1)$
        \EndIf
        \State $Avanzar(it)$ \Comment $O(1)$
      \EndWhile
    \Else

    \algstore{myalg}
  \end{algorithmic}
\end{algorithm}

\clearpage

\begin{algorithm}
  \begin{algorithmic}
      \algrestore{myalg}



      \State $dato \gets DatoStr("temp")$ \Comment $O(1)$
      \State $AgregarAdelante\big(e.campoIndexadoString,  \big\langle c, dato, dato, true\big\rangle\big)$ \Comment $O\big(copy(\langle c, dato, dato, bool\rangle)\big) \in O(1)$
      \State $ $

      \State $it \gets CrearIt(e.registros)$ \Comment $O(1)$
      \comentario{Si hay algun registro entonces lo seteo como maximo y minimo y en el while pregunto}
      \If{$HaySiguiente(it)$} \Comment $O(1)$
        \State $\big(Primero(e.campoIndexadoString)\big).vacio? \gets false$ \Comment $O(1)$
        \State $\big(Primero(e.campoIndexadoString)\big).max? \gets Obtener(c, Siguiente(it))$ \Comment $O(1)$
        \State $\big(Primero(e.campoIndexadoString)\big).min? \gets Obtener(c, Siguiente(it))$ \Comment $O(1)$
      \EndIf
      \While{$HaySiguiente(it)$} \Comment $O\big(\#registros(e) * ...\big)$
        \State $temp \gets ValorStr\big(Obtener(c, Siguiente(it))\big)$ \Comment $O(1)$
        \If{$\NOT Def?\big(Obtener(temp, e.indicesString)\big)$} \Comment $O\big(|L|\big)$
            \State $Definir\big(temp, Vacio(), e.indicesString\big)$ \Comment $O\big(|L|\big)$
        \EndIf
        \State $AgregarRapido\big(it, Obtener(temp, e.indicesString)\big)$ \Comment $O\big(copy(it)\ +\ log\ \#registros\big)$
        \If{$Obtener\big(c, Siguiente(it)\big) > \big(Primero(e.campoIndexadoString)\big).max$} \Comment $O(1)$
          \State $\big(Primero(e.campoIndexadoString)\big).max \gets Obtener\big(c, Siguiente(it)\big)$ \Comment $O(1)$
        \EndIf
        \If{$Obtener\big(c, Siguiente(it)\big) < \big(Primero(e.campoIndexadoString)\big).min$} \Comment $O(L)$
          \State $\big(Primero(e.campoIndexadoString)\big).min \gets Obtener\big(c, Siguiente(it)\big)$ \Comment $O(1)$
        \EndIf
        \State $Avanzar(it)$ \Comment $O(1)$
      \EndWhile
    \EndIf


    \medskip
    \Statex \underline{Complejidad:} {$O\big(|registros| * L * (L\ +\ log\ |registros(e)|)\big)$, donde $L$ es el m�ximo string para el campo $c$ en cualquier registro.}
    \Statex \underline{Justificaci�n:} {\\
\quad\quad En el peor caso se recorren todos los registros definiendo un iterador suyo $\big(O(1)\big)$ en un diccString $\big($inserci�n en $O(L)\big)$ o insertando en un diccNat $\big($en $O(log\ |registros|)$ para caso promedio$\big)$, por el costo de copiar cada valorStr si es m�ximo o m�nimo (acotado por $L$).
}
  \end{algorithmic}
\end{algorithm}



\begin{algorithm}[H]{\textbf{iBuscar}(\In{c}{campo}, \In{d}{dato}, \In{e}{estrTabla}) $\to$ $res$ : lista(registro)}
  \begin{algorithmic}
    \State $res \gets Vacia()$ \Comment $O(1)$
    \If $Nat?(d)$ \Comment $O(1)$
      \comentario{caso campoJOIN, donde esta indexado}
      \If{$\big(Primero(e.campoIndexadoNat)\big).nombre = c$} \comentariocompl{nombres acotados}{$O(|c|) = O(1)$}
        \If{$Def?\big(ValorNat(d), e.indicesNat\big)$} \comentariocompl{n cantidad de registros}{$O(log\ n)$}
          \State $itConjIts \gets CrearIt\big(Obtener(ValorNat(d), e.indicesNat)\big)$ \Comment $O(log\ n)$
          \While{$HaySiguiente?(itConjIts)$} \Comment $O(1 * ...)\ si\ c\ es\ clave\ / O(n * ...)\ si\ no$
            \State $AgregarAtras\big(Siguiente\big(Siguiente(itConjIts)\big), res\big)$ \comentariocompl{L mayor string de la tabla}{$O(\#campos * |L|) = O\big(|L|\big)$}
            \State $Avanzar(itConjIts)$ \Comment $O(1)$
          \EndWhile
        \EndIf
      \Else
        \State $it \gets CrearIt(e.registros)$ \Comment $O(1)$
        \While{$HaySiguiente?(it)$} \Comment $O(n * ...)$
          \If{$Obtener\big(c, Siguiente(it)\big) = d$} \Comment $O(|ValorStr(d)|) = O(|L|)$
            \State $AgregarAtras\big(Siguiente(it), res\big)$     \Comment $O(|L|)$
          \EndIf
          \State $Avanzar(it)$ \Comment $O(1)$
        \EndWhile
      \EndIf
    \Else
      \comentario{caso campoJOIN, donde esta indexado}
      \If{$\big(Primero(e.campoIndexadoString)\big).nombre = c$} \comentariocompl{nombres acotados}{$O(|c|) = O(1)$}
        \If{$Def?\big(ValorStr(d), e.indicesString\big)$} \Comment$O(|L|)$
          \State $itConjIts \gets CrearIt\big(Obtener(ValorStr(d), e.indicesString)\big)$ \Comment $O(L)$
          \While{$HaySiguiente?(itConjIts)$} \Comment $O(1 * ...)\ si\ c\ es\ clave\ / O(n * ...)\ si\ no$
            \State $AgregarAtras\big(Siguiente\big(Siguiente(itConjIts)\big), res\big)$ \Comment$O(|L|)$
            \State $Avanzar(itConjIts)$ \Comment $O(1)$
          \EndWhile
        \EndIf
      \Else
        \State $it \gets CrearIt(e.registros)$ \Comment $O(1)$
        \While{$HaySiguiente?(it)$} \Comment $O(n * ...)$
          \If{$Obtener\big(c, Siguiente(it)\big) = d$} \Comment $O(|ValorStr(d)|) = O(|L|)$
            \State $AgregarAtras\big(Siguiente(it), res\big)$
          \EndIf
          \State $Avanzar(it)$ \Comment $O(1)$
        \EndWhile
      \EndIf
    \EndIf


    \medskip
    \Statex \underline{Complejidad:} {\\
\quad\quad Campo indexado nat y clave $\implies$ $O(log\ n\ +\ |L|)$ promedio.\\
\quad\quad Campo indexado nat y no clave $\implies$ $O(log\ n\ +\ n * |L|)$ promedio.\\
\\
\quad\quad Campo indexado String y clave $\implies$ $O(|L|\ +\ |L|)$ = $O(|L|)$. \\
\quad\quad Campo indexado String y no clave $\implies$ $O(|L|+ n * |L|)$ = $O(n * |L|)$.\\\\
\quad\quad Campo NO indexado $\implies$ $O(n * |L|)$.
\\
\quad\quad Donde $n$ es la cantidad de registros y $L$ el string m�s largo de la tabla.

}
    \Statex \underline{Justificaci�n:} {En el peor caso se recorren todos los registros, con cada caso detallado anteriormente.}
  \end{algorithmic}
\end{algorithm}


\begin{algorithm}[H]{\textbf{iMinimo}(\In{c}{campo}, \In{e}{estrTabla}) $\to$ $res$ : dato}
  \begin{algorithmic}
    \If{$c = \big(Primero(e.campoIndexadoNat)\big).nombre$} \Comment $O(1)$
      \State $res \gets \big(Primero(e.campoIndexadoNat)\big).min$ \Comment $O(1)$
    \Else
      \State $res \gets \big(Primero(e.campoIndexadoStr)\big).min$ \Comment $O(1)$
    \EndIf

    \medskip
    \Statex \underline{Complejidad:} {$O(1)$}
    \Statex \underline{Justificaci�n:} {El resultado se devuelve por referencia.}
  \end{algorithmic}
\end{algorithm}

\begin{algorithm}[H]{\textbf{iMaximo}(\In{c}{campo}, \In{e}{estrTabla}) $\to$ $res$ : dato}
  \begin{algorithmic}
    \If{$c = \big(Primero(e.campoIndexadoNat)\big).nombre$} \Comment $O(1)$
      \State $res \gets \big(Primero(e.campoIndexadoNat)\big).max$ \Comment $O(1)$
    \Else
      \State $res \gets \big(Primero(e.campoIndexadoStr)\big).max$ \Comment $O(1)$
    \EndIf

    \medskip
    \Statex \underline{Complejidad:} {$O(1)$}
    \Statex \underline{Justificaci�n:} {El resultado se devuelve por referencia.}
  \end{algorithmic}
\end{algorithm}


\end{Algoritmos}

\newpage
\section{M�dulo Base de Datos}

\begin{Interfaz}
  
  \textbf{se explica con}: \tadNombre{Base de Datos}.

  \textbf{g�neros}: \TipoVariable{base}.

  \Titulo{Operaciones b�sicas de base}

  \InterfazFuncion{NuevaBDD}{}{base}
  [true]  
  {$res$ $\igobs$ nuevaBDD}
  [$O(1)$]
  [crea una base de datos sin tablas.]
  
  \InterfazFuncion{AgregarTabla}{\In{t}{tabla}, \Inout{b}{base}}{}
  [vacio?(registros($t$)) $\land$ $b$ = $b_0$]
  {$b$ $\igobs$ agregarTabla($t$, $b_0$)}
  [$O(1)$]
  [devuelve un iterador al conjunto de campos claves de la tabla indicada.]
  
  \InterfazFuncion{InsertarEntrada}{\In{r}{registro}, \In{t}{tabla}, \Inout{b}{base}}{}
  [$t$ $\in$ tablas($b$) $\yluego$ puedoInsertar?($r$, $t$) $\land$ $b$ = $b_0$]
  {$b$ $\igobs$ insertarEntrada($r$, $t$, $b_0$)}
  [$O\big(log\ n + |L| * \#tablas(b)\big)$, donde $L$ es el dato string m�s largo de $r$ y $n$ es la cantidad de registros en la tabla.]
  [inserta un registro en una tabla de la base de datos.]

  \InterfazFuncion{Borrar}{\In{cr}{registro}, \In{t}{string}, \Inout{b}{base}}{}
  [\#campos($cr$) = 1 $\land$ $t$ $\in$ tablas($b$) $\land$ $b$ = $b_0$ $\land$ dameUno$\big($campos($cr$)$\big)$ $\in$ claves$\big($dameTabla($t$, $b$)$\big)$]
  {$b$ $\igobs$ borrar($cr$, $t$, $b_0$)}
  [\\
\tab Campo indexado $\implies$ $O\big(log\ n + |L| * \#tablas(b)\big)$ \\
\tab Campo no indexado $\implies$  $O\big(|L| * (n + \#tablas(b))\big)$, donde $L$ es el dato string m�s largo de $cr$ y $n$ es la cantidad de registros en la tabla.]
  [borra todos los registros que coincidan con el campo del registro $cr$ en la tabla $t$.]
  

  \InterfazFuncion{CombinarRegistros}{\In{t_1}{String}, \In{t_2}{String}, \In{c}{Campo}}{conj(Registro)}
  [true]
  {$res$ $\igobs$ combinarRegistros$\big(c$, registros(dameTabla($t_1$)), registros(dameTabla($t_2$))$\big)$}
  [\\
  \tab Si $c$ est� indexado en alguna de las tablas y es tipo string $\implies$ $O\big(n * |L|)\big)$\\
  \tab Si $c$ est� indexado y es nat $\implies$ $O\big(n * (|L|\ +\ log\ m)\big)$\\
  \tab Si no est� indexado $\implies$ $O\big(n * m * |L|)\big)$]
  [uni�n de todos los registros (en caso de que ambas tablas tengan campos con mismo nombre no claves, desempata para $t_1$).]


  \InterfazFuncion{GenerarVistaJoin}{\In{t_1}{string}, \In{t_2}{string}, \In{c}{campo}, \Inout{b}{base}}{itBi(registro)}
  [$t_1$ $\nigobs$ $t_2$ $\land$ \{$t1$, $t2$\} $\incluido$ tablas($b$) $\yluego$ $c$ $\in$ claves$\big($dameTabla($t_1$, $b$)$\big)$ $\land$ $c$ $\in$ claves$\big($dameTabla($t_2$, $b$)$\big)$ $\land$ $\neg\big($hayJoin?($t_1$, $t_2$, $b$)$\big)$ $\land$ tipoCampo($c$, $t_1$) = tipoCampo($c$, $t_2$) $\land$ $b$ = $b_0$]
  {$b$ $\igobs$ generarVistaJoin($t_1$, $t_2$, $c$, $b_0$) $\land$ alias$\big($secuAConj(Siguientes($res$)) $\igobs$ vistaJoin($t_1$, $t_2$, $b$)$\big)$}
  [\\
  \tab $c$ tipo string indexado en $t_1$ � $t_2$ $\implies$ $O\big((n+m) * L\big)$ \\
  \tab $c$ tipo nat indexado en $t_1$ � $t_2$ $\implies$ $O\big((n+m) * (log(n+m) + L)\big)$ \\
  \tab $c$ cualquier tipo no indexado $\implies$ $O\big((n+m) * \big(L\ +\ log(n+m)\big)\ +\ L * n * m\big)$ \\
  \tab Donde $n$ = \#registros($t_1$), $m$ = \#registros($t_2$) y $L$ el dato string m�s largo de cualquiera de las dos tablas.]
  [crea un join entre dos tablas de la base de datos y devuelve un iterador no modificable a sus registros.]
  [res no es modificable, se itera s�lo a modo de vista del conjunto.]
  
  \InterfazFuncion{BorrarJoin}{\In{t_1}{string}, \In{t_2}{string}, \Inout{b}{base}}{}
  [hayJoin?($t_1$, $t_2$, $b$) $\land$ $b$ = $b_0$]
  {$b$ $\igobs$ borrarJoin($t_1$, $t_2$, $b_0$)}
  [$O(1)$]
  [elimina el join entre dos tablas.]
  
  \InterfazFuncion{Tablas}{\In{b}{base}}{itConj(string)}
  [true]
  {$res$ $\igobs$ crearIt$\big($tablas($b$)$\big)$}
  [$O(1)$]
  [se obtienen todas las tablas de la base de datos.]
  
  \InterfazFuncion{DameTabla}{\In{s}{string}, \In{b}{base}}{tabla}
  [$s$ $\in$ tablas($b$)]
  {$res$ $\igobs$ dameTabla($s$)}
  [$O(1)$]
  [dado un nombre, devuelve la tabla con ese nombre en la base de datos.]

  \InterfazFuncion{HayJoin?}{\In{t_1}{string}, \In{t_2}{string}, \In{b}{base}}{bool}
  [true]
  {$res$ $\igobs$ hayJoin?($t_1$, $t_2$, $b$)}
  [$O(1)$]
  [devuelve \texttt{true} si hay un join entre los dos nombres de las tablas dados.]
  
  \InterfazFuncion{CampoJoin}{\In{t_1}{string}, \In{t_2}{string}, \In{b}{base}}{campo}
  [hayJoin?($t_1$, $t_2$, $b$)]
  {$res$ $\igobs$ campoJoin($t_1$, $t_2$, $b$)}
  [$O(1)$]
  [devuelve el campo que une al join entre las dos tablas.]
  
  \InterfazFuncion{VistaJoin}{\In{t_1}{string}, \In{t_2}{string}, \In{b}{base}}{itBi(registro)}
  [hayJoin?($t_1$, $t_2$, $b$)]
  {alias(secuAConj(Siguientes($res$)) $\igobs$ vistaJoin($t_1$, $t_2$, $b$))}
  [\\
\tab Join campo nat indexado $\implies$ $O\big(R*\big(|L| + log(n*m)\big)\big)$\\
\tab Join campo nat no indexado $\implies$ $O\big(R*\big(|L|*(m+n) + log(n+m)\big)\big)$\\
\tab Join campo string indexado $\implies$ $O(R*|L|)$\\
\tab Join campo string no indexado $\implies$ $O\big(R*|L|*(n+m)\big)$\\
\\
\tab $L$, cota para toda longitud de dato string en las dos tablas\\
\tab $n$ y $m$, cantidad de registros de las tablas con nombre s1 y s2 respectivamente\\
\tab $R$, cantidad de registros a 'actualizar' (uni�n de las listas de cambios de ambas tablas)]
  [devuelve un iterador a los conjuntos del join (ya definido) entre las dos tablas.]
  [res no es modificable, se itera s�lo a modo de vista del conjunto.]


  \InterfazFuncion{BusquedaCriterio}{\In{criterio}{registro}, \In{t}{String},\In{b}{base}}{conj(registro)}
  [$t$ $\in$ tablas($b$) $\land$ $\big($campos($criterio$) $\subset$ campos(dameTabla($t$, $b$))$\big)$]  
  {$res$ $\igobs$ buscar($r$, $t$, $b$)}
  [depende de la disposici�n de �ndices en $t$ y si son claves o no (consultar complejidad y justificaci�n debajo del algoritmo).]
  [devuelve por copia una lista de todos los registros de $t$ que coinciden en todos los campos con el registro $criterio$.]


  \InterfazFuncion{CoincidenTodosCrit}{\In{crit}{registro}, \In{r}{registro}}{bool}
  [campos($crit$) $\subset$ campos($r$)]
  {$res$ $\igobs$ coincidenTodos$\big(crit$, campos($crit$), $r\big)$}
  [$O\big(|L|\big)$]
  [determina si el registro comparte los mismos valores para los campos de crit.]


  \InterfazFuncion{TablaMaxima}{\In{b}{base}}{string}
  [\#tablas($b$) > 0]
  {alias $\big(res$ $\igobs$ tablaMaxima($b$)$\big)$}
  [$O(1)$]
  [devuelve el nombre de la tabla m�s accedida.]
  [res no es modificable]


\end{Interfaz}

\begin{Representacion}
  
  \Titulo{Representaci�n de base}

  \begin{Estructura}{base}[estr]

      \dondees{estr}{tupla}$\Big($\emph{tablaMasAccedida}: \TipoVariable{puntero(string)},
        \emph{nombreATabla}: \TipoVariable{diccString(tabla)}, \veryquad \\
        \emph{tablas}: \TipoVariable{conj(string)}, 
        \emph{joinPorCampoNat}: \TipoVariable{diccString\big(diccString\big(diccNat(itConj(registro))\big)\big)}, \\
        \emph{joinPorCampoString}: \TipoVariable{diccString\big(diccString\big(diccString(itConj(registro))\big)\big)}, \\
        \emph{registrosDelJoin}: \TipoVariable{diccString\big(diccString\big(conj(registro)\big)\big)}, \\
        \emph{hayJoin}: \TipoVariable{diccString}\big(\TipoVariable{diccString}\big(\TipoVariable{tupla<} \emph{campoJoin}: \TipoVariable{campo}, \emph{cambiosT1}: \TipoVariable{lista}(\TipoVariable{tupla<}\emph{reg}: \TipoVariable{registro}, \emph{agregar?}: \TipoVariable{bool}\TipoVariable{>}), \emph{cambiosT2}: \TipoVariable{lista}(\TipoVariable{tupla<}\emph{reg}: \TipoVariable{registro}, \emph{agregar?}: \TipoVariable{bool}\TipoVariable{>})\TipoVariable{>}\big)\big)$\Big)$ \veryquad


     \dondees{registro}{diccString(dato)}y se explica con \tadNombre{Registro}.
  \end{Estructura}

  \noindent\textbf{Invariante de representaci�n}

  \noindent\tab 1) Las claves de nombreATabla est�n en tablas, sus significados tienen su nombre y son todas las tablas de la lista $e$.tablas. \\
\noindent\tab 2) La tabla m�s accedida est� entre las tablas y tiene m�s accesos que todas las dem�s. \\
\noindent\tab 3) Las claves de JoinPorCampo, hayJoin y registrosDelJoin son las tablas de $e$.tablas (por (1), lo mismo que claves de $e$.nombreATabla) y las tablas con las que tienen joins a su vez tambi�n son de tablas de la base. \\
\noindent\tab 4) No hay tablas con joins con ellas mismas. \\
\noindent\tab 5) Los significados de una clave en las estructuras relacionadas a los joins son las mismas para cada estructura (son aquellas con las que comparten un join). \\
\noindent\tab 6) Los registros del join se crean por combinaci�n de sus tablas \\
\noindent\tab 7) Entre dos tablas solamente puede haber un �nico join (en una direcci�n). \\
\noindent\tab 8) El campo del join tambi�n es rec�proco y es clave para los dos. \\
\noindent\tab 9) El campo del join en hayJoin es el que lo define en el diccionario seg�n su tipo (que es el mismo tipo para ambas tablas). \\
\noindent\tab 10) Los significados de cada diccionario de joins tienen siguiente perteneciente a registros del join para las mismas claves. \\
\noindent\tab 11) Para cada registro en registros del join hay un iterador en alguno de los dos diccionarios (nat o string) con siguiente en �l. \\
\noindent\tab 12) Los registros en la lista de cambios tienen por campos a los campos de la primer clave y, para cada �ltima aparaci�n de un registro en la lista, el bool agregar refleja si pertenece el registro o no a los registros de las primer clave (para ambas listas). \\


  \Rep[estr][e]{}


  \noindent\tab\tab\bigpar {\textbf{1)} 
    $\big(\forall$ $s$ : string$\big)$ $\Big(s$ $\in$ claves($e$.nombreATabla) $\yluego$ 
    nombre$\big($obtener($s$, $e$.nombreATabla)$\big)$ $\igobs$ $s\Big)$ $\ssi$ \\
    $s$ $\in$ $e$.tablas)} $\land$ \\

    \noindent\tab\tab\bigpar{\textbf{2)} $e$.tablaMasAccedida $\in$ claves($e$.nombreATabla) $\yluego$ $\big(\forall$ $n$ : string, $n$ $\in$ claves($e$.nombreATabla)$\big)$\\
    cantAccesos$\big($obtener($n$, $e$.nombreATabla)$\big)$ $\le$ cantAccesos$\big($obtener(*$e$.tablaMasAccedida, $e$.nombreATabla)$\big)$}$\land$\\

    \noindent\tab\tab\bigpar{\textbf{3)} claves($e$.joinPorCampoNat) $\igobs$ claves($e$.hayJoin) $\land$\\
    claves($e$.joinPorCampoString) $\igobs$ claves($e$.hayJoin) $\land$\\
    claves($e$.registrosDelJoin) $\igobs$ claves($e$.hayJoin) $\land$\\
    claves($e$.hayJoin) $\igobs$ claves($e$.nombreATabla)} $\land$\\

    \noindent\tab\tab$\Big($\textbf{4)} $\neg\big(\exists$ $s$ : string, $s$ $\in$ claves($e$.hayJoin)$\big)$ $\big(s$ $\in$ claves$\big($obtener($s$, $e$.hayJoin)$\big)\big)\Big)$ $\land$\\

    \noindent\tab\tab\bigpar{\textbf{5)} $\big(\forall$ $n$ : string, $n$ $\in$ claves($e$.hayJoin)$\big)$\\
    \bigpar{claves$\big($obtener($n$, $e$.JoinPorCampoNat)$\big)$ $\cup$\\
    claves$\big($obtener($n$, $e$.JoinPorCampoString)$\big)$} $\igobs$ claves$\big($obtener($n$, $e$.hayJoin)$\big)$ $\land$\\
    claves$\big($obtener($n$, $e$.registrosDelJoin)$\big)$ $\igobs$ claves$\big($obtener($n$, $e$.hayJoin)$\big)$ $\land$\\
    claves$\big($obtener($n$, $e$.hayJoin)$\big)$ $\subset $ $\big($claves($e$.nombreATabla)$\big)$} $\land$\\

    \noindent\tab\tab\bigpar{\textbf{6)}
      $\big(\forall$ $s_1$ : string, $s_1$ $\in$ claves($e$.registrosDelJoin)$\big)$ \\
      $\big(\forall$ $s_2$ : string, $s_2$ $\in$ Obtener($s_1$, $e$.registrosDelJoin)$\big)$ \\
      \bigpar{$\big(\exists$ $r_m$, $r_1$, $r_2$ : registro, $r_m$ $\in$ Obtener($s_2$, Obtener($s_1$, $e$. registrosDelJoin)) $\land$ \\   
        $r_1$ $\in$ Registros(Obtener($s_1$, $e$.nombreATabla)) $\land$ \\
        $r_2$ $\in$ Registros(Obtener($s_2$, $e$.nombreATabla)) $\big)$
      } \\
      \ $r_m$ $\igobs$ AgregarCampos($r_1$, $r_2$)
    } $\land$ \\

    \noindent\tab\tab\bigpar{\textbf{7)} $\big(\forall$ $s_1$ : string, $s_1$ $\in$ claves($e$.hayJoin)$\big)$\\
    $\big(\forall$ $s_2$ : string, $s_2$ $\in$ claves$\big($obtener($s_1$, $e$.hayJoin)$\big)\big)$ \\
    $\Big($def?$\big(s_2$, obtener($s_1$, $e$.joinPorCampoNat)$\big)$ $\implies$ $\neg$def?$\big(s_2$, obtener($s_1$, $e$.joinPorCampoString)$\big)\Big)$ $\land$ \\
    $\Big($def?$\big(s_2$, obtener($s_1$, $e$.joinPorCampoString)$\big)$ $\implies$ $\neg$def?$\big(s_2$, obtener($s_1$, $e$.joinPorCampoNat)$\big)\Big)$} $\land$\\
    
    \noindent\tab\tab\bigpar{\textbf{8)} $\big(\forall$ $s_1$ : string, $s_1$ $\in$ claves($e$.hayJoin)$\big)$\\
    $\big(\forall$ $s_2$ : string, $s_2$ $\in$ claves$\big($obtener($s_1$, $e$.hayJoin)$\big)\big)$ \\
    $\Big($obtener$\big(s_2$, obtener($s_1$, $e$.hayJoin)$\big)$.campo $\in$ claves$\big($obtener($s_1$, e.nombreATabla)$\big)\Big)$ $\land$ \\
    $\Big($obtener$\big(s_2$, obtener($s_1$, $e$.hayJoin)$\big)$.campo $\in$ claves$\big($obtener($s_2$, $e$.nombreATabla)$\big)\Big)$} $\land$ \\
    
    \noindent\tab\tab\bigpar{\textbf{9)} $\big(\forall$ $s_1$ : string, $s_1$ $\in$ claves($e$.hayJoin)$\big)$\\
    $\big(\forall$ $s_2$ : string, $s_2$ $\in$ claves$\big($obtener($s_1$, $e$.hayJoin)$\big)\big)$ \\
    \bigpar{tipoCampo$\Big($obtener$\big(s_2$, obtener($s_1$, $e$.hayJoin)$\big)$.campo, obtener$\big(e$.nombreATabla($s_1$)$\big)\Big)$ $\igobs$ \\
    tipoCampo$\Big($obtener$\big(s_2$, obtener($s_1$, $e$.hayJoin)$\big)$.campo, obtener$\big($e.nombreATabla($s_2$)$\big)\Big)$ $\yluego$\\
      \bigpar{
        \bigpar{$\Big($tipoCampo$\Big($obtener$\big(s_2$, obtener($s_1$, $e$.hayJoin)$\big)$.campo, obtener$\big($e.nombreATabla($s_2$)$\big)\Big)\Big)$ $\implies$ \\
        def?$\big(s_2$, obtener($s_1$, $e$.joinPorCampoNat)$\big)$} $\land$ \\
        \bigpar{$\Big(\neg$tipoCampo$\Big($obtener$\big(s_2$, obtener($s_1$, $e$.hayJoin)$\big)$.campo, obtener$\big($e.nombreATabla($s_2$)$\big)\Big)\Big)$ $\implies$ \\
        def?$\big(s_2$, obtener($s_1$, $e$.joinPorCampoString)$\big)$}
      }
    }}$\land$ \\
    
    \noindent\tab\tab
      \bigpar{
        \textbf{10)} $\big(\forall$ $s_1$ : string, $s_1$ $\in$ claves($e$.joinPorCampoNat)$\big)$\\
        \bigpar{
          $\big(\forall$ $s_2$ : string, $s_2$ $\in$ claves$\big($obtener($s_1$, $e$.joinPorCampoNat)$\big)\big)$\\
          \bigpar{
            $\Big(\forall$ $n$ : nat, def?$\big(n$, obtener$\big(s_2$, obtener($s_1$, $e$.joinPorCampoNat)$\big)\big)\Big)$\\
            \bigpar{
              siguiente$\Big($obtener$\big(n$, obtener$\big(s_2$, obtener($s_1$, $e$.joinPorCampoNat)$\big)\big)\Big)$ $\in$ \\
              obtener$\big(s_2$, obtener($s_1$, $e$.registrosDelJoin)$\big)$
            }
          }
        } $\land$ \\ \\
        $\big(\forall$ $s_1$ : string, $s_1$ $\in$ claves($e$.joinPorCampoString)$\big)$\\
        \bigpar{
          $\big(\forall$ $s_2$ : string, $s_2$ $\in$ claves$\big($obtener($s_1$, $e$.joinPorCampoString)$\big)\big)$\\
          \bigpar{
            $\Big(\forall$ $n$ : string, def?$\big(n$, obtener$\big(s_2$, obtener($s_1$, $e$.joinPorCampoString)$\big)\big)\Big)$\\
            \bigpar{
              siguiente$\Big($obtener$\big(n$, obtener$\big(s_2$, obtener($s_1$, $e$.joinPorCampoString)$\big)\big)\Big)$ $\in$ \\
              obtener$\big(s_2$, obtener($s_1$, $e$.registrosDelJoin)$\big)$
            }
          }
        }
      } $\land$\\
   
    \noindent\tab\tab
      \bigpar{
        \textbf{11)}
        $\big(\forall$ $s_1$ : string, $s_1$ $\in$ claves($e$.registrosDelJoin)$\big)$\\
        \bigpar{
          $\big(\forall$ $s_2$ : string, $s_2$ $\in$ claves$\big($obtener($s_1$, $e$.registrosDelJoin)$\big)\big)$\\
          \bigpar{
            $\big(\forall$ $r$ : registro, $r$ $\in$ obtener$\big(s_2$, obtener($s_1$, $e$.registrosDelJoin)$\big)\big)$\\
            \bigpar{
              def?$\big(s_2$, obtener($s_1$, $e$.joinPorCampoNat)$\big)$ $\implies$\\
              \bigpar{
                $\Big(\exists$ $n$ : nat, def?$\big(n$, obtener$\big(s_2$, obtener($s_1$, $e$.joinPorCampoNat)$\big)\big)\Big)$\\
                siguiente$\Big($obtener$\big(n$, obtener$\big(s_2$, obtener($s_1$, $e$.joinPorCampoNat)$\big)\big)\Big)$ $\igobs$ $r$
              }
            } $\land$\\
            \bigpar{
              def?$\big(s_2$, obtener($s_1$, $e$.joinPorCampoString)$\big)$ $\implies$\\
              \bigpar{
                $\Big(\exists$ $n$ : string, def?$\big(n$, obtener$\big(s_2$, obtener($s_1$, $e$.joinPorCampoString)$\big)\big)\Big)$\\
                siguiente$\Big($obtener$\big(n$, obtener$\big(s_2$, obtener($s_1$, $e$.joinPorCampoString)$\big)\big)\Big)$ $\igobs$ $r$
              }
            }
          }
        }
      } $\land$\\

      \noindent\tab\tab
      \bigpar{
        \textbf{12)}
        $\big(\forall$ $s_1$ : string, $s_1$ $\in$ claves($e$.hayJoin)$\big)$\\
        \bigpar{
          $\big(\forall$ $s_2$ : string, $s_2$ $\in$ claves$\big($obtener($s_1$, $e$.hayJoin)$\big)\big)$\\
          \bigpar{
            $\Big(\forall$ $t$ : tupla(registro, bool), esta?$\Big(t$, obtener$\big(s_2$, obtener($s_1$, $e$.hayJoin)$\big)$.cambiosT1$\Big)\Big)$\\
            \bigpar{
              campos$\big(\Pi_1$($t$)$\big)$ $\igobs$ campos$\big($obtener($s_1$, $e$.nombreATabla)$\big)$ $\land$ \\
              estaAgregado?$\Big(\Pi_1$($t$), obtener$\big(s_2$, obtener($s_1$, $e$.hayJoin)$\big)$.cambiosT2$\Big)$ $\ssi$ \\
              $\Pi_1$($t$) $\in$ registros$\big($obtener($s_1$, $e$.nombreATabla)$\big)$
            }
          } $\land$ \\
          \bigpar{
            $\Big(\forall$ $t$ : tupla(registro, bool), esta?$\Big(t$, obtener$\big(s_2$, obtener($s_1$, $e$.hayJoin)$\big)$.cambiosT2$\Big)\Big)$\\
            \bigpar{
              campos$\big(\Pi_1$($t$)$\big)$ $\igobs$ campos$\big($obtener($s_2$, $e$.nombreATabla)$\big)$ $\land$ \\
              estaAgregado?$\Big(\Pi_1$($t$), obtener$\big(s_2$, obtener($s_1$, $e$.hayJoin)$\big)$.cambiosT2$\Big)$ $\ssi$ \\
              $\Pi_1$($t$) $\in$ registros$\big($obtener($s_2$, $e$.nombreATabla)$\big)$
            }
          }
        }
      }

  \medskip
  \medskip

  \tadOperacion{sinRepetidos}{secu(string)}{bool}{}
  \tadAxioma{sinRepetidos($ls$)}
  {\IF vacia?($ls$) THEN
    true 
  ELSE
    $\neg$esta?$\big($prim($ls$), fin($ls$)$\big)$ $\yluego$ sinRepetidos$\big($fin($ls$)$\big)$
  FI}

  ~

  \tadOperacion{estaAgregado?}{registro/r, secu$\big($tupla(registro$\text{,}$ bool)$\big)$/s}{bool}{esta?$\big($<$r$, true>, $s\big)$ $\lor$ esta?$\big($<$r$, false>, $s\big)$}
  \tadAxioma{estaAgregado?($t$, $s$)}
  {\IF $\Pi_1\big($ult($s$)$\big)$ $\igobs$ $r$ THEN
    $\Pi_2\big($ult($s$)$\big)$ 
  ELSE
    estaAgregado?$\big(t$, com($s$)$\big)$
  FI}

  ~

\medskip
 
  \Abs[estr]{base}[e]{b}{tablas($b$) $\igobs$ $\big($claves($e$.nombreATabla)$\big)$ $\land$ \\
  $\Big($($\forall$ $s$ : string) $\big(s$ $\in$ claves($e$.nombreATabla)$\big)$ $\implies$ $\big($(dameTabla($s$, $b$) $\igobs$ obtener($s$, $e$.nombreATabla)$\big)\Big)$ $\land$ $\Big($($\forall$ $s_1$, $s_2$ : string) $\big(s_1$ $\in$ claves($e$.nombreATabla) $\land$ $s_2$ $\in$ claves($e$.nombreATabla)$\big)$ $\implies$ $\big($hayJoin?($s_1$, $s_1$, $b$) $\igobs$ def?$\big(s_2$, obtener($s_1$, $e$.hayJoin)$\big)\big)$ $\yluego$ campoJoin($s_1$, $s_2$, $b$) $\igobs$ $\Pi_1\big($obtener$\big(s_2$, $\big($obtener($s_1$, $e$.hayJoin)$\big)\big)\big)$ $\Big)$}

  ~

\end{Representacion}

\bigskip

\begin{Algoritmos}

\medskip
	
 \Titulo{Algoritmos de base}
  	\medskip
  
\begin{algorithm}[H]{\textbf{iNuevaBDD}() $\to$ $res$ : base}
    	\begin{algorithmic}
        \State $res.tablaMasAccedida \gets NULL$              \Comment $O(1)$
        \State $res.nombreATabla \gets Vacio()$               \Comment $O(1)$
        \State $res.tablas \gets Vacia()$                     \Comment $O(1)$
        \State $res.hayJoin \gets Vacio()$                    \Comment $O(1)$
        \State $res.joinPorCampoNat \gets Vacio()$            \Comment $O(1)$
        \State $res.joinPorCampoString \gets Vacio()$         \Comment $O(1)$
        \State $res.registrosDelJoin \gets Vacio()$           \Comment $O(1)$


        \medskip
        \Statex \underline{Complejidad:} {$O(1)$}
        \Statex \underline{Justificaci�n:} {Todas las funciones llamadas tienen complejidad $O(1)$.}
      \end{algorithmic}
\end{algorithm}

\begin{algorithm}[H]{\textbf{iAgregarTabla}(\In{t}{tabla}, \Inout{e}{estr})}
      \begin{algorithmic}
        \comentario{Si no hay tabla m�s accedida o la tabla que agregue tiene m�s accesos que la m�s accedida de la bdd...}
        \If{$e.tablaMasAccedida = NULL \oluego CantidadDeAccesos\big(nombreATabla(*e.tablaMasAccedida)\big) < CantidadDeAccesos(t)$}               \Comment $O(1)$
          \State $e.tablaMasAccedida \gets \&Nombre(t)$               \Comment $O(1)$
        \EndIf
        \State $ $

        \comentario{Agrego la tabla a todos lados}
        \State $Definir\big(Nombre(t), t, e.nombreATabla\big)$     \Comment $O(|nombre(t)|) = O(1)$ 
        \State $AgregarRapido\big(nombre(t), e.tablas\big)$       \Comment $O(|nombre(t)|) = O(1)$
        \State $Definir\big(Nombre(t), Vacio(), e.hayJoin\big)$\Comment $O(|nombre(t)|) = O(1)$
        \State $Definir\big(Nombre(t), Vacio(), e.joinPorCampoNat\big)$    \Comment $O(|nombre(t)|) = O(1)$
        \State $Definir\big(Nombre(t), Vacio(), e.joinPorCampoString\big)$   \Comment $O(|nombre(t)|) = O(1)$
        \State $Definir\big(Nombre(t), Vacio(), e.registrosDelJoin\big)$   \Comment $O(|nombre(t)|) = O(1)$


        \medskip
        \Statex \underline{Complejidad:} {$O(1)$}
        \Statex \underline{Justificaci�n:} {Por estar acotados los nombres de las tablas, Definir en diccString con nombres por clave se hace en $O(1)$.}
      \end{algorithmic}
\end{algorithm}

\begin{algorithm}[H]{\textbf{iInsertarEntrada}(\In{r}{registro}, \In{t}{String}, \Inout{e}{estr})}
     \begin{algorithmic}
       \State $tabla \gets Obtener(t, e.nombreATabla)$    \Comment $O(1)$
       \State $AgregarRegistro(r, tabla)$                          \Comment $O(|L|\ +\ log\ n)\ indexado\ /\ O(|L|)\ no\ indexado$
       \comentario{Me fijo si cambi� la tabla m�s accedida}
       \State $tabMax \gets Obtener\big(*(e.tablaMasAccedida), e.nombreATabla\big)$    \Comment $O(1)$
       \If{$CantidadDeAccesos(tabla) > CantidadDeAccesos(tabMax)$}            \Comment $O(1)$
         \State $e.tablaMasAccedida \gets \&t$                   \Comment $O(1)$
       \EndIf
       \State $ $
 
       \comentario{Lo tengo que agregar a cambios con las tablas que tenga join}
       \State $iter \gets VistaDicc\big(Obtener(t, e.hayJoin)\big)$ \Comment $O(1)$
       \While{$HaySiguiente?(iter)$}      \Comment $O(\#tablas...)$          
           \State $AgregarAtras\big(<r, true>, (Siguiente(iter).significado).cambiosT1\big)$\\ \Comment $O\big(copy(r)\big) = O\big(\#campos * dato\ mas\ costoso\big) = O(L)$
           \State $Avanzar(iter)$         \Comment $O(1)$
       \EndWhile
 
       \State $iter \gets CrearIt\big(e.tablas\big)$ \Comment $O(1)$
       \While{$HaySiguiente?(iter)$}      \Comment $O(\#tablas...)$  
               \If{$Def?\big(t, Obtener(Siguiente(iter), e.hayJoin)\big)$}            \Comment $O(1)$
                \State $cambios \gets Obtener\big(t, Obtener(Siguiente(iter), e.hayJoin)\big)$                   \Comment $O(1)$
 
           \State $AgregarAtras\big(<r, true>, (cambios.significado).cambiosT2\big)$\\ \Comment $O\big(copy(r)\big) = O\big(\#campos * dato\ mas\ costoso\big) = O(L)$
           
       \EndIf
       \State $Avanzar(iter)$         \Comment $O(1)$
 
       \EndWhile
 
       \medskip
 
\Statex \underline{Complejidad:} {\\
\quad\quad $O\big(|L|\ +\ log\ n\ +\ \#tablas * |L|\big)$ = $O\big(log\ n\ +\ |L| * (\#tablas + 1)\big)$ = $O\big(log\ n\ +\ |L| * \#tablas(b)\big)$\\
\quad\quad Donde $L$ es el dato string m�s largo de $r$ y $n$ es la cantidad de registros en la tabla.}
\Statex \underline{Justificaci�n:} {\\
\quad\quad Por interfaz de Tabla, agregar el registro a la tabla indicada cuesta $O\big(|L|\ +\ log\ n)\big)$ si hay alg�n campo indexado, y si no, $O(|L|)$. \\
\quad\quad Obtener la tabla m�s accedida a partir de su nombre cuesta $O(Nombre\ mas\ largo\ de\ tabla\ de\ la\ base)$, pero como est�n acotadas en longitud de nombre eso equivale a $O(1)$. \\
\quad\quad Las operaciones \& y * para el tipo primitivo puntero cuestan $O(1)$. \\
       \quad\quad El puntero al nombre de la tabla m�s accedida se asigna por referencia en $O(1)$. \\
       \quad\quad VistaDicc exporta complejidad $O(1)$. \\
               \quad\quad CrearIterador a tablas es $O(1)$. \\
 
       \quad\quad En el peor caso se agrega por copia el registro a la lista de cambios de todas las dem�s tablas (asumiendo que tiene joins con todas) dos veces. Eso equivale a $O(\#campos * dato\ mas\ costoso\ de\ copiar)$ por cada inserci�n, pero como los registros tienen cantidad de campos acotados, se reduce la complejidad a $O(L)$. Por lo tanto el ciclo cuesta $O\big(L * \#tablas(b)\big)$.}
     \end{algorithmic}

\end{algorithm}

\begin{algorithm}[H]{\textbf{iBorrar}(\In{cr}{registro}, \In{t}{string}, \Inout{e}{estr})}
  \begin{algorithmic}
    \State $tabla \gets Obtener(t, e.nombreATabla)$    \Comment $O(1)$
    \State $BorrarRegistro(cr, tabla)$ \Comment $Campo\ indexado\ \implies\ O(log\ n + L)\ /\ Campo\ no\ indexado \implies O(n * |L|)$
    \State $tabMax \gets Obtener\big(*(e.tablaMasAccedida), e.nombreATabla\big)$    \Comment $O(1)$
    \If{$CantidadDeAccesos(tabla) > CantidadDeAccesos(tabMax)$}       \Comment $O(1)$
      \State $e.tablaMasAccedida \gets \&t$                \Comment $O(1)$
    \EndIf
    \State $iter \gets VistaDicc\big(Obtener(t, e.hayJoin)\big)$   \Comment $O(1)$
    \While{$HaySiguiente?(iter)$}          \Comment $O(\#tablas * ...)$
      \State $AgregarAtras\big(<cr, false>, (Siguiente(iter).significado).cambiosT1\big)$\\         \Comment $O\big(copy(r)\big) = O(\#campos * dato\ mas\ costoso) = O(L)$
      \State $Avanzar(iter)$           \Comment $O(1)$
    \EndWhile
    \State $iter \gets CrearIt\big(e.tablas)\big)$   \Comment $O(1)$
    \While{$HaySiguiente?(iter)$}          \Comment $O(\#tablas * ...)$
      \If{$Def?\big(t, Obtener(siguiente(iter), e.hayJoin)\big)$}       \Comment $O(1)$
        \State $cambios \gets Obtener\big(t, Obtener(siguiente(iter), e.hayJoin)\big)$                \Comment $O(1)$
        \State $agregarAtras\big(<r,false>, (cambios.significado).cambiosT2\big)$                \Comment $O(L)$
      \EndIf
      \State $Avanzar(iter)$           \Comment $O(1)$
    \EndWhile


    \medskip
    \Statex \underline{Complejidad:} {\\
    \quad\quad Campo indexado $\implies$ $O\big(log\ n\ +\ |L|\ + \#tablas * |L|\big)$ = $O\big(log\ n\ +\ |L| * (\#tablas + 1)\big)$ = $O\big(log\ n\ +\ |L| * \#tablas\big)$ \\
    \quad\quad Campo no indexado $\implies$ $O\big(n * |L| + \#tablas * |L|\big)$ = $O\big(|L| * (n + \#tablas)\big)$
    }
    \Statex \underline{Justificaci�n:} {\\
    \quad\quad Obtener la tabla m�s accedida a partir de su nombre cuesta $O(Nombre\ mas\ largo\ de\ tabla\ de\ la\ base)$, pero como est�n acotadas en longitud de nombre, eso equivale a $O(1)$. \\
    \quad\quad Las operaciones \& y * para el tipo primitivo puntero cuestan $O(1)$. \\
    \quad\quad El puntero al nombre de la tabla m�s accedida se asigna por referencia en $O(1)$.\\
    \quad\quad VistaDicc exporta complejidad $O(1)$. \\
    \quad\quad En el peor caso se agrega por copia el registro a la lista de cambios de todas las dem�s tablas (asumiendo que tiene joins con todas). Eso equivale a $O(\#campos * dato\ mas\ costoso\ de\ copiar)$ por cada inserci�n, pero como los registros tienen cantidad de campos acotados, se reduce la complejidad a $O(L)$. Por lo tanto el ciclo cuesta $O\big(L * \#Tablas(b)\big)$. \\
    \quad\quad BorrarRegistro exporta complejidad distinta dependiendo de si hay �ndice sobre el campo criterio y se suma al resto diferenciando cada caso.
    }
  \end{algorithmic}
\end{algorithm}



\begin{algorithm}[H]{\textbf{iCombinarRegistros}(\In{t_1}{string}, \In{t_2}{string}, \In{c}{campo}) $\to$ $res$ : conj(registro)}
\begin{algorithmic}
   \State $tabla1 \gets Obtener(t_1, e.nombreATabla) $ \Comment $O(1)$
   \State $tabla2 \gets Obtener(t_2, e.nombreATabla) $ \Comment $O(1)$
   \If{$Pertenece?\big(Indices(tabla1), c\big)$}  \Comment $O(2 * |max\ nombre\ de\ campo|) = O(1)$
     \State $tablaIt \gets tabla2$  \Comment $O(1)$
     \State $tablaBusq \gets tabla1$  \Comment $O(1)$
   \Else
     \State $tablaIt \gets tabla1$    \Comment $O(1)$
     \State $tablaBusq \gets tabla2$   \Comment $O(1)$
   \EndIf
   \State $res \gets Vacio()$    \Comment $O(1)$
   \State $it \gets CrearIt\big(Registros(tablaIt)\big)$   \Comment $O(1)$
   \While{$HaySiguiente(it)$}  \Comment $O(n * ...)$
     \State $d \gets Obtener\big(c, Siguiente(it)\big)$     \Comment $O\big(|max\ nombre\ de\ campo|\ +\ L\big) = O(L)$
      \State $coincis \gets Buscar(c, d, tablaBusq)$  \Comment $c\ campo\ string:\ Si\ esta\ indexado\ \implies\ O(|L|),\ si\ no\ O\big(m * |L|\big)$
     \State $ $ \Comment $c\ campo\ nat:\ Si\ esta\ indexado\ \implies\ O\big(log\ m\ +\ |L|\big),\ si\ no\ esta\ indexado\ \implies\ O\big(m * |L|\big)$
     \If{$\NOT Vacia?(coincis)$} \Comment $O(1)$
       \If{$nombre(tablaBusq) = t_1$} \Comment $O(1)$
         \State $regMergeado \gets Merge\big(Prim(coincis), Siguiente(it)\big)$ \Comment $O(L)$
       \Else
         \State $regMergeado \gets Merge\big(Siguiente(it), Prim(coincis)\big)$ \Comment $O(L)$
       \EndIf
     \EndIf
     \State $AgregarRapido(regMergeado, res)$ \Comment $O(1)$
     \State $Avanzar(it)$ \Comment $O(1)$
   \EndWhile
 
   \medskip
   \Statex \underline{Complejidad:} {\\
   \quad\quad $O(n * buscar\ +\ L)$ = \\
   \quad\quad Si $c$ est� indexado en alguna de las tablas y es tipo string $\implies$ $O(n * |L|)$ = $O\big((n+m) * |L|\big)$ \\
   \quad\quad Si $c$ est� indexado y es nat $\implies$ $O\big(n * (log\ m\ +\ |L|)\big)$ = $O\big((n*m)(log(m)\ +\ |L|)\big)$ \\
   \quad\quad Si no est� indexado $\implies$ $O(n * m * |L|)$ \\
   \\
   \quad\quad Siendo $|L|$ el mayor largo de string entre registros, $n$ y $m$ la cantidad de registros de ambas tablas (var�an en base a criterio de b�squeda indexado o no).\\
   \quad\quad Esta indeterminaci�n se puede salvar considerando que las cantidades de registros de las tablas (que ahora s� ser�an n y m) pueden ser tomadas como $O\big(max(n,m)\big)$ = $O(n+m)$.\\
   \quad\quad La cantidad de registros mergeados est� acotada tanto por $m$ como por $n$ por ser intersecci�n.}
   \Statex \underline{Justificaci�n:} {\\
  \quad\quad Siempre se itera linealmente una tabla, y se realizan b�squedas sobre la otra (complejidad variable exportada por buscar de tabla, seg�n campo indexado o no).}
  \end{algorithmic}
\end{algorithm}





=======
\end{algorithm}

\begin{algorithm}[H]{\textbf{iBorrar}(\In{cr}{registro}, \In{t}{string}, \Inout{e}{estr})}
  \begin{algorithmic}
    \State $tabla \gets Obtener(t, e.nombreATabla)$    \Comment $O(1)$
    \State $BorrarRegistro(cr, tabla)$ \Comment $Campo\ indexado\ \implies\ O(log\ n + L)\ /\ Campo\ no\ indexado \implies O(n * |L|)$
    \State $tabMax \gets Obtener\big(*(e.tablaMasAccedida), e.nombreATabla\big)$    \Comment $O(1)$
    \If{$CantidadDeAccesos(tabla) > CantidadDeAccesos(tabMax)$}       \Comment $O(1)$
      \State $e.tablaMasAccedida \gets \&t$                \Comment $O(1)$
    \EndIf
    \State $iter \gets VistaDicc\big(Obtener(t, e.hayJoin)\big)$   \Comment $O(1)$
    \While{$HaySiguiente?(iter)$}          \Comment $O(\#tablas * ...)$
      \State $AgregarAtras\big(<cr, false>, (Siguiente(iter).significado).cambiosT1\big)$\\         \Comment $O\big(copy(r)\big) = O(\#campos * dato\ mas\ costoso) = O(L)$
      \State $Avanzar(iter)$           \Comment $O(1)$
    \EndWhile
    \State $iter \gets CrearIt\big(e.tablas)\big)$   \Comment $O(1)$
    \While{$HaySiguiente?(iter)$}          \Comment $O(\#tablas * ...)$
      \If{$Def?\big(t, Obtener(siguiente(iter), e.hayJoin)\big)$}       \Comment $O(1)$
        \State $cambios \gets Obtener\big(t, Obtener(siguiente(iter), e.hayJoin)\big)$                \Comment $O(1)$
        \State $agregarAtras\big(<r,false>, (cambios.significado).cambiosT2\big)$                \Comment $O(L)$
      \EndIf
      \State $Avanzar(iter)$           \Comment $O(1)$
    \EndWhile


    \medskip
    \Statex \underline{Complejidad:} {\\
    \quad\quad Campo indexado $\implies$ $O\big(log\ n\ +\ |L|\ + \#tablas * |L|\big)$ = $O\big(log\ n\ +\ |L| * (\#tablas + 1)\big)$ = $O\big(log\ n\ +\ |L| * \#tablas\big)$ \\
    \quad\quad Campo no indexado $\implies$ $O\big(n * |L| + \#tablas * |L|\big)$ = $O\big(|L| * (n + \#tablas)\big)$
    }
    \Statex \underline{Justificaci�n:} {\\
    \quad\quad Obtener la tabla m�s accedida a partir de su nombre cuesta $O(Nombre\ mas\ largo\ de\ tabla\ de\ la\ base)$, pero como est�n acotadas en longitud de nombre, eso equivale a $O(1)$. \\
    \quad\quad Las operaciones \& y * para el tipo primitivo puntero cuestan $O(1)$. \\
    \quad\quad El puntero al nombre de la tabla m�s accedida se asigna por referencia en $O(1)$.\\
    \quad\quad VistaDicc exporta complejidad $O(1)$. \\
    \quad\quad En el peor caso se agrega por copia el registro a la lista de cambios de todas las dem�s tablas (asumiendo que tiene joins con todas). Eso equivale a $O(\#campos * dato\ mas\ costoso\ de\ copiar)$ por cada inserci�n, pero como los registros tienen cantidad de campos acotados, se reduce la complejidad a $O(L)$. Por lo tanto el ciclo cuesta $O\big(L * \#Tablas(b)\big)$. \\
    \quad\quad BorrarRegistro exporta complejidad distinta dependiendo de si hay �ndice sobre el campo criterio y se suma al resto diferenciando cada caso.
    }
  \end{algorithmic}
\end{algorithm}





\begin{algorithm}[H]{\textbf{iCombinarRegistros}(\In{t_1}{string}, \In{t_2}{string}, \In{c}{campo}) $\to$ $res$ : conj(registro)}
\begin{algorithmic}
   \State $tabla1 \gets Obtener(t_1, e.nombreATabla) $ \Comment $O(1)$
   \State $tabla2 \gets Obtener(t_2, e.nombreATabla) $ \Comment $O(1)$
   \If{$Pertenece?\big(Indices(tabla1), c\big)$}  \Comment $O(2 * |max\ nombre\ de\ campo|) = O(1)$
     \State $tablaIt \gets tabla2$  \Comment $O(1)$
     \State $tablaBusq \gets tabla1$  \Comment $O(1)$
   \Else
     \State $tablaIt \gets tabla1$    \Comment $O(1)$
     \State $tablaBusq \gets tabla2$   \Comment $O(1)$
   \EndIf
   \State $res \gets Vacio()$    \Comment $O(1)$
   \State $it \gets CrearIt\big(Registros(tablaIt)\big)$   \Comment $O(1)$
   \While{$HaySiguiente(it)$}  \Comment $O(n * ...)$
     \State $d \gets Obtener\big(c, Siguiente(it)\big)$     \Comment $O\big(|max\ nombre\ de\ campo|\ +\ L\big) = O(L)$
      \State $coincis \gets Buscar(c, d, tablaBusq)$  \Comment $c\ campo\ string:\ Si\ esta\ indexado\ \implies\ O(|L|),\ si\ no\ O\big(m * |L|\big)$
     \State $ $ \Comment $c\ campo\ nat:\ Si\ esta\ indexado\ \implies\ O\big(log\ m\ +\ |L|\big),\ si\ no\ esta\ indexado\ \implies\ O\big(m * |L|\big)$
     \If{$\NOT Vacia?(coincis)$} \Comment $O(1)$
       \If{$nombre(tablaBusq) = t_1$} \Comment $O(1)$
         \State $regMergeado \gets Merge\big(Prim(coincis), Siguiente(it)\big)$ \Comment $O(L)$
       \Else
         \State $regMergeado \gets Merge\big(Siguiente(it), Prim(coincis)\big)$ \Comment $O(L)$
       \EndIf
     \EndIf
     \State $AgregarRapido(regMergeado, res)$ \Comment $O(1)$
     \State $Avanzar(it)$ \Comment $O(1)$
   \EndWhile
 
   \medskip
   \Statex \underline{Complejidad:} {\\
   \quad\quad $O(n * buscar\ +\ L)$ = \\
   \quad\quad Si $c$ est� indexado en alguna de las tablas y es tipo string $\implies$ $O(n * |L|)$ = $O\big((n+m) * |L|\big)$ \\
   \quad\quad Si $c$ est� indexado y es nat $\implies$ $O\big(n * (log\ m\ +\ |L|)\big)$ = $O\big((n*m)(log(m)\ +\ |L|)\big)$ \\
   \quad\quad Si no est� indexado $\implies$ $O(n * m * |L|)$ \\
   \\
   \quad\quad Siendo $|L|$ el mayor largo de string entre registros, $n$ y $m$ la cantidad de registros de ambas tablas (var�an en base a criterio de b�squeda indexado o no).\\
   \quad\quad Esta indeterminaci�n se puede salvar considerando que las cantidades de registros de las tablas (que ahora s� ser�an n y m) pueden ser tomadas como $O\big(max(n,m)\big)$ = $O(n+m)$.\\
   \quad\quad La cantidad de registros mergeados est� acotada tanto por $m$ como por $n$ por ser intersecci�n.}
   \Statex \underline{Justificaci�n:} {\\
  \quad\quad Siempre se itera linealmente una tabla, y se realizan b�squedas sobre la otra (complejidad variable exportada por buscar de tabla, seg�n campo indexado o no).}
  \end{algorithmic}
\end{algorithm}





>>>>>>> f4c8557bfab2483cedf8199b11946af8d4db5e5f
\begin{algorithm}[H]{\textbf{iGenerarVistaJoin}(\In{t_1}{string}, \In{t_2}{string}, \In{c}{campo}, \Inout{e}{estr}) $\to$ $res$ : itConj(registro)}
 \begin{algorithmic}
   \comentario{Creo en el diccionario hayJoin de t1 la otra tabla}
   \State $aux \gets <c, Vacio(), Vacio()>$   \Comment $O(1)$
   \State $Definir(t_2, aux, Obtener(t_1, e.hayJoin))$ \Comment $O\big(|maximo\ nombre\ de\ tabla|\big) = O(1)$
   \State $Definir\big(t_2, Vacio(), Obtener(t_1, e.registrosDelJoin)\big)$   \Comment $O\big(|maximo\ nombre\ de\ tabla|\big) = O(1)$
   \State $tabla1 \gets Obtener(t_1, e.nombreATabla)$ \Comment $O(1)$
   \State $tabla2 \gets Obtener(t_2, e.nombreATabla)$ \Comment $O(1)$
   \State $ $
   \comentario {Si es nat el campoJoin...}
   \If{$TipoCampo(c, tabla1)$} \Comment $O(1)$
     \comentario {Defino en el diccionario de joinPorCampoNat de t1 a t2}
     \State $Definir(t_2, Vacio(), Obtener(t_1,e.joinPorCampoNat))$ \Comment $O(1)$
     \State $regsMergeados \gets CombinarRegistros(t_1, t_2, c)$ \Comment $c\ esta\ indexado\ en\ alguna\ tabla\ \implies\ O\big(n * (log\ m\ +\ |L|)\big)$
     \State $ $                                              \Comment $Si\ no\ esta\ indexado\ \implies\ O\big(n * m * |L|\big)$
     \State $it \gets CrearIt(regsMergeados)$                \Comment $O(1)$
     \While{$HaySiguiente(it)$}                            \Comment $O(n * ...)$
       \State $d \gets Obtener\big(c, Siguiente(it)\big)$             \Comment $O(1)$
       \comentario {Agrego al conjunto de registros y al diccNat}
       \State $iter \gets AgregarRapido\big(Siguiente(it), Obtener\big(t_2,Obtener(t_1,e.registrosDelJoin)\big)\big) $ \Comment $O(L)$
       \State $n \gets ValorNat(d)$    \Comment $O(1)$
       \State $Definir(n, iter, Obtener(t_2, Obtener(t_1, e.joinPorCampoNat)))$ \Comment $O(log\ n)$
       \State $Avanzar(it)$ \Comment $O(1)$
     \EndWhile
   \Else
     \comentario {Defino en el diccionario de joinPorCampoStr de t1 a t2}
     \State $Definir\big(t_2, Vacio(), Obtener(t_1,e.joinPorCampoString)\big)$ \Comment $O(1)$
     \State $regsMergeados \gets CombinarRegistros(t_1, t_2, c)$ \Comment $c\ esta\ indexado\ en\ alguna\ tabla\ \implies\ O(n * |L|)$
     \State $ $                                             \Comment $Si\ no\ esta\ indexado\ \implies\ O\big(n * m * |L|\big)$
     \State $it \gets CrearIt(regsMergeados)$                \Comment $O(1)$
     \While{$HaySiguiente(it)$}                            \Comment $O(n * ...)$
       \State $d \gets Obtener(c, Siguiente(it))$              \Comment $O(1)$
       \comentario {Agrego al conjunto de registros y al diccString}
       \State $iter \gets AgregarRapido\big(Siguiente(it), Obtener\big(t_2, Obtener(t_1,e.registrosDelJoin)\big)\big)$ \Comment $O(L)$
       \State $s \gets valorStr(d)$                                \Comment $O(1)$
       \State $Definir\big(n, iter, Obtener\big(t_2, Obtener(t_1, e.joinPorCampoString)\big)\big)$ \Comment $O(L)$
       \State $Avanzar(it)$ \Comment $O(1)$
       \EndWhile
   \EndIf
   \State $ CrearIt\big(Obtener\big(t_2, Obtener(t_1, e.registrosDelJoin)\big)\big) $ \Comment $O(1)$
 
   \medskip
   \Statex \underline{Complejidad:}{\\
\quad\quad Si $c$ est� indexado en alguna de las dos tablas y es string $\implies$ $O\big(n * L\ +\ n * L\big)$ = $O(n * L)$ = $O\big((n+m) * L\big)$ \\
\quad\quad Si $c$ est� indexado y es nat $\implies$ $O\big(n * (log\ m\ +\ |L|) + n * (log\ n\ +\ L)\big)$ = $O\big(n * (log\ n\ +\ log\ m + L)\big)$ = $O\big(n * (log(n+m)\ +\ log(n+m) + L)\big)$ = $O\big(n * (log(n+m)\ +\ L)\big)$ = $O\big((n+m) * (log(n+m)\ +\ L)\big)$\\
\quad\quad Si $c$ no est� indexado $\implies$ $O\big(n * m * |L| + n * (log\ n + L)\big)$ = $O\big(n * (m * L\ +\ log\ n\ +\ L)\big)$ = $O\big(n * (m * L\ +\ log\ n)\big)$ \\
\\
\quad\quad Siendo $L$ el mayor largo de string entre registros, $n$ y $m$ la cantidad de registros de ambas tablas (var�an en base a criterio de b�squeda indexado o no). \\
\quad\quad Esta indeterminaci�n se puede salvar considerando que las cantidades de registros de las tablas (que ahora s� ser�an $n$ y $m$) pueden ser tomadas como $O\big(max(n,m)\big)$ = $O(n+m)$ (quedando complejidades acorde al enunciado).\\
\quad\quad La cantidad de registros mergeados est� acotada tanto por $m$ como por $n$ por ser intersecci�n. \\ 
}
   \Statex \underline{Justificaci�n:}{\\
\quad\quad Adem�s del costo por combinar registros, se agrega el de iterar todos los registros combinados, el de recorrerlos (como dijimos, est�n acotados por $n$) realizando inserciones por copia en sus respectivos diccionarios.}
 
 \end{algorithmic}
\end{algorithm}



\begin{algorithm}[H]{\textbf{iBorrarJoin}(\In{t_1}{string}, \In{t_2}{string}, \Inout{e}{estr})}
\begin{algorithmic}
   \State $Borrar\big(t_2, Obtener(t_1, e.hayJoin)\big)$             \Comment $O(1)$
   \State $Borrar\big(t_2, Obtener(t_1, e.registrosDelJoin)\big)$    \Comment $O(1)$
   \If{$Def?\big(t_2, Obtener(t_1, e.joinPorCampoNat)\big)$}         \Comment $O(1)$
     \State $Borrar\big(t_2, Obtener(t_1, e.joinPorCampoNat)\big)$   \Comment $O(1)$
   \Else
     \State $Borrar\big(t_2, Obtener(t_1, joinPorCampoString)\big)$  \Comment $O(1)$
   \EndIf
\medskip
   \Statex \underline{Complejidad:} {$O(1)$}
   \Statex \underline{Justificaci�n:} {Todas las operaciones de buscar y borrar en diccString se hacen en el orden del largo del m�ximo nombre de todas las tablas; al estar acotados estos nombres, estas operaciones se resuelven en $O(1)$.}
 \end{algorithmic}
\end{algorithm}

\begin{algorithm}[H]{\textbf{iTablas}(\In{e}{estr}) $\to$ $res$ : itLista(string)}
  \begin{algorithmic}
    \State $res \gets crearIt(e.tablas)$             \Comment $O(1)$
    

    \medskip
    \Statex \underline{Complejidad:} {$O(1)$}
    \Statex \underline{Justificaci�n:} {Crear un iterador de una lista tiene complejidad $O(1)$.}
  \end{algorithmic}
\end{algorithm}

\begin{algorithm}[H]{\textbf{iDameTabla}(\In{s}{string}, \In{e}{estr}) $\to$ $res$ : tabla}
  \begin{algorithmic}
    \State $res \gets Obtener(s, e.nombreATabla)$             \Comment $O(1)$
    
    \medskip
    \Statex \underline{Complejidad:} {$O(1)$}
    \Statex \underline{Justificaci�n:} {Los nombres de las tablas est�n acotados, por lo tanto, buscar en un diccString con nombres por claves es $O(1)$.}
  \end{algorithmic}
\end{algorithm}

\begin{algorithm}[H]{\textbf{iHayJoin?}(\In{s_1}{string}, \In{s_2}{string}, \In{e}{estr}) $\to$ $res$ : bool}
  \begin{algorithmic}
    \State $res \gets Def?\big(s_2, Obtener(s_1, e.hayJoin)\big)$             \Comment $O(1)$
    
    \medskip
    \Statex \underline{Complejidad:} {$O(1)$}
    \Statex \underline{Justificaci�n:} {Los nombres de las tablas est�n acotados, por lo tanto, buscar en un diccString con nombres por claves es $O(1)$.}
  \end{algorithmic}
\end{algorithm}

\begin{algorithm}[H]{\textbf{iCampoJoin}(\In{s_1}{string}, \In{s_2}{string}, \In{e}{estr}) $\to$ $res$ : campo}
  \begin{algorithmic}
    \State $res \gets \big(Obtener\big(s_2, Obtener(s_1, e.hayJoin)\big)\big).campoJoin$             \Comment $O(1)$
    
    \medskip
    \Statex \underline{Complejidad:} {$O(1)$}
    \Statex \underline{Justificaci�n:} {Los nombres de las tablas est�n acotados, por lo tanto, buscar en un diccString con nombres por claves es $O(1)$.}
  \end{algorithmic}
\end{algorithm}


\InterfazFuncion{Merge}{\In{r_1}{registro}, \In{r_2}{registro}}{registro}
  [true]
  {$res$ $\igobs$ copiarCampos$\big($campos($r_2$), $r_1$, $r_2\big)$}
  [$O(L)$, donde $L$ es el dato string m�s largo de $r_1$.]
  [devuelve la uni�n de dos registros, pero sin campos repetidos.]

\begin{algorithm}[H]{\textbf{iMerge}(\In{r_1}{registro}, \In{r_2}{registro}) $\to$ $res$ : registro}
  \begin{algorithmic}
    \State $res \gets Copiar(r_1)$             \comentariocompl{L = dato string m�s largo de r1}{$O\big(\#campos*(max\ nombre\ de\ campo\ +\ L)\big) = O(L)$}
    \State $ite \gets VistaDicc(r_2)$         \Comment $O(1)$
    \While{$HaySiguiente(ite)$} \Comment $O(\#campos * ...) =  O(1 * ...)$
      \If{$\NOT Def?\big(Siguiente(it).clave, res\big)$} \Comment $O(max\ nombre\ de\ campo) = O(1)$
        \State $Definir\big(Siguiente(it).clave, Siguiente(it).significado, res\big)$ \Comment $O(max\ nombre\ de\ campo\ *\ L) = O(L)$
      \EndIf
      \State $Avanzar(it)$ \Comment $O(1)$
    \EndWhile
    
    \medskip
    \Statex \underline{Complejidad:} {$O(L)$, donde $L$ es el dato string m�s largo de $r_1$.}
    \Statex \underline{Justificaci�n:} {\\
\quad\quad Copiar el registro cuesta copiar \#campos veces la clave y significados m�s costosos por interfaz de diccString. Como la cantidad de campos y los nombres de los campos est�n acotados, eso equivale a $O(L)$ siendo $L$ el dato string m�s largo de $r_1$, y por lo tanto el m�s costoso de copiar. Por los mismos motivos se itera una cantidad acotada de veces; y preguntar si un campo est� definido en un registro es tambi�n $O(1)$. \\
\quad\quad La inserci�n de cada <campo, dato> nuevo cuesta $O(max\ nombre\ de\ campo\ *\ L)$ (acotando) pero, nuevamente, los nombres de los campos est�n acotados y eso equivale al orden del dato m�s costoso.}
  \end{algorithmic}
\end{algorithm}

\begin{algorithm}[H]{\textbf{iVistaJoin}(\In{s_1}{string}, \In{s_2}{string}, \In{b}{estr}) $\to$ $res$ : itConj(registro)}
  \begin{algorithmic}

    \comentario{campito = CAMPOJOIN}
    \State $campito \gets Obtener\big(s_2, \big(Obtener(s_1,b.hayJoin)\big)\big).campoJoin$ \Comment $O(1)$
    \State $ $

    \comentario{convertimos s1 a tabla y preguntamos de qu� tipo es su campoJoin con s2}
    \State $tabla1 \gets Obtener(s_1, b.nombreATabla)$ \Comment $O(1)$
    \State $ $
    \State $esNat \gets TipoCampo?(campito,tabla1)$ \Comment $O(1)$
    \State $ $
    \State $tabla2 \gets obtener(s_2,b.nombreATabla)$ \Comment $O(1)$
    \State $ $
    \If{$esNat$}
      \comentario{Join por campo nat}
      \State $diccDeIters \gets Obtener\big(s_2, Obtener\big(s_1, b.joinPorCampoNat\big)\big)$ \Comment $O(1)$
      \State $ $
      \comentario{Creamos un iterador a la lista de cambios de tipo <registro, bool> de s1}
      \State $itT1 \gets CrearIt\Big(obtener\big(s_2,\big(obtener(s_1,b.hayJoin)\big)\big).cambiosT1\Big)$ \Comment $O(1)$
      \State $ $
      \comentario{Guardo o elimino los registros en el join}
      \comentario{Si no hay ninguno $\implies$ No actualizo nada $\implies$ O(1)}
      \While{$HaySiguiente?(itT1)$} \comentariocompl{R regs. en '.cambiosT1' de s1 y s2}{$O(R * ...)$}
        \State $tupSiguiente \gets Siguiente(itT1)$ \Comment $O(1)$
        \State $claveNat \gets Obtener(campito, tupSiguiente.reg)$ \Comment $O(1)$
        \State $ $


        \comentario{Si no existe reg en s2 que tenga el mismo valor claveNat para 'campito', no necesito ni borrar ni agregar en el join}
        \State $coincidencias \gets Buscar(campito, claveNat, tabla2)$\\ \Comment $Campo\ indexado \implies O(log\ m + |L|) promedio\ /\ Campo\ no\ indexado \implies O(m * |L|)$
        \If{$\NOT Vacia?(coincidencias)$} \Comment $O(1)$
          \comentario{CampoJoin siempre es clave, \#coincidencias es 1}
          \State $regTablaActual \gets Primero(coincidencias)$ \Comment $O(1)$

          \If{$tupSiguiente.agregar?$} \Comment $O(1)$
            \State $registroMergeado \gets Merge(tupSiguiente.reg, regTablaActual)$ \Comment $O(L)$
            \State $iter \gets AgregarRapido\big(registroMergeado, Obtener(s_2, Obtener(s_1, e.registrosDelJoin))\big)$\\ \Comment $O\big(copy(reg)\big) = O(L)$
            
      \algstore{myalg}
  \end{algorithmic}
\end{algorithm}

\begin{algorithm}
  \begin{algorithmic}
      \algrestore{myalg}

            \State $Definir(claveNat, iter, diccDeIters)$\\ \comentariocompl{m regs en s2, n regs en s1}{$O\big(log\ (n+m)\big)\ +\ O\big(copy(iter)\big) = O\big(log(n+m)\big)$}

            \Else
            \State $EliminarSiguiente\big(Obtener(claveNat, diccDeIters)\big)$ \Comment $O\big(log(n+m)\big)$
            \State $Borrar(claveNat, diccDeIters)$ \Comment $O\big(log(n+m)\big)$
          \EndIf
        \EndIf

        \State $EliminarSiguiente(itT1)$ \Comment $O(1)$
      \EndWhile

      \State $ $

      \comentario{Ahora me fijo de la tabla 2}
      \State $itT2 \gets CrearIt\Big(obtener\big(s_2,\big(obtener(s_1,b.hayJoin)\big)\big).cambiosT2\Big)$ \Comment $O(1)$
      \While{$HaySiguiente?(itT2)$} \Comment $O(R * ...)$
        \State $tupSiguiente \gets Siguiente(itT2)$ \Comment $O(1)$
        \State $claveNat \gets Obtener(campito, tupSiguiente.reg)$ \Comment $O(1)$
        \State $coincidencias \gets Buscar(campito, claveNat,tabla1)$\\ \Comment $Campo\ indexado\ \implies\ O(log\ n + |L|) promedio\ /\ Campo\ no\ indexado \implies O(n * |L|)$
        \comentario{Pregunto si esta definido para no agregar el registro dos veces}
        \If{$\NOT Vacia?(coincidencias)\ \AND \NOT Def?(claveNat, diccIters)$} \Comment $O\big(log(n+m)\big)$
          \comentario{CampoJoin siempre es clave, \#coincidencias es 1}
          \State $regTablaActual \gets Primero(coincidencias)$ \Comment $O(1)$
          \If{$tupSiguiente.agregar?$} \Comment $O(1)$
            \State $registroMergeado \gets Merge(tupSiguiente.reg, regTablaActual)$ \Comment $O(L)$
            \State $iter \gets AgregarRapido\big(registroMergeado, Obtener(s_2, Obtener(s_1, e.registrosDelJoin))\big)$\\ \Comment $O\big(copy(reg)\big) = O(L)$
            \State $Definir(claveNat, iter, diccDeIters)$ \Comment $O\big(log(n+m)\big)$

          \Else
            \State $EliminarSiguiente\big(Obtener(claveNat, diccDeIters)\big)$\\ \Comment $O\big(log(n+m)\big) + O\big(copy(iter)\big) = O\big(log(n+m)\big)$
            \State $Borrar(claveNat, diccDeIters)$ \Comment $O\big(log(n+m)\big)$
          \EndIf
        \EndIf
        \State $EliminarSiguiente(itT2)$ \Comment $O(1)$
      \EndWhile

      \State $res \gets CrearIt\big(Obtener\big(s_2, Obtener(s_1, e.registrosDelJoin)\big)\big)$ \Comment $O(1)$
    \Else
      \comentario{Join por campo string}
      \State $itT1 \gets CrearIt\Big(Obtener\big(s_2, Obtener\big(s_2, b.hayJoin\big)\big.cambiosT1\Big)$ \Comment $O(1)$
      \State $diccDeIters \gets Obtener\big(s_2, Obtener\big(s_1, b.joinPorCampoString\big)\big)$ \Comment $O(1)$
      \State $ $

      \While{$HaySiguiente?(itT1)$} \Comment $O(R * ...)$
        \State $tupSiguiente \gets Siguiente(itT1)$ \Comment $O(1)$
        \State $claveString \gets Obtener(campito, tupSiguiente.reg)$ \Comment $O(1)$
        \State $coincidencias \gets Buscar(campito, claveString, tabla2)$\\ \Comment $Campo\ indexado \implies O(|L|)\ /\ Campo\ no\ indexado \implies O(m * |L|)$
        \If{$\#coincidencias > 0$} \Comment $O(1)$

          \comentario{como campoJoin siempre es clave, \#coincidencias es 1}
          \State $regTablaActual \gets Primero(coincidencias)$ \Comment $O(1)$
          \If{$tupSiguiente.loAgrego?$} \Comment $O(1)$
            \State $registroMergeado \gets Merge(tupSiguiente.reg, regTablaActual)$ \Comment $O(L)$
            \State $iter \gets AgregarRapido\big(registroMergeado, Obtener(s_2, Obtener(s_1, e.registrosDelJoin))\big)$\\ \Comment $O\big(copy(reg)\big) = O(L)$
            \State $Definir(claveString, iter, diccDeIters)$ \Comment $O(L)\ +\ O\big(copy(iter)\big) = O(L)$
    
      \algstore{myalg}
  \end{algorithmic}
\end{algorithm}

\begin{algorithm}
  \begin{algorithmic}
      \algrestore{myalg}

          \Else
            \State $EliminarSiguiente\big(Obtener(claveString, diccDeIters)\big)$ \Comment $O(L)$
            \State $Borrar(claveString, diccDeIters)$ \Comment $O(L)$
          \EndIf
        \EndIf
        \State $EliminarSiguiente(itT1)$ \Comment $O(1)$
      \EndWhile

      \State $listCambios \gets obtener\big(s_1,\big(obtener(s_2,b.hayJoin)\big)\big).cambiosT2$ \Comment $O(1)$
      \State $itT2 \gets CrearIt(listCambios)$ \Comment $O(1)$
      \While{$HaySiguiente?(itT2)$} \Comment $O(R * ...)$
        \State $tupSiguiente \gets Siguiente(itT2)$ \Comment $O(1)$
        \State $claveString \gets Obtener(campito, tupSiguiente.reg)$ \Comment $O(1)$
        \State $coincidencias \gets Buscar(campito, claveString, tabla1)$\\ \Comment $Campo\ indexado \implies O(|L|)\ /\ Campo\ no\ indexado \implies O(n * |L|)$

        \If{$\#coincidencias > 0\ \AND\ \NOT Def?\big(claveString, diccIters\big)$} \Comment $O\big(|L|)$
          \comentario{Como campoJoin siempre es clave, \#coincidencias es 1}
          \State $regTablaActual \gets Primero(coincidencias)$ \Comment $O(1)$

          \If{$tupSiguiente.loAgrego?$} \Comment $O(1)$
            \State $registroMergeado \gets Merge(tupSiguiente.reg, regTablaActual)$ \Comment $O(L)$
            \State $iter \gets AgregarRapido\big(registroMergeado, Obtener(s_2, Obtener(s_1, e.registrosDelJoin))\big)$\\ \Comment $O\big(copy(reg)\big) = O(L)$
            \State $Definir(claveString, iter, diccDeIters)$\\ \Comment $O(L) + O\big(copy(iter)\big) = O(L)$
          \Else
            \State $EliminarSiguiente\big(Obtener(claveString, diccDeIters)\big)$ \Comment $O(L)$
            \State $Borrar(claveString, diccDeIters)$ \Comment $O(L)$
          \EndIf

        \EndIf
        \State $EliminarSiguiente(itT2)$ \Comment $O(1)$
      \EndWhile
      \State $res \gets CrearIt\big(Obtener\big(s_2, Obtener(s_1, e.registrosDelJoin)\big)\big)$ \Comment $O(1)$
    \EndIf

    \medskip
    \Statex \underline{Complejidad:} {\\
    \quad\quad Campo nat indexado $\implies$ \\
    \quad\quad $O(R)$ * $\big(O(log m + |L|)$ + $O(|L|)$ + $O\big(log(n+m)\big)\big)$ + $O(R)$ * $\Big(O(log\ n + |L|)$ + $O(|L|)$ + $O\big(log (n+m)\big)\Big)$ = $O(R)$ * ( $O(log\ m + |L|)$ + $O(|L|)$ + $O\big(log(n+m)\big)$ + $O(log\ n + |L|)$ + $O(|L|)$ + $O\big(log (n+m)\big)$ ) = \\
    \quad\quad $O(R)$ * $\big(O(|L|)$ + $O\big(log(n+m)\big)$ + $O(log m)$ + $O(log n)\big)$ =\\
    \quad\quad $O\big(R*\big(|L|+ log(n+m) + log(m) + log\ n\big)\big)$ =  $O\big(R*\big(|L| + log(n+m) + log(n*m)\big)\big)$ =\\
    \quad\quad $O\big(R*\big(|L| + log(n*m)\big)\big)$\\
    \\
    \quad\quad Campo nat no indexado $\implies$ \\
    \quad\quad $O(R)$ * $\big(O(m*|L|)$ + $O(|L|)$ + $O\big(log(n+m)\big)\big)$ + $O(R)$ * $\big(O(n*|L|)$ + $O(|L|)$ + $O\big(log (n+m)\big)\big)$ =\\
    \quad\quad $O(R)$ * $\big(O(m*|L|)$ + $O(|L|)$ + $O\big(log(n+m)\big)$ + $O(n*|L|)$ + $O(|L|)$ + $O\big(log (n+m)\big)\big)$ =\\
    \quad\quad $O\big(R*(m*|L| + |L|) + log(n+m) + n*|L| + |L| + log (n+m)\big)$ =\\
    \quad\quad $O\big(R*\big(|L|*(m+n+2) + log(n+m) + log(n+m)\big)\big)$ = \\
    \quad\quad $O\big(R*\big(|L|*(m+n) + log(n+m)\big)\big)$\\
    \\
    \quad\quad Campo string indexado $\implies$ \\
    \quad\quad $O(R)$ * $O(|L|)$ + $O(R)$ * $O(|L|)$ = $O(R*|L|)$\\
    \\
    \quad\quad Campo string no indexado $\implies$ $O(R)$ * $\big(O(m * |L|)$ + $O(|L|)\big)$ + $O(R)$ * $\big(O(n * |L|)$ + $O(|L|)\big)$ = $O\big(R * (m*|L| + |L| + n*|L| + |L|)\big)$ = $O\big(R*(|L|(n+m+2))\big)$ = $O\big(R*|L|*(n+m)\big)$\\

    \quad\quad $L$, cota para toda longitud de dato string en las dos tablas\\
    \quad\quad $n$ y $m$, cantidad de registros de las tablas con nombre s1 y s2 respectivamente\\
    \quad\quad $R$, cantidad de registros a 'actualizar' (uni�n de las listas de cambios de ambas tablas)\\

    }

      \algstore{myalg}
  \end{algorithmic}
\end{algorithm}

\begin{algorithm}
  \begin{algorithmic}
      \algrestore{myalg}

    \Statex \underline{Justificaci�n:} {\\
\quad\quad Por cada uno de los $R$ registros a actualizar se determina si se borran o se agregan. En peor caso se agregan (para borrar solo hace falta eliminar el siguiente de cada iterador y luego borrar la clave del diccionario): buscan coindencias en la otra tabla (complejidad var�a seg�n caso str/nat, indexado/no indexado), si las hay se debe hacer el merge en $O(L)$ e insertar al conjunto de registros $\big(O(|L|)$ para agregar registros por copia, por tener nombres y cantdidad de campos acotados, solo se paga por su m�ximo valor string una cantidad acotada de veces$\big)$. Finalmente se agrega el iterador a ese conjunto, tambi�n copiado en $O(1)$, a su respectivo diccionario de iteradores seg�n tipo de campo $\big(O(|L|)$ para campos string, $O(log (n+m))$ para campos nat, dado que en el peor caso todos los registros de ambas tablas est�n en el join$\big)$.\\
\quad\quad Se repite el proceso para los elementos de la otra tabla, pero agregando el costo de preguntar si ya fueron definidos en el procedimiento anterior $\big($tambi�n $O(|L|)$ para campos string y $O(log (n+m))$ para campos nat$\big)$
    }
  \end{algorithmic}
\end{algorithm}




\begin{algorithm}[H]{\textbf{iBusquedaCriterio}(\In{criterio}{Registro}, \In{t}{string}, \In{b}{estr}) $\to$ $res$ : conj(registro)}
 \begin{algorithmic}
    \State $tabla \gets Obtener(t, b.nombreATabla)$         \Comment $O(1)$
    \State $it \gets CrearIt(Indices(tabla))$               \Comment $O(1)$
    \State $termine \gets false$                                \Comment $O(1)$
    \State $res \gets Vacio()$                              \Comment $O(1)$
    \State $ $
    \comentario{Por si hay alg�n campo indexado para facilitar b�squeda}
    \While{$HaySiguiente(it)\ \AND \NOT termine$}    \Comment $O(\#regs * ...) = O(1 * ...)$
        \If{$Def?\big(Siguiente(it), crit\big)$}                   \Comment $O(1)$
            \State $campoIndice \gets Siguiente(it)$            \Comment $O(1)$
            \State $valorCampo \gets Obtener(campoIndice, crit)$    \Comment $O(1)$
            \State $coincis \gets Buscar(campoIndice, valorCampo, tabla)$ \Comment $Campo\ nat\ \implies\ O\big(log\ n + |L|\big)\ promedio$
            \State $ $  \Comment $Campo\ string\ \implies\ O(|L|)$
            \State $ $
            \State $itAux \gets CrearIt(coincis)$           \Comment $O(1)$
            \While{$HaySiguiente(itAux)$}                 \Comment $O(\#coincis * ...)$
                \If{$CoincidenTodosCrit\big(crit, Siguiente(itAux)\big)$} \Comment $O(L)$
                    \State $AgregarRapido\big(res, Siguiente(itAux)\big)$ \Comment $O(L)$
                \EndIf
                \State $Avanzar(itAux)$ \Comment $O(1)$
            \EndWhile  
            \State $termine \gets true$  \Comment $O(1)$
            \State $Avanzar(it)$ \Comment $O(1)$
        \EndIf 
    \EndWhile      
    \State $ $
    \comentario{No hab�a �ndices -> todos los registros contra crit}    
    \State $ $
    \If {$\NOT termine $} \Comment $O(1)$
        \State $itRegs \gets CrearIt\big(Registros(tabla)\big)$ \Comment $O(1)$
        \While{$HaySiguiente(itRegs)$}   \Comment $O(n * ...)$
            \If{$ coincidenTodosCrit\big(crit, Siguiente(itRegs)\big) $} \Comment $O(L)$
                \State $AgregarRapido\big(res, Siguiente(itRegs)\big)$ \Comment $O(L)$
            \EndIf
            \State $Avanzar(itRegs)$  \Comment $O(1)$
        \EndWhile
    \EndIf
   
   \Statex \underline{Complejidad:} {\\
   \quad\quad campoIndice nat $\implies$ $O\big(log\ n\ +\ |L|\ +\ n * |L|\big)$ = $O\big(log\ n\ +\ n * |L|\big)$. $O(log\ n\ +\ |L|)$ si adem�s es clave.\\
 
    \quad\quad campoIndice string $\implies$ $O\big(|L|\ +\ n * |L|\big)$ = $O(n * |L|)$. $O(|L|)$ si adem�s es clave.\\
    \quad\quad Sin campoIndice $\implies$ $O(n * |L|)$\\       
    \quad\quad Donde $L$ es la longitud de dato string m�s largo de $t$ y $n$ su cantidad de registros.}
   
   \Statex \underline{Justificaci�n:} {\\
   \quad\quad Su campoIndice es clave, entonces \#coincis = 1, de no serlo est� acotada por $n$. \\
   \quad\quad Los registros tienen cantidad acotada de campos, copiar uno por lo tanto es el costo de sus string m�s largo.\\
   \quad\quad Operaciones Def? y Obtener en registros (con campos acotados por enunciado) es $O(1)$.}
 \end{algorithmic}
\end{algorithm}





\begin{algorithm}[H]{\textbf{iCoincidenTodosCrit}(\In{crit}{registro}, \In{r}{registro}) $\to$ $res$ : bool}
 \begin{algorithmic}
   \State $itCrit \gets VistaDicc(crit)$             \Comment $O(1)$
   \State $res \gets true$  \Comment $O(1)$
   \While{$HaySiguiente(itCrit) \AND res$}  \Comment $O(\#campos(crit) * ...)$
     \State $tuplaCrit \gets Siguiente(itCrit)$      \Comment $O(1)$
     \If{$\NOT Obtener(tuplaCrit.clave) = tuplaCrit.significado$}  \Comment $O(L)$
       \State $res \gets false$      \Comment $O(1)$
     \EndIf
     \State $Avanzar(itCrit)$         \Comment $O(1)$
   \EndWhile
   \medskip
   \Statex \underline{Complejidad:} {$O\big(\#campos(crit) * |L|\big) = O\big(|L|\big)$, donde $L$ valor string m�s largo en $r$.}
   \Statex \underline{Justificaci�n:} {\\
\quad\quad En peor caso se compara en todos los campos un dato string de m�xima longitud en $r$.\\
\quad\quad Por enunciado los registros tienen acotadas las cantidades de campos, por lo tanto el costo es el de comparar una cantidad acotada de veces el string m�s largo.}
 \end{algorithmic}
\end{algorithm}






\begin{algorithm}[H]{\textbf{iTablaMaxima}(\In{b}{estr}) $\to$ $res$ : string}
  \begin{algorithmic}
    \State $res \gets *(b.tablaMasAccedida)$             \Comment $O(1)$
    
    \medskip
    \Statex \underline{Complejidad:} {$O(1)$}
    \Statex \underline{Justificaci�n:} {El algoritmo pasa por referencia un string, por lo tanto es $O(1)$.}
  \end{algorithmic}
\end{algorithm}



\begin{algorithm}[H]{\textbf{iBusquedaCriterio}(\In{criterio}{Registro}, \In{t}{string}, \In{b}{estr}) $\to$ $res$ : conj(registro)}
  \begin{algorithmic}
    \State $tabla \gets obtener(t, b.nombreATabla)$        	\Comment $O(1)$
	\State $it \gets CrearIt(Indices(tabla))$				\Comment $O(1)$
	\State $termine \gets false$								\Comment $O(1)$
	\State $res \gets Vacio()$								\Comment $O(1)$
	\State $ $
	\comentario{Por si hay alg�n campo indexado para facilitar b�squeda}
	\While{$ HaySiguiente(it) \land �termine $}	\Comment $O(#regs*..) = O(1*..)$ 
		\If{$ Def?(Siguiente(it), crit)$}					\Comment $O(1)$
			\State $campoIndice \gets Siguiente(it)$			\Comment $O(1)$
			\State $valorCampo \gets Obtener(campoIndice, crit)$	\Comment $O(1)$
			\State $coincis \gets Buscar(campoIndice, valorCampo, tabla)$
			\Comment Campo nat  $\implies O(log(n) + |L|)$ promedio 
			\State $ $	\Comment Campo string $\implies O(|L|)  $
			\State $ $
			\State $itAux \gets CrearIt(coincis)$ 			\Comment $O(1)$
			\While{$ HaySiguiente(itAux) $}					\Comment $O(#coincis*...)$
				\If{$ coincidenTodosCrit(crit, Siguiente(itAux)) $} \Comment $O(L)$
					\State $AgregarRapido(res, Siguiente(itAux))$ \Comment $O(L)$
				\EndIf
				\State $Avanzar(itAux)$ \Comment $O(1)$
			\EndWhile	
			\State $termine \gets true$  \Comment $O(1)$
			\State $Avanzar(it)$ \Comment $O(1)$
		\EndIf	
	\EndWhile	    
	\State $ $
	\comentario{No hab�a �ndices -> todos los registros contra crit}    
	\State $ $
	\If {$ �termine $} \Comment $O(1)$
		\State $itRegs \gets CrearIt(Registros(tabla))$ \Comment $O(1)$
		\While{$ HaySiguiente(itRegs) $}   \Comment $O(n*...)$
			\If{$ coincidenTodosCrit(crit, Siguiente(itRegs)) $} \Comment $O(L)$
				\State $AgregarRapido(res, Siguiente(itRegs))$ \Comment $O(L)$
			\EndIf
			\State $Avanzar(itRegs)$  \Comment $O(1)$
		\EndWhile
	\EndIf
	
	
    \Statex \underline{Complejidad:} {
    \quad\quad
    \quad\quad \underline{campoIndice nat} $\implies O(log(n) + |L| + n*|L|) = O(log (n) + n*|L|)$, \\
    $O(log(n) + |L|)$ si adem�s es clave

	\quad\quad \underline {campoIndice string} $\implies O(|L| + n*|L|) ) = O(n*|L|)$, \\
	$O(|L|)$ si adem�s es clave
	
	\quad\quad \underline {Sin campoIndice} $\implies O(n*|L|)$	
	   
	\quad\quad Donde L es la longitud de dato string m�s largo de t y n su cantidad de registros.      
    \quad\quad
    }
    
    \Statex \underline{Justificaci�n:} {Su campoIndice es clave entonces $#coincis = 1$, de no serlo est� acotada por $n$. \\
    Los registros tienen cantidad acotada de campos, copiar uno por lo tanto es el costo de sus string m�s largo.
    Operaciones def? y obtener en registros (con campos acotados por enunciado) es $O(1)$}
  \end{algorithmic}
\end{algorithm}


\end{Algoritmos}

\newpage

\end{document}
